\section{Nachhaltigkeit}

%%%%%%%%%%%%%%% MT NACHHALTIGKEIT %%%%%%%%%%%%%%%%%%%%%
%%%%%%%%%%%%%%%%%%%%%%%%%%%%%%%%%%%%%%%%%%%%%%%%%%%%%%%



%%%%%%%%%%%%%%% ET NACHHALTIGKEIT %%%%%%%%%%%%%%%%%%%%%
%%%%%%%%%%%%%%%%%%%%%%%%%%%%%%%%%%%%%%%%%%%%%%%%%%%%%%%


%%%%%%%%%%%%%%% IT NACHHALTIGKEIT %%%%%%%%%%%%%%%%%%%%%
%%%%%%%%%%%%%%%%%%%%%%%%%%%%%%%%%%%%%%%%%%%%%%%%%%%%%%%

\subsection{Bilderkennung}

Als erster Schritt, um die Bilderkennung zu testen, musste in der Mensa ein Graph aufgeklebt werden.
Um dabei nicht so viel Klebeband und Klebepunkte zu verschwenden, wurde dies jeweils mit einem anderen Team zusammen gemacht. Zusätzlich konnten so gemeinsam die vorhandenen Hindernisse verwendet werden, damit vermieden werden kann mehrere Hindernisse zu drucken oder zu kaufen.
Nach dem Aufkleben wurden möglichst viele Bilder gemacht. Aus vielen verschiedenen Winkeln mit verschiedenen Hindernissen. So haben wir eine grosse Auswahl an möglichen Szenarien, damit dieser Versuchsaufbau so wenig wie möglich aufgebaut werden muss. Dies minimiert nicht nur unseren Zeitaufwand für zukünftige Tests, sondern auch PLACEHOLDER m Klebeband, welches nach den Tests wieder abgerissen und weggeworfen wird.

Die Bilder wurden bereits früh mit einem Raspberry Pi und eine Raspberry Pi Kamera aufgenommen. Bevor dafür etwas neues bestellt wurde, wurden Dinge verwendet, die schon besessen wurden. Es wurden ein Raspberry Board und eine Kamera verwendet, die ein Teammitglied bereits zu Hause hatte. Beide hatten ebenfalls je eine Hülle.
Um die beiden Einzelteile miteinander zu verwenden, wurden sie lediglich zusammengeklebt, damit nicht eine neue Hülle gedruckt oder gekauft werden muss, die wahrscheinlich nicht im Roboter verwendet werden wird.
Um die Höhe und die Ausrichtung der Kamera zu simulieren, wurde ein kleines Stativ verwendet, das ebenfalls schon im Besitz eines Teammitgliedes war.
Erst nachdem klar war, dass die Qualität der Bilder in Ordnung ist, wurden die neusten Versionen der Raspberry Pis bestellt. Diese wurden benötigt, da vergangene Versionen zu wenig Rechenleistung liefern.

TODO: DO WE KEEP THIS PARAGRPAH? Die Technologien, die für die Bilderkennung verwendet werden, sind alle Open Source. Open Source Software kann nachhaltiger sein, durch die Nachverfolgbarkeit der Entscheidungen und Entwickler:innen. Durch diese Transparenz wird Greenwashing\footnote{https://www.un.org/en/climatechange/science/climate-issues/greenwashing} vermieden. Ausserdem wird die benötigte Rechenpower auf einzelne User verteilt\cite{oss-sustainability}.

\subsection{Code Repositories}

Die Code Repositories befinden sich auf GitHub. Dabei gibt es für den Simulator keine GitHub Action\footnote{https://docs.github.com/en/actions/about-github-actions/understanding-github-actions}, die den Code testet. Dies bedeutet, dass nur das Minimum an Rechenleistung von den GitHub Server verwendet wird, sprich, nur diesen Speicher, der nötig ist, um Code zu speichern. Die automatischen Tests werden lokal ausgeführt.

Zusätzlich setzt sich GitHub für Nachhaltigkeit ein.\cite{github-sustainability}

Sie haben mit drei anderen Firmen die Green Software Initiative gegründet. Diese soll dabei helfen grüne Software zu erstellen.  Dabei geht es um Software Tools, die sich ihrem Kohlenstoffausstoss bewusst sind und dabei effizient handeln. Das Ziel von der Green Software Initiative ist es, Kohlenstoff zu reduzieren und nicht nur zu neutralisieren.\cite{green-software-initiative}

GitHub hat ebenfalls bei One Tree Planted mitgemacht. Deren Ziel ist es, die Umwelt wieder herzustellen, indem Bäume gepflanzt werden\cite{one-tree-planted}. Auch mit The Ocean Cleanup hat GitHub kollaboriert. Diese wollen Plastik aus den Meeren entfernen.\cite{ocean-cleanup}.

Ausserdem wird nachhaltige Software von GitHub unterstützt und gefördert. So gibt es ein GitHub Repository, dass eine extensive Liste von grüner Software hat. Sie sind in 4 Kategorien eingeteilt: Messungssoftware, Kohlenstoffeffizienz (Software, die dabei hilft Kohlenstoff auszustossen), Kohlenstoffbewusstsein und spezielle Tools. \cite{green-software}

GitHub ist seit 2019 CO2-neutral und hat auch weitere nachhaltige Ziele. 2025 soll GitHub zu 100\% auf erneuerbare Energien setzen und in 2030 sogar CO2-negativ sein.\cite{github-goals}





%%%%%%%%%%%%%%% ALLG NACHHALTIGKEIT %%%%%%%%%%%%%%%%%%%
%%%%%%%%%%%%%%%%%%%%%%%%%%%%%%%%%%%%%%%%%%%%%%%%%%%%%%%


\subsection{Dokumentation}

Damit Overleaf produktiv für unsere LaTeX Dokumentation verwendet werden konnte, wurde eine Vollversion benötigt. Damit das Budget nicht dafür verwendet werden musste, wurde eine eigene Overleaf Instanz aufgesetzt. Damit diese möglichst nachhaltig läuft, wurde sie auf einem Server aufgesetzt, der einen anderen Hauptzweck hat und bereits vor PREN immer eingeschaltet war. So wird kein zusätzlicher Strom benötigt.

Das Backup der Dokumentation ist auf einem anderen Server eingerichtet. Auch dieser Server lief bereits vorher Non-Stop und hat eigentlich einen anderen Zweck. Zusätzlich wird das Backup auf einmal am Tag beschränkt, damit nicht bei jeder Änderung ein Backup gestartet wird. Das Backup wird auf GitHub gepusht und in einer GitHub Action wird die LaTeX Dokumentation zu einem PDF gebaut. Dadurch, dass dies maximal einmal am Tag geschieht, wird die Nutzung des geteilten Speichers von GitHub gesenkt.

\subsection{Datenaustausch}

Zu Beginn des Projektes wurde in der Gruppe entschieden, dass auf jegliches Papier verzichtet wird. Der gesamte Datenaustausch findet elektronisch statt. Es wird komplett auf Flipcharts oder Notizpapier verzichtet. Alle Mitglieder besitzen ein elektronisches Tablet, welches Papier sehr einfach ersetzen kann.