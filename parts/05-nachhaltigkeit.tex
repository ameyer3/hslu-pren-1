\section{Nachhaltigkeit}

%%%%%%%%%%%%%%% MT NACHHALTIGKEIT %%%%%%%%%%%%%%%%%%%%%
%%%%%%%%%%%%%%%%%%%%%%%%%%%%%%%%%%%%%%%%%%%%%%%%%%%%%%%



%%%%%%%%%%%%%%% ET NACHHALTIGKEIT %%%%%%%%%%%%%%%%%%%%%
%%%%%%%%%%%%%%%%%%%%%%%%%%%%%%%%%%%%%%%%%%%%%%%%%%%%%%%


%%%%%%%%%%%%%%% IT NACHHALTIGKEIT %%%%%%%%%%%%%%%%%%%%%
%%%%%%%%%%%%%%%%%%%%%%%%%%%%%%%%%%%%%%%%%%%%%%%%%%%%%%%

\subsection{Bilderkennung}

- erste versuche mit raspi \& kamera mit dingen die schon besessen. (huelle, zusammen getaped anstatt 3d druckte kombination, eigentliche produkte, stativ, um unnoetige 3D gedruckte cases zu vermeiden), erst danach neue raspi version bestellt

- Graph immer mit anderen aufgeklebt und auch Hindernisse e.g mit anderen Gruppen geteilt, damit vermieden werden konnte, Barrieren selber z udrucken/kayfen

- Namal oeppis wo likely viel z wiit hergholt isch @Lukas: OpenSource Technologien, anstatt proprieaetere software zu unterstuetzen, die zu grossfirmen gehoeren, die schlechten einfluss auf environment haben (obwohl viel OSS doch auch zu firmen gehoeren..., depends on what we have iguess)  + verteilung von rechenpower auf einzelne Users 

\subsection{Code Repositories}

GitHub\cite{github-sustainability}, keine Pipeline, um weniger Ressourcen zu brauchen?

%%%%%%%%%%%%%%% ALLG NACHHALTIGKEIT %%%%%%%%%%%%%%%%%%%
%%%%%%%%%%%%%%%%%%%%%%%%%%%%%%%%%%%%%%%%%%%%%%%%%%%%%%%


\subsection{Dokumentation}

Damit Overleaf produktiv verwendet werden konnte fuer die LaTeX Dokumentation, wurde eine Vollversion benoetigt. Damit das Budget nicht dafuer verwendet werden musste, wurde eine eigene Overleaf Instanz aufgesetzt. Damit diese moeglichst nachhaltig laeuft, wurde sie auf einem Server aufgesetzt, der einen anderen Hauptzweck hat und bereits vor PREN immer eingeschaltet war. So wird kein zusaetzlicher Strom benoetigt.

Das Backup der Dokumentation ist auf einem anderen Server eingerichtet. Auch dieser Server lief bereits vorher Non-Stop und hat eigentlich einen anderen Zweck. Zusaetzlich wird das Backup beschraenkt auf einmal am Tag, damit nicht bei jeder Aenderung an ein Backup gestartet wird. Das Backup wird auf GitHub pushed und in einer GitHub Action\footnote{https://docs.github.com/en/actions/about-github-actions/understanding-github-actions} wird die LaTeX Dokumentation zu einem PDF gebaut. Dadurch, dass dies maximal einmal am Tag geschieht, wird die Nutzung des geteilten Speichers von GitHub gesenkt.

\subsection{Datenaustausch}

Zu Beginn des Projektes wurde in der Gruppe entschieden, dass auf jegliches Papier verzichtet wird. Der gesamte Datenaustausch findet elektronisch statt. Es wird komplett auf Flipcharts oder Notizpapier verzichtet. Alle Mitglieder besitzen ein elektronisches Tablet, welches Papier sehr einfach ersetzen kann.