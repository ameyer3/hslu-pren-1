\section{Nachhaltigkeit}
\label{section:Nachhaltigkeit}

Das Thema der Nachhaltigkeit wird immer relevanter und wurde aus diesem Grund in diesem Projekt berücksichtigt. In den folgenden Kapiteln wird aufgezeigt auf welche Arten nachhaltig gearbeitet wurde.

Ein wichtiger Aspekt um nachhaltig zu handeln, sind die 3 R: Reduce, Reuse und Recycle. Das Ziel dieser drei Rs ist es, weniger Abfall zu produzieren. Dabei wird ebenfalls dafuer gesorgt, dass weniger Platz fuer Entsorgungsplaetze benoetigt wird.\cite{3-r}

In den folgenden Kapiteln sind die drei Rs erlaeutert. Ebenfalls ist aufgezeit, welche Handlungen zu dem jeweiligen R im Verlauf des Semester umgesetzt wurden.

\subsection{Reduce}

Reduce handelt davon, dass vermindert werden soll, wie viel Abfall produziert werden soll. Einige Beispiele, um Reduce umzusetzen, sind, dass nur gekauft wird, was wirklich gebraucht wird oder dass Produkte mit weniger Verpackungsmaterialien gewaehlt werden. Ebenfalls hilft es wenn Produkte in groesseren Mengen eingekauft werden.

Beim Prozess der Entwicklung des Designs wurde darauf geachtet, dass nicht von Anfang an Produkte bestellt werden, die eventuell nicht verwendet werden wuerden. Es wurde zuerst ein Grundkonzept ausgearbeitet und Bestellungen wurden im Team besprochen. Zusaetzlich wurden besimmte Dinge, wie beispielsweise PLA und Endschalter in eher groesseren Mengen gekauft. Dies sind beides Dinge, die im privaten Gebrauch von Teammitgliedern ebenfalls verwendet werden. Es konnten jedoch Verpackung und Lieferwege gespart werden, indem nicht mehrere Bestellungen getaetigt wurden. 

Es werden nicht zwei komplette Roboter Protoypen erstellt, um parallel zu arbeiten. Dies ist nicht noetig, da nicht immer jeder Teil des Roboters gebracht wird, weswegen nur Teil-Prototypen erstellt werden. Dabei wird versucht so viel Material wie noetig und so wenig wie moeglich zu verwenden.

\subsubsection{Motorentest}

Der Motorentest wurde im Labor der Hochschule durchgeführt. Für den Aufbau wurde ein Steckplatine verwendet, diese kann mehrmals genutzt werden. Die nötigen elektronischen Bauteile wurden nicht extra bestellt, sondern von der Werkstatt ausgeliehen. Um den Stromverbrauch gering zu halten wurde der Motor zwischen den Tests ständig abgeschaltet. 

\subsubsection{Datenaustausch}

Zu Beginn des Projektes wurde in der Gruppe entschieden, dass auf jegliches Papier verzichtet wird. Der gesamte Datenaustausch findet elektronisch statt. Es wird komplett auf Flipcharts oder Notizpapier verzichtet. Alle Mitglieder besitzen ein elektronisches Tablet, welches Papier sehr einfach ersetzen kann.

%%%%%%%%%%%%%%%%%%%%%%%%%%%%%%%%%%%%%%%%%%%%%%%%%%%%%%%%%
%%%%%%%%%%%%%%%%%%%%%%%%%%%%%%%%%%%%%%%%%%%%%%%%%%%%%%%%%5

\subsection{Reuse}

Bei Reuse geht es darum, dass Produkte solange wie moeglich verwendet werden sollen. Produkte sollten nicht entsorgt, wenn dieso noch funktionstuechtig sind. Beispielsweise sollte eine Wasserflasche noch einmal verwendet werden, anstatt dass eine neue gekauft wird.

Sowohl in dem richtigen Roboter, als auch beim Prototyping wurden Gegenstaende wiederverwendet.

Die Raeder, die im Prototype und im richtigen Roboter eingesetzt werden, wurden ausgeliehen von der Hochschule.

Das Prototyping mit dem Ultraschall wurde mit einem Exemplar durchgefuehrt, das ein Teammitglied bereits besass. Ebenfalls wurde das Prototyping der Bilderkennung mit bereits vorhandenen Produkten begonnen (siehe Kapitel \ref{nachhaltigkeit:bilderkennung}.

Die Produkte des Prototypings wurden alle so gebaut, dass sie einfach zusammengefuegt und angepasst werden koennen. Dabei sind das Fahrwerk (siehe Kapitel \ref{nachhaltigkeit:mod}), der Kameraturm und der Greifer sehr einfach anzupassen (siehe Kapitel \ref{prototyp-products}. Der Greifer Prototyp wird ebenfalls direkt in den Roboter verbaut werden (siehe Kapitel \ref{subsubsection:gripper-prototype-2}.


\subsubsection{Modularität}\label{nachhaltigkeit:mod}

Bei der Planung des Prototyps wird bewusst auf hohe Komplexität und weitreichende Funktionsintegration der Bauteile verzichtet. Eine modulare Konstruktion aus vielen Einzelteilen hilft dabei, die Einsatzdauer der verwendeten Materialien zu verlängern und somit die graue Energie\footnote{\url{https://www.baunetzwissen.de/glossar/g/graue-energie-664290}} des Roboters zu minimieren. Bei Änderungen müssen nur einzelne Komponenten ausgetauscht oder verändert werden.

Die Grundplatte wurde gezielt aus einem leicht bearbeitbaren Material gefertigt. Dies ermöglicht eine flexible Anpassung bei späteren Änderungen. Teile können problemlos an anderen Stellen durch neue Bohrungen angebracht werden. Ebenfalls können einfach neue Teile ergänzt werden. So kann die Lebensdauer der Grundplatte und den anderen Bauteilen erhöht werden.

Leiterplatten und \acrfull{pcb} sind essentiell in der Elektronik. Jedoch sind viele der verwendeten Materialien, wie beispielsweise Blei, für die Umwelt giftig. Andere, wie der in der Fertigung eingesetzte Klebstoff, sind nicht biologisch abbaubar. Damit moeglichst wenige Leiterplatten und PCBs verwendet werden muessen und der Abfall minimiert werden kann, wird bereits in der Designphase auf Wiederverwendbarkeit geachtet. Auf diese Weise koennen Teile innerhalb dieses Projektes, wie auch im privaten Gebrauch nach dem Projekt, wiederverwendet werden. Ebenfalls sollen komplexe Formen und Aussparungen in den Leiterplatten vermieden werden, da diese zu erhöhtem Materialabfall fuehren.

Dies wird erzielt indem nicht eine grosse, monolithische Leiterplatte entwickelt wird, sondern mehrere kleinere Platinen, die jeweils auf spezifische Teilaufgaben abgestimmt sind. Ein Beispiel dafür ist die Verwendung eines separaten Moduls für den Mikroprozessor, wie etwa beim Modell \gls{tinyk22} (siehe Abbildung \ref{fig:Tiny_K22_PCB}). Dieses modulare Design ermöglicht es, einzelne Komponenten dieses Projekts auch in zukünftigen Vorhaben weiterzuverwenden, wodurch Materialressourcen effizienter genutzt und Abfälle reduziert werden können.


\subsubsection{Bilderkennung}\label{nachhaltigkeit:bilderkennung}

Die Bilder wurden bereits früh mit einem Raspberry Pi und einer Raspberry Pi Kamera aufgenommen. Bevor dafür etwas neues bestellt wurde, wurden ein Raspberry Board und eine Kamera verwendet, die ein Teammitglied bereits zu Hause hatte. Beide hatten ebenfalls je eine Hülle.
Um die beiden Einzelteile miteinander zu verwenden, wurden sie lediglich zusammengeklebt, damit nicht eine neue Hülle gedruckt oder gekauft werden muss, die wahrscheinlich nicht im Roboter verwendet werden wird.
Erst nachdem klar war, dass die Qualität der Bilder in Ordnung ist, wurden die neuste Version des Raspberry Pis, die mehr Rechenleistung liefert, bestellt. Die Bestellung wurde über den schweizer Webshop Digitec getätigt und die Hardware war bereits im lokalen Lager von Digitec. 

\subsubsection{Dokumentation}

Damit Overleaf produktiv für die LaTeX Dokumentation verwendet werden konnte, wurde eine Vollversion benötigt. Damit das Budget nicht dafür verwendet werden musste, wurde eine eigene Overleaf Instanz aufgesetzt. Damit diese möglichst nachhaltig läuft, wurde sie auf einem Server aufgesetzt, der einen anderen Hauptzweck hat und bereits vor PREN immer eingeschaltet war. So wird kein zusätzlicher Strom benötigt.

Das Backup der Dokumentation ist auf einem anderen Server eingerichtet. Auch dieser Server lief bereits vorher Non-Stop und hat eigentlich einen anderen Zweck. Zusätzlich wird das Backup auf einmal am Tag beschränkt, damit nicht bei jeder Änderung ein Backup gestartet wird. Das Backup wird auf GitHub gepusht und in einer GitHub Action wird die LaTeX Dokumentation zu einem PDF gebaut. Dadurch, dass dies maximal einmal am Tag geschieht, wird die Nutzung des geteilten Speichers von GitHub gesenkt.

 Durch die Wiederverwendung existierender Server konnte Strom gespart werden. Es waere moeglich gewesen die Overleaf Instanz und das Ausfuehren des Backups auf dem selben Server zu implementieren, jedoch war es hier in Ordnung, da zwei Server bereits existieren. Aus diesem Grund sind jedoch die folgenden Berechnung nur mit einem Server gemacht.

 Der Server, auf dem das Backup eingerichtet ist, ist ein HP ProLiant MicroServer Gen8. Laut dessen Datenblatt verbraucht dieser im Leerlauf ca. 30-35 Watt und wenn die maximale Last ausgenutzt ist ca. 50-60 Watt\.cite{proliant} Da sowohl die Overleaf Instanz, als auch das Backup keine Hochleistungsprogramme sind, wird sich der Server die meiste Zeit nahe am Leerlauf aufhalten.

Waehrend des ganzen Semester wurde eine Overleaf Instanz benoetigt. Angenommen ein HP ProLiat MicroServer Gen8 waere in der zweiten Semesterwoche (27.09.2024) aufgesetzt worden und liefe bis zu der Abgabe der Dokumentation (10.01.2025), waere er 106 Tage eingschaltet gewesen.
Dies sind 2544 Stunden. Daraus ergibt sich folgende Berechnung:

\[Energie(kWh)=Leistung(W) * Zeit(h) / 1000\]
\[Energie(kWh)=35W * 2544h / 1000 = 89.04kWh\]

 Durch die Wiederverwendung des Servers konnten 89.04 kWh Strom gespart werden.

 
\subsubsection{Ausblick}

Nachdem Ende von \acrshort{pren2} sollen die Einzelteile des Roboters aufgeteilt werden, damit sie in zukuenftigen privaten Projekten wiederverwendet werden und nicht entsorgt werden.


%%%%%%%%%%%%%%%%%%%%%%%%%%%%%%%%%%%%%%%%%%%%%%%%%%%%%%%%%
%%%%%%%%%%%%%%%%%%%%%%%%%%%%%%%%%%%%%%%%%%%%%%%%%%%%%%%%%

\subsection{Recycle}

Das dritte R steht fuer Recycle. Produkt sollen wenn immer moeglich recycled werden. Dabei sollen neue Zwecke fuer Produkte gefunden werden, anstatt diese wegzuwerfen, nur weil der usrpuengliche Nutzen nicht mehr gebraucht wird. Recycling soll sowohl auf offiziellen Wegen, mit beispielsweise Muelltrennung, als auch persoenlich, indem Produkte bewusst zweckentfremdet werden, durchgefuehrt werden. Diese Zweckentfremdung, wurde mehrmals im Prozess des Prototypings durchgefuehrt. Beispielsweise wurde fuer das Testen der ersten Greiferprototypes ein Gegenstand gesucht, der [...]g wiegt, um zu testen, ob die Barriere mit der definierten Kraft angehoben werden kann. Schlussendlich wurde ein Emmi Kaffe dafuer verwendet (siehe Abbildung REF). Ebenfalls wurden die Bilder des Graphens (siehe Kapitel REF) auf unterschiedliche Arten geschossen. Die ersten Bilder wurden mit einer Laptopwebcam erstellt. Dadurch wurde eine konstant Hoehe sichergestellt und es konnte getestet werden, ob diese Hoehe fuer das Gesamtkonzept passt. Ebenfalls wurde die Hoehe der Kamera sowohl mit Buechern, als auch mit einem kleinen Stativ getestet. Auf diese Weise wurde sichergestellt, dass die geplante Hoehe funktionieren wird, bevor ein Kameraturm gebaut werden musste, der womoeglich nicht passen wuerde.


% \subsection{Transport}

% Ein Grossteil der Elektronikkomponenten wird aus China oder Taiwan importiert. Die damit verbundenen Transportwege, per Flugzeug oder Schiff, sind in der Elektronikindustrie Realität.

% Für die Herstellung eines einzelnen Prototyps ist der Aufbau einer eigenen Industrie zur Produktion in der Schweiz wirtschaftlich nicht sinnvoll. Ebenso wurde in diesem Projekt keine Zeit verwendet, alternative Bezugsquellen bei europäischen Herstellern zu identifizieren und einzubeziehen.

% \subsection{Energie}

% Die Herstellung der einzelnen Elektronikkomponenten erfordert erhebliche Mengen an Energie. Da die Energieproduktion zu den grössten Verursachern von Umweltbelastungen gehört, ist es sinnvoll, den Energiemix der Hauptproduktionsstandorte zu berücksichtigen.

% Die Bauteile für dieses Projekt stammen überwiegend aus den USA, China\footnote{\url{https://www.bpb.de/kurz-knapp/zahlen-und-fakten/europa/75143/eu-usa-china-energiemix/}} und Taiwan\footnote{\url{https://moeaea.gov.tw/ECW/English/content/Content.aspx?menu_id=1679}}, während die Endmontage in der Schweiz erfolgt. In den Produktionsländern USA, China und Taiwan wird ein Grossteil der Energie aus fossilen Brennstoffen wie Kohle gewonnen, was erhebliche CO$_{2}$-Emissionen verursacht. Obwohl diese Form der Energiegewinnung umweltschädlich ist, stellt sie in der globalen Lieferkette eine gegenwärtige Realität dar, die sich kurzfristig nicht ändern lässt. Diese Länder bieten aufgrund ihrer etablierten Produktionsinfrastruktur und ihrer Kosteneffizienz wesentliche Vorteile, die sie zu unverzichtbaren Akteuren in der Elektronikfertigung machen.

% Im Gegensatz dazu erzeugt die Schweiz ihren Strom überwiegend aus erneuerbaren Quellen\footnote{\url{https://www.kernenergie.ch/de/schweizer-strommix-_content---1--1069.html}}, insbesondere aus Wasserkraft, wodurch die Endmontage mit einem deutlich geringeren CO$_{2}$-Ausstoss verbunden ist. Die Menge an CO$_{2}$, die durch die Energieversorgung der einzelnen Komponenten freigesetzt wurde, wird in diesem Projekt jedoch nicht untersucht.

\subsection{Material}

Es wird darauf geachtet, dass an moeglichst vielen Orten \acrfull{pla} verwendet wird. PLA ist ein Biokunststoff und wird aus nachwachsenden Ressourcen gemacht, wie Mais oder Zuckerrohr.\cite{pla} Im Vergleich zu Stahl, Aluminum oder ABS sind die CO2 Emissionen bei der Herstellung deutlich kleiner, ebenso der Energieverbrauch.

\begin{table}[H]
\centering
\begin{tabularx}{\textwidth}{|X | X | X | l|}
\hline
\textbf{Material} & \textbf{CO2-Emissionen (kg CO2/kg)} & \textbf{Energieverbrauch (MJ/kg)} & \textbf{Quelle} \\
\hline
PLA & 0.5-1.5 & 50-70 MJ & \cite{sourcePLA} \\
\hline
Stahl (Primär) & 1.8-3.2 & 60-120 MJ & \cite{sourceSteel} \\
\hline
Aluminium (Primär) & 8-12 & 170-250 MJ & \cite{sourceAluminium} \\
\hline
ABS & 2.7-3.6 & \text{50-70 MJ} & \cite{sourceABS} \\
\hline
\end{tabularx}
\caption{Vergleich PLA, Stahl, Aluminium, ABS}
\label{table:pla-steel-alu-abs}
\end{table}



\subsection{Code Repositories}

Die Code Repositories befinden sich auf GitHub. Dabei gibt es für den Simulator keine GitHub Action\footnote{https://docs.github.com/en/actions/about-github-actions/understanding-github-actions}, die den Code automatisiert testet. Dies bedeutet, dass nur das Minimum an Rechenleistung von den GitHub Server verwendet wird, sprich, nur diesen Speicher, der nötig ist, um Code zu speichern. Die automatischen Tests werden lokal ausgeführt.

Zusätzlich setzt sich GitHub für Nachhaltigkeit ein.\cite{github-sustainability}

Sie haben mit drei anderen Firmen die Green Software Initiative gegründet. Diese soll dabei helfen grüne Software zu erstellen.  Dabei geht es um Software Tools, die sich ihrem Kohlenstoffausstoss bewusst sind und dabei effizient handeln. Das Ziel von der Green Software Initiative ist es, Kohlenstoff zu reduzieren und nicht nur zu neutralisieren.\cite{green-software-initiative}

GitHub hat ebenfalls bei One Tree Planted mitgemacht. Deren Ziel ist es, die Umwelt wieder herzustellen, indem Bäume gepflanzt werden.\cite{one-tree-planted} Auch mit The Ocean Cleanup hat GitHub kollaboriert. Diese wollen Plastik aus den Meeren entfernen.\cite{ocean-cleanup}.

Ausserdem wird nachhaltige Software von GitHub unterstützt und gefördert. So gibt es ein GitHub Repository, dass eine extensive Liste von grüner Software hat. Sie sind in 4 Kategorien eingeteilt: Messungssoftware, Kohlenstoffeffizienz, Kohlenstoffbewusstsein und spezielle Tools. Dieses Repository wird von GitHub selber gefördert.\cite{green-software}

GitHub ist seit 2019 CO$_{2}$-neutral und hat auch weitere nachhaltige Ziele. 2025 soll GitHub zu 100\% auf erneuerbare Energien setzen und in 2030 sogar CO$_{2}$-negativ sein.\cite{github-goals}

