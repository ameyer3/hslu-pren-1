\section{Technologierecherchen}

\subsection{Mechanik}



\newpage
\subsection{Elektrotechnik}

\subsubsection{Stromversorgung}

\subsubsection{Antriebe}

\subsubsection{Weg- und Distanzerkennung}




\newpage
\subsection{Informatik}

\subsubsection{Programmiersprache}

\begin{table}[H]
\centering
\small
\begin{tabular}{|l|l|l|}
\hline
  \textbf{} & \textbf{Python} & \textbf{Java}\\
  \hline
  \textbf{Beschreibung}  & \makecell{} & \makecell{}\\
  \hline
  \textbf{Vorteile}  & \makecell{-Lightweight\\-Beliebt für ML} & \makecell{-Schnell \\-Möglichkeit für Threading}\\
  \hline
  \textbf{Nachteile} & \makecell{-Langsam} & \makecell{-Heavyweight}\\
  \hline
  \textbf{Links} & https://www.python.org/ & https://www.java.com/en/ \\
  \hline
\end{tabular}
\caption{Programmiersprachen Vergleich}
\label{table:lang-compare}
\end{table}


\subsubsection{Wegfindung Algorithmus}

Es gibt mehrere Algorithmen, mit denen ein Weg in einem Wegenetzwerk gefunden werden können. Nachfolgend werden eingie miteinander vergliechen.

\vspace{10pt}
\hrule

\textbf{Variante 1: Dijkstras Algorithmus}

Beschreibung:

Vorteile:
\begin{itemize}
    \item ...
\end{itemize}

Nachteile:

Links:

\vspace{5pt}
\hrule

\textbf{Variante 2: A* Algorithmus}

Beschreibung:

Vorteile:
\begin{itemize}
    \item ...
\end{itemize}

Nachteile:

Links:

\vspace{5pt}
\hrule




\subsubsection{Bilderkennung}

\textbf{Bilderkennung Software:} Nachfolgend werden einige Softwarelösungen verglichen mit denen eine Bilderkennung durchgeführt werden könnten.

\vspace{10pt}
\hrule
\textbf{Variante 1: TensorFlow}

Beschreibung:

Vorteile:

Nachteile:
\begin{itemize}
    \item Internetzugang nötig
\end{itemize}

Links: https://www.tensorflow.org/

\vspace{5pt}
\hrule

\textbf{Variante 2: OpenCV}

Beschreibung:

Vorteile:
\begin{itemize}
    \item kein Internetzugang nötig
\end{itemize}

Nachteile:

Links: https://opencv.org/

\vspace{5pt}
\hrule

\textbf{Variante 3: Google Vision API}

Beschreibung:

Vorteile:
\begin{itemize}
    \item ...
\end{itemize}

Nachteile:

Links: https://opencv.org/

\vspace{5pt}
\hrule
\vspace{10pt}

\textbf{Bilderkennung Kamera:} Nachfolgend werden einige Varianten verglichen, die als Kamera verwendet werden könnten.

Muss-Kriterium: Platz auf Raspi