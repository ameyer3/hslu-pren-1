\section{Technologierecherchen}

\subsection{Mechanik}



\newpage
\subsection{Elektrotechnik}

\subsubsection{Stromversorgung}

\subsubsection{Antriebe}

\begin{table}[H]
\centering
\small
\begin{tabular}{|l|l|l|}
\hline
  \textbf{} & \textbf{Bürstenbehafteter Motor} & \textbf{Bürstenloser Motor (BLDC)}\\
  \hline
  \textbf{Beschreibung}  & \makecell{Einfach und gutes Preis-Leistungsverhältnis} & \makecell{Leichter Motor benötigt jedoch \\zusätzliche Schaltung}\\
  \hline
  \textbf{Vorteile}  & \makecell{-aufgrund linearer Strom-Drehmoment \\Charakteristik gutes Regelverhalten\\-Drehzahleinstellbereich gross} & \makecell{-Belastbar\\-wartungsfrei\\-niedriges Gewicht}\\
  \hline
  \textbf{Nachteile} & \makecell{-schlechte Wärmeableitung\\-Warten von Kommutator und Bürsten}& \makecell{-Sensorsystem notwendig\\-zusätzliche Systemkosten}\\
  \hline
  \textbf{Links} & https://doi.org/10.1007/978-3-658-37423-5& https://doi.org/10.1007/978-3-658-37423-5\\
  \hline
\end{tabular}
\caption{Antriebe}
\label{table:motor1-compare}
\end{table}


\begin{table}[H]
\centering
\small
\begin{tabular}{|l|l|l|}
\hline
  \textbf{} & \textbf{Asynchron Motor} & \textbf{Schrittmotoren}\\
  \hline
  \textbf{Beschreibung}  & \makecell{effizienter Motor} & \makecell{Präziser Motor ohne zusätzliche Steuerung}\\
  \hline
  \textbf{Vorteile}  & \makecell{-robust, wartungsfrei \\-tiefe Herstellungskosten} & \makecell{-wartungsfrei\\-kostengünstig\\-Betrieb ohne weiteren Sensoren}\\
  \hline
  \textbf{Nachteile} & \makecell{-zusätzliche Komponenten nötig für \\Drehzahlverstellung\\-unpräzise ohne Steuerelektronik}& \makecell{-Leistungen müssen bekannt sein\\-niedrige Leistungsdichte}\\
  \hline
  \textbf{Links} & https://doi.org/10.1007/978-3-658-37423-5& https://doi.org/10.1007/978-3-658-37423-5\\
  \hline
\end{tabular}
\caption{Antriebe}
\label{table:motor2-compare}
\end{table}

\subsubsection{Objekterkennung}

\begin{table}[H]
\centering
\small
\begin{tabular}{|l|l|l|}
\hline
  \textbf{} & \textbf{Ultraschall} & \textbf{Optoelektrisch}\\
  \hline
  \textbf{Beschreibung}  & \makecell{Senden und Empfangen von Ultraschall} & \makecell{Senden und Emfpangen von Licht}\\
  \hline
  \textbf{Vorteile}  & \makecell{-geringe Störeinflüsse von Material und Umwelt\\-grosse Reichweite} & \makecell{-Präzise Messungen möglich}\\
  \hline
  \textbf{Nachteile} & \makecell{-Blindzone (min. Reichweite)} & \makecell{-Störeinflüsse von Licht\\-Materialbeschaffenheit}\\
  \hline
  \textbf{Links} & https://doi.org/10.1007/978-3-658-12562-2& doi.org/10.1007/978-3-658-12562-2\\
  \hline
\end{tabular}
\caption{Objekterkennung}
\label{table:et-object-detection-compare}
\end{table}





\newpage
\subsection{Informatik}

\subsubsection{Programmiersprache}

\begin{table}[H]
\centering
\small
\begin{tabularx}{\textwidth}{|l|X|X|}
\hline
\textbf{} & \textbf{Python} & \textbf{Java}\\
  \hline
  \textbf{Beschreibung}  & Interpretierte Programmiersprache & Kompilierte Programmiersprache\\
  \hline
  \textbf{Vorteile}  & \makecell{-Lightweight\\-Beliebt für ML\\-Umfangreiche Bibliotheken\\-Grosser Community Support} & \makecell{-Schnell \\-Möglichkeit für Threading}\\
  \hline
  \textbf{Nachteile} & \makecell{-Langsam} & \makecell{-Heavyweight}\\
  \hline
  \textbf{Links} & https://www.python.org/ & https://www.java.com/en/ \\
  \hline
\end{tabularx}
\caption{Programmiersprachen Vergleich}
\label{table:lang-compare}
\end{table}

\subsubsection{Wegfindung Algorithmus}

Es gibt mehrere Algorithmen, mit denen ein Weg in einem Wegenetzwerk gefunden werden können. Nachfolgend werden einige miteinander verglichen.


\begin{table}[H]
\centering
\small
\begin{tabularx}{\textwidth}{|l|X|X|}
\hline
\textbf{} & \textbf{Dijkstras Algorithmus} & \textbf{A* Algorithmus}\\
  \hline
  \textbf{Beschreibung}  & & \\
  \hline
  \textbf{Vorteile}  & \makecell{} & \makecell{}\\
  \hline
  \textbf{Nachteile} & \makecell{} & \makecell{}\\
  \hline
  \textbf{Links} &  &  \\
  \hline
\end{tabularx}
\caption{Wegfindung Algorithmus Vergleich}
\label{table:path-algo-compare}
\end{table}




\subsubsection{Bilderkennung}

\begin{table}[H]
\centering
\small
\begin{tabularx}{\textwidth}{|l|X|X|}
\hline
\textbf{} & \textbf{Tensorflow} & \textbf{LiteRT} \\
  \hline
  \textbf{Beschreibung}  & TensorFlow ist ein von Google entwickeltes Open-Source-Framework für maschinelles Lernen, das sich durch seine Skalierbarkeit und umfangreichen Bibliotheken auszeichnet. & TensorFlow Lite (LiteRT) ist die abgespeckte Version von TensorFlow, optimiert für die Ausführung auf mobilen und eingebetteten Geräten mit begrenzten Ressourcen. \\
  \hline
  \textbf{Vorteile}  & \makecell{-Gut Dokumentiert\\-Grosse Community\\-Gute Performance} & \makecell{-Optimiert für On-Device ML \\ -Gut Dokumentiert \\ -Gute Performance} \\
  \hline
  \textbf{Nachteile} & \makecell{-Internetzugang nötig \\ -Steile Lernkurve \\ } & \makecell{-Steile Lernkurve} \\
  \hline
  \textbf{Links} & https://www.tensorflow.org & https://ai.google.dev/edge/litert \\
  \hline
\end{tabularx}
\begin{tabularx}{\textwidth}{|l|X|X|}
\hline
\textbf{} & \textbf{PyTorch} & \textbf{OpenCV}\\
  \hline
  \textbf{Beschreibung} & PyTorch ist ein von Facebook entwickeltes Open-Source-Framework für maschinelles Lernen, das für seine dynamischen Berechnungsgraphen und einfache Handhabung bekannt ist. & OpenCV ist eine Open-Source-Bibliothek für Computer Vision, die effiziente Algorithmen für Bild- und Videoverarbeitung bereitstellt. \\
  \hline
  \textbf{Vorteile} & \makecell{-Einfacher zu erlernen als TensorFlow \\ -Unterstützt dynamische Graphen \\ (flexibler bei Modellen) \\ -Stark in Forschung und Experimenten \\ -Grosse Community} & \makecell{-Flache Lernkurve \\ -Grosse Community} \\
  \hline
  \textbf{Nachteile} & \makecell{-Weniger ausgereift für die Produktion \\ im Vergleich zu TensorFlow \\ -Hoher Speicherbedarf} & \makecell{-Rechenintensiv \\ -Schlechte Performance \\ -Nicht die beste Wahl für Deep-Learn-\\ing-Fähigkeiten} \\
  \hline
  \textbf{Links} & https://pytorch.org/ & https://opencv.org/ \\
  \hline
\end{tabularx}
\caption{Vision Objekterkennung Vergleich}
\label{table:vision-object-detection-compare}
\end{table}

\textbf{Bilderkennung Software:} Nachfolgend werden einige Softwarelösungen verglichen mit denen eine Bilderkennung durchgeführt werden könnten.

\vspace{10pt}
\hrule
\textbf{Variante 1: TensorFlow}

Beschreibung:

Vorteile:
\begin{itemize}
    \item Gut Dokumentiert
    \item Grosse Community
    \item Gute Performance dank Hardware Beschleunigung
\end{itemize}


Nachteile:
\begin{itemize}
    \item Internetzugang nötig
    \item Steile Lernkurve
    \item Überdimensioniert für einfacher Aufgaben
\end{itemize}

Links: https://www.tensorflow.org/

\vspace{5pt}
\hrule

\textbf{Variante 1: LiteRT (ehemals: TensorFlow Lite}

Beschreibung:

Vorteile:
\begin{itemize}
    \item Optimiert für On-Device ML Applikationen
    \item Gut Dokumentiert
    \item Grosse Community
    \item Gute Performance dank Hardware Beschleunigung
    \item Kein internetzugang nötig
\end{itemize}


Nachteile:
\begin{itemize}
    \item Steile Lernkurve
    \item Überdimensioniert für einfacher Aufgaben
\end{itemize}

Links: https://www.tensorflow.org/

\vspace{5pt}
\hrule

\textbf{Variante 2: OpenCV}

Beschreibung:

Vorteile:
\begin{itemize}
    \item kein Internetzugang nötig
    \item Leicht zu erlenen und zu implementieren
    \item Grosse Community, viele Tutorials und Beispiele verfügbar.
\end{itemize}

Nachteile:
\begin{itemize}
    \item Rechenintensiv, schlechte Performance (je nach Aufgabe)
    \item Nicht die beste Wahl für Deep-Learning-Fähigkeiten
\end{itemize}
Links: https://opencv.org/



\textbf{Bilderkennung Kamera:} Nachfolgend werden einige Varianten verglichen, die als Kamera verwendet werden könnten.

Muss-Kriterium: Platz auf Raspi

\begin{table}[H]
\centering
\small
\begin{tabularx}{\textwidth}{|l|X|X|}
\hline
\textbf{} & \textbf{} & \textbf{} \\
  \hline
  \textbf{Beschreibung}  & & \\
  \hline
  \textbf{Vorteile}  & \makecell{} & \makecell{} \\
  \hline
  \textbf{Nachteile} & \makecell{} & \makecell{} \\
  \hline
  \textbf{Links} & & \\
  \hline
\end{tabularx}
\begin{tabularx}{\textwidth}{|l|X|X|}
\hline
\textbf{} & \textbf{Raspberry Camera Modul 3} & \textbf{USB Webcam}\\
  \hline
  \textbf{Beschreibung} & Kamer & \\
  \hline
  \textbf{Vorteile}  & \makecell{} & \makecell{} \\
  \hline
  \textbf{Nachteile} & \makecell{} & \makecell{} \\
  \hline
  \textbf{Links} & & \\
  \hline
\end{tabularx}
\caption{Kamera Vergleich}
\label{table:camera-compare}
\end{table}