\section{Technologierecherchen}

\subsection{Mechanik}

\subsubsection{Materialien}

\begin{table}[H]
\centering
\small
\begin{tabularx}{\textwidth}{|l|X|X|}
\hline
  \textbf{} & \textbf{Aluminium} & \textbf{Stahl} \\
  \hline
  \textbf{Vorteile}  & \makecell{-hohe spezifische Festigkeit\\ -gute Wärmeleitfähigkeit \\ -Normteile verfügbar \\ -recyclebar} & \makecell{-hohe Härte erreichbar \\-hoher E-Modul\\-hohe Festigkeit \\ -Normteile verfügbar\\ -recyclebar}\\ 
  \hline
  \textbf{Nachteile} & \makecell{-Preis \\ -aufwendige Verarbeitung \\} & \makecell{-Preis\\-aufwendige Verarbeitung \\ -Gewicht}\\
  \hline
\end{tabularx}
\caption{Metalle}
\label{table:metals-comparison}
\end{table}


\begin{table}[H]
\centering
\small
\begin{tabularx}{\textwidth}{|l|X|X|}
\hline
  \textbf{} & \textbf{ABS} & \textbf{PLA} \\
  \hline
  \textbf{Vorteile}  & \makecell{-hohe Stossfestigkeit\\ -duktil \\ -Wärmebeständig \\ -rapid Prototyping} & \makecell{-hohe Festigkeit \\-einfach druckbar\\ -rapid Prototyping}\\
  \hline
  \textbf{Nachteile} & \makecell{-schwer druckbar \\ -VOC's  \\} & \makecell{-spröde\\-schlechte Wärmebeständigkeit}\\
  \hline
\end{tabularx}
\caption{3D-Druck Thermoplaste}
\label{table:fdm-thermoplasts-comparison}
\end{table}


\begin{table}[H]
\centering
\small
\begin{tabularx}{\textwidth}{|l|X|X|}
\hline
  \textbf{} & \textbf{Holz} & \textbf{Mitteldichte Faserplatte (MDF)} \\
  \hline
  \textbf{Vorteile}  & \makecell{-Biologisch abbaubar\\ -Preis \\ -einfache Verarbeitung} & \makecell{-kann "gelasert"  werden \\-Preis\\ -rapid Prototyping}\\
  \hline
  \textbf{Nachteile} & \makecell{-brennbar \\ -Variationen in Festigkeit} & \makecell{-brennbar\\-nimmt Feuchtigkeit auf\\-nicht recyclebar}\\
  \hline
\end{tabularx}
\caption{Holzwerkstoffe}
\label{table:woods-comparison}
\end{table}

\newpage

\subsubsection{Mechanische Komponenten}




\begin{table}[H]
\centering
\small
\begin{tabularx}{\textwidth}{|l|X|X|}
\hline
\textbf{} & \textbf{Differential Wheel Drive} & \textbf{Überlagerungslenkgetriebe} \\
  \hline
  \textbf{Beschreibung}  & 2 einzeln Angetriebene Räder, 1 Schwenkrad & "Panzerlenkung" 1 Motor mit 2 Ausgangswellen, Drehzahl/Richtung unabhängig einstellbar \\
  \hline
  \textbf{Vorteile}  & \makecell{-kein Lenkgestänge\\-leicht\\-Punktwendung möglich} & \makecell{-1 Antriebsmotor\\-1Lenkmotor\\-kein Lenkgestänge}  \\
  \hline
  \textbf{Nachteile} & \makecell{-instabil\\-2 Antriebsmotoren} & \makecell{-Getriebe mit Verlusten\\-komplexer Aufbau} \\
  \hline
  \textbf{Links} & \url{https://en.wikipedia.org/wiki/Differential_wheeled_robot} & \url{https://de.wikipedia.org/wiki/Überlagerungslenkgetriebe} \\
  \hline
\end{tabularx}
\begin{tabularx}{\textwidth}{|l|X|X|}
\hline
\textbf{} & \textbf{Lenkgetriebe} & \textbf{Omniwheel}\\
  \hline
  \textbf{Beschreibung} & zur Umwandlung Rotation in unterschiedliche Einschlagwinkel & Rad, welches sich in der Ebene in alle Richtungen bewegen kann\\
  \hline
  \textbf{Vorteile} & \makecell{-simpler Aufbau\\-1 Aktuator} & \makecell{-keine Lenkung nötig\\-scharfe kurven möglich}\\
  \hline
  \textbf{Nachteile} & \makecell{-begrenzter Einschlag} & \makecell{-komplexer Aufbau}\\
  \hline
  \textbf{Links} & \url{https://de.wikipedia.org/wiki/Lenkgetriebe} & \url{https://en.wikipedia.org/wiki/Mecanum_wheel}\\
  \hline
  \end{tabularx}
\begin{tabularx}{\textwidth}{|l|X|X|}
\hline
\textbf{} & \textbf{Raupenfahrzeug} & \textbf{Bionischer Greifer}\\
  \hline
  \textbf{Beschreibung}& FZ wird über Ketten/Raupen angetrieben  &      Bionischer Greifer zum Aufheben des Hindernis     \\
  \hline
  \textbf{Vorteile}& \makecell{-Punktwenden möglich\\-geringer spez. Bodendruck\\-hohe Traktion\\-keine Lenkung nötig} & {-gute Haftung durch flexible Kontur \\   -Greifkräfte können geregelt werden    }\\
  \hline
  \textbf{Nachteile}&\makecell{-komplex\\-Gewicht\\-tiefe Vmax} &           \\
  \hline
  \textbf{Links}& \url{https://de.wikipedia.org/wiki/Kettenfahrzeug}          &               \\ \hline
  \end{tabularx}
\begin{tabularx}{\textwidth}{|l|X|X|}
\hline
\textbf{} & \textbf{Schutzklassen} & \textbf{}\\
  \hline
  \textbf{Beschreibung}&     DIN-Norm DIN EN 60529       &           \\
  \hline
  \textbf{Definition}&       Anfroderungen für IP44 und IP00      &           \\
  \hline
  \textbf{Links}&    https://www.jungheinrich-profishop.ch/ch-de/profi-guide/ip-schutzarten/       &               \\
  \hline
\end{tabularx}
\caption{Mechanische Komponenten}
\label{table:mechanical-components}
\end{table}










\newpage
\subsection{Elektrotechnik}

\subsubsection{Eingabeoptionen}

\begin{table}[H]
\centering
\small
\begin{tabularx}{\textwidth}{|l|X|X|}
\hline
  \textbf{} & \textbf{Schalter} & \textbf{Touchscreen} \\
  \hline
  \textbf{Beschreibung}  & Physische Eingabeoptionen wie Ein-/Aus- oder Stellschalter & Eingabe über ein Touchscreen\\
  \hline
  \textbf{Vorteile}  & \makecell{-keine Software benötigt\\-einfache Implementation} & \makecell{-sieht Modern aus \\ -Gewicht \\-Teuer}\\
  \hline
  \textbf{Nachteile} & \makecell{-Gewicht \\-Platzausnutzung} & \makecell{-Software\\-Reaktionsgeschwindigkeit}\\
  \hline
  \textbf{Links} &  \url{https://de.wikipedia.org/wiki/Schalter_(Elektrotechnik)} & \url{https://www.pi-shop.ch/display}\\
  \hline
\end{tabularx}
\caption{Eingabeoptionen}
\label{table:inputs-compare}
\end{table}

\subsubsection{Ausgabeoptionen}

\begin{table}[H]
\centering
\small
\begin{tabularx}{\textwidth}{|l|X|X|}
\hline
  \textbf{} & \textbf{Horn} & \textbf{Leuchte} \\
  \hline
  \textbf{Beschreibung}  & \makecell{Akustische Ausgabe von Angaben} & \makecell{Optische Ausgabe von Angaben}\\
  \hline
  \textbf{Vorteile}  & \makecell{-nicht überhöhrbar\\-musikfähig} & \makecell{-Billig \\-einfach einbaubar\\-klare Fehlerangabe}\\
  \hline
  \textbf{Nachteile} & \makecell{-Stromversorgung \\-unterschiede schwer erkennbar} & \makecell{-Stromverbrauch kann hoch sein}\\
  \hline
  \textbf{Links} &  \url{https://de.wikipedia.org/wiki/Summer_(Elektrik)} & \url{https://de.wikipedia.org/wiki/Leuchtdiode}\\
  \hline
\end{tabularx}
\caption{Ausgabeoptionen}
\label{table:outputs-compare}
\end{table}


\subsubsection{Stromversorgung}

\begin{table}[H]
\centering
\small
\begin{tabularx}{\textwidth}{|l|X|X|}
\hline
  \textbf{} & \textbf{Akkumulator} & \textbf{Batterie} \\
  \hline
  \textbf{Beschreibung}  & Spannungsstabilisierte Spannungsversorgung & Leicht wechselbare Spannungsversorgung\\
  \hline
  \textbf{Vorteile}  & \makecell{-geeignet für hoher Stromverbrauch\\-Ausgangsspannungsstabilität} & \makecell{-Billig \\-einfach einbaubar\\-Gewicht}\\
  \hline
  \textbf{Nachteile} & \makecell{-Gewicht} & \makecell{-keine Spannungsstabilität\\-keine grosse Ströme möglich}\\
  \hline
  \textbf{Links} &  \url{https://de.wikipedia.org/wiki/Akkumulator} & \url{https://de.wikipedia.org/wiki/Batterie_(Elektrotechnik)}\\
  \hline
\end{tabularx}
\caption{Steuerung}
\label{table:power-supply-compare}
\end{table}


\subsubsection{Antriebe}

\begin{table}[H]
\centering
\small
\begin{tabularx}{\textwidth}{|l|X|X|}
\hline
  \textbf{} & \textbf{Bürstenbehafteter Motor} & \textbf{Bürstenloser Motor (BLDC)}\\
  \hline
  \textbf{Beschreibung}  & Einfach und gutes Preis-Leistungsverhältnis & Leichter Motor benötigt jedoch zusätzliche Schaltung\\
  \hline
  \textbf{Vorteile}  & \makecell{-aufgrund linearer Strom-Drehmoment \\Charakteristik gutes Regelverhalten\\-Drehzahleinstellbereich gross} & \makecell{-Belastbar\\-wartungsfrei\\-niedriges Gewicht}\\
  \hline
  \textbf{Nachteile} & \makecell{-schlechte Wärmeableitung\\-Warten von Kommutator und Bürsten}& \makecell{-Sensorsystem notwendig\\-zusätzliche Systemkosten}\\
  \hline
  \textbf{Links} & \url{https://doi.org/10.1007/978-3-658-37423-5} & \url{https://doi.org/10.1007/978-3-658-37423-5}\\
  \hline
\end{tabularx}
\caption{Antriebe}
\label{table:motor1-compare}
\end{table}


\begin{table}[H]
\centering
\small
\begin{tabularx}{\textwidth}{|l|X|X|}
\hline
  \textbf{} & \textbf{Asynchron Motor} & \textbf{Schrittmotoren}\\
  \hline
  \textbf{Beschreibung}  & \makecell{effizienter Motor} & Präziser Motor ohne zusätzliche Steuerung\\
  \hline
  \textbf{Vorteile}  & \makecell{-robust, wartungsfrei \\-tiefe Herstellungskosten} & \makecell{-wartungsfrei\\-kostengünstig\\-Betrieb ohne weiteren Sensoren}\\
  \hline
  \textbf{Nachteile} & \makecell{-zusätzliche Komponenten nötig für \\Drehzahlverstellung\\-unpräzise ohne Steuerelektronik}& \makecell{-Leistungen müssen bekannt sein\\-niedrige Leistungsdichte}\\
  \hline
  \textbf{Links} & \url{https://doi.org/10.1007/978-3-658-37423-5} & \url{https://doi.org/10.1007/978-3-658-37423-5}\\
  \hline
\end{tabularx}
\caption{Antriebe}
\label{table:motor2-compare}
\end{table}

\subsubsection{Objekterkennung}

\begin{table}[H]
\centering
\small
\begin{tabularx}{\textwidth}{|l|X|X|}
\hline
  \textbf{} & \textbf{Ultraschall} & \textbf{Optoelektrisch}\\
  \hline
  \textbf{Beschreibung}  & \makecell{Senden und Empfangen von Ultraschall} & \makecell{Senden und Emfpangen von Licht}\\
  \hline
  \textbf{Vorteile}  & \makecell{-geringe Störeinflüsse von Material und \\Umwelt\\-grosse Reichweite} & \makecell{-Präzise Messungen möglich}\\
  \hline
  \textbf{Nachteile} & \makecell{-Blindzone (min. Reichweite)} & \makecell{-Störeinflüsse von Licht\\-Materialbeschaffenheit}\\
  \hline
  \textbf{Links} & \url{https://doi.org/10.1007/978-3-658-12562-2} & \url{https://doi.org/10.1007/978-3-658-12562-2}\\
  \hline
\end{tabularx}
\caption{Objekterkennung}
\label{table:et-object-detection-compare}
\end{table}


\subsubsection{Steuerung}

\begin{table}[H]
\centering
\small
\begin{tabularx}{\textwidth}{|l|X|X|}
\hline
  \textbf{} & \textbf{Mikrocontroller} & \textbf{Raspberry Pi} \\
  \hline
  \textbf{Beschreibung}  & \makecell{Steuerung durch echzeitfähiges System} & \makecell{System mit grosser Rechenleistung}\\
  \hline
  \textbf{Vorteile}  & \makecell{-geringer Stromverbrauch\\-echtzeitfähig} & \makecell{-Multithreading möglich \\-Bildverarbeitung möglich}\\
  \hline
  \textbf{Nachteile} & \makecell{-Bildverarbeitung nicht möglich} & \makecell{-Hoher Stromverbrach\\-Materialbeschaffenheit}\\
  \hline
  \textbf{Links} & \url{https://www.elektronik-kompendium.de/sites/com/1907171.html} & \url{https://www.raspberrypi.com}\\
  \hline
\end{tabularx}
\caption{Steuerung}
\label{table:controller-compare}
\end{table}



\newpage
\subsection{Informatik}

\subsubsection{Programmiersprache}

\begin{table}[H]
\centering
\small
\begin{tabularx}{\textwidth}{|l|X|X|}
\hline
\textbf{} & \textbf{Python} & \textbf{Java}\\
  \hline
  \textbf{Beschreibung}  & Interpretierte Programmiersprache & Kompilierte Programmiersprache\\
  \hline
  \textbf{Vorteile}  & \makecell{-Lightweight\\-Beliebt für ML\\-Umfangreiche Bibliotheken\\-Grosser Community Support} & \makecell{-Schnell \\-Möglichkeit für Threading}\\
  \hline
  \textbf{Nachteile} & \makecell{-Langsam} & \makecell{-Heavyweight}\\
  \hline
  \textbf{Links} & \url{https://www.python.org/} & \url{https://www.java.com/en/} \\
  \hline
\end{tabularx}
\caption{Programmiersprachen Vergleich}
\label{table:lang-compare}
\end{table}

\subsubsection{Wegfindung Algorithmus}

Es gibt mehrere Algorithmen, mit denen ein Weg in einem Wegenetzwerk gefunden werden können. Nachfolgend werden einige miteinander verglichen.


\begin{table}[H]
\centering
\small
\begin{tabularx}{\textwidth}{|l|X|X|}
\hline
\textbf{} & \textbf{Dijkstras Algorithmus} & \textbf{A* Algorithmus}\\
  \hline
  \textbf{Beschreibung} & Ein algorithmischer Ansatz, der den kürzesten Pfad von einem Startknoten zu allen anderen Knoten in einem gewichteten Graphen findet, ohne Zielinformationen zu berücksichtigen. & Ein Heuristik-basierter Suchalgorithmus, der den kürzesten Pfad zwischen einem Start- und Zielknoten effizient findet, indem er sowohl die tatsächlichen Kosten als auch geschätzte zukünftige Kosten berücksichtigt. \\
  \hline
  \textbf{Vorteile}  & \makecell{-Findet immer den kürzesten Weg \\ -Einfach zu implementieren} & \makecell{-Effizienter als Dijkstra \\ -Flexibilität durch Heuristik \\ -Anwendbar auf viele Probleme}\\
  \hline
  \textbf{Nachteile} & \makecell{-Keine Berücksichtigung von \\ Zielinformationen \\ -Nicht optimal für dynamische Graphen \\ -Langsame Laufzeit} & \makecell{-Speicherintensiv \\-Abhängig von Heuristik}\\
  \hline
  \textbf{Links} & \url{https://algorithms.discrete.ma.tum.de/}  & \url{https://en.wikipedia.org/wiki/A*_search_algorithm} \\
  \hline
\end{tabularx}
\caption{Wegfindung Algorithmus Vergleich}
\label{table:path-algo-compare}
\end{table}

\newpage

\subsubsection{Bilderkennung}

\textbf{Bilderkennung Software:} Nachfolgend werden einige Softwarelösungen verglichen mit denen eine Bilderkennung durchgeführt werden könnten.

\begin{table}[H]
\centering
\small
\begin{tabularx}{\textwidth}{|l|X|X|}
\hline
\textbf{} & \textbf{Tensorflow} & \textbf{LiteRT} \\
  \hline
  \textbf{Beschreibung}  & TensorFlow ist ein von Google entwickeltes Open-Source-Framework für maschinelles Lernen, das sich durch seine Skalierbarkeit und umfangreichen Bibliotheken auszeichnet. & TensorFlow Lite (LiteRT) ist die abgespeckte Version von TensorFlow, optimiert für die Ausführung auf mobilen und eingebetteten Geräten mit begrenzten Ressourcen. \\
  \hline
  \textbf{Vorteile}  & \makecell{-Gut Dokumentiert\\-Grosse Community\\-Gute Performance} & \makecell{-Optimiert für On-Device ML \\ -Gut Dokumentiert \\ -Gute Performance} \\
  \hline
  \textbf{Nachteile} & \makecell{-Steile Lernkurve \\-Überdimensioniert für einfachere\\ Aufgaben } & \makecell{-Steile Lernkurve} \\
  \hline
  \textbf{Links} & \url{https://www.tensorflow.org} & \url{https://ai.google.dev/edge/litert} \\
  \hline
\end{tabularx}
\begin{tabularx}{\textwidth}{|l|X|X|}
\hline
\textbf{} & \textbf{PyTorch} & \textbf{OpenCV}\\
  \hline
  \textbf{Beschreibung} & PyTorch ist ein von Facebook entwickeltes Open-Source-Framework für maschinelles Lernen, das für seine dynamischen Berechnungsgraphen und einfache Handhabung bekannt ist. & OpenCV ist eine Open-Source-Bibliothek für Computer Vision, die effiziente Algorithmen für Bild- und Videoverarbeitung bereitstellt. \\
  \hline
  \textbf{Vorteile} & \makecell{-Einfacher zu erlernen als TensorFlow \\ -Unterstützt dynamische Graphen \\ (flexibler bei Modellen) \\ -Stark in Forschung und Experimenten \\ -Grosse Community} & \makecell{-Flache Lernkurve \\ -Grosse Community} \\
  \hline
  \textbf{Nachteile} & \makecell{-Weniger ausgereift für die Produktion \\ im Vergleich zu TensorFlow \\ -Hoher Speicherbedarf} & \makecell{-Rechenintensiv \\ -Schlechte Performance \\ -Nicht die beste Wahl für Deep-Learn-\\ing-Fähigkeiten} \\
  \hline
  \textbf{Links} & \url{https://pytorch.org/} & \url{https://opencv.org/} \\
  \hline
\end{tabularx}
\caption{Vision Objekterkennung Vergleich}
\label{table:vision-object-detection-compare}
\end{table}

\newpage

\textbf{Bilderkennung Kamera:} Nachfolgend werden einige Varianten verglichen, die als Kamera verwendet werden könnten.

Muss-Kriterium: Kompatibilität mit Raspberry Pi

\begin{table}[H]
\centering
\small
\begin{tabularx}{\textwidth}{|l|X|X|}
\hline
\textbf{} & \textbf{Raspberry Camera Modul} & \textbf{USB Webcam}\\
  \hline
  \textbf{Beschreibung} & Kamera Module die speziell für den Einsatz mit dem Raspberry Pi entwickelt sind. Mit einer eigenen CSI Schnittstelle für den Anschluss an einen Single Board Computer & Handelsübliche Webcam wie Sie für Video Meetings eingesetzt wird. \\
  \hline
  \textbf{Vorteile}  & \makecell{-Kompatibilität mit Raspberry Pi \\ -CSI Konnektor} & \makecell{-Grössere Auswahl an Möglichkeiten \\ -Halterung meist inklusive \\ -Schwenkbar und Drehbar} \\
  \hline
  \textbf{Nachteile} & \makecell{-Kein Gehäuse \\ -Keine Halterung} & \makecell{} \\
  \hline
  \textbf{Links} & \url{https://www.raspberrypi.com/documentation/accessories/camera.html} & \url{https://www.logitech.com/de-ch/shop/c/webcams} \\
  \hline
\end{tabularx}
\begin{tabularx}{\textwidth}{|l|X|X|}
\hline
\textbf{} & \textbf{Industriekamera} & \textbf{} \\
  \hline
  \textbf{Beschreibung} & Eine Industriekamera wird oft für Vision Anwendungen im Industriebereich eingesetzt. Sie wird in den meisten Fällen via Ethernet Gygabyte Anbindung angeschlossen. In einigen Konfigurationen auch mit USB 3.0 verfügbar. & \\
  \hline
  \textbf{Vorteile}  & \makecell{-Sehr gute Qualität \\ -Objektiv frei wählbar \\ -Konfigurierbar} & \makecell{} \\
  \hline
  \textbf{Nachteile} & \makecell{-Kostspielig \\ -Schlechte Kompatibilität \\ -Keine Halterung inklusive} & \makecell{} \\
  \hline
  \textbf{Links} & \url{https://machinevisionkamera.de/Industriekamera-USB3-Vision} & \\
  \hline
\end{tabularx}
\caption{Kamera Vergleich}
\label{table:camera-compare}
\end{table}


\subsection{Simulator}

\subsubsection{Programmiersprache}

Wie bei der Software für den Roboter kommen für den Simulator Java oder Python infrage. Der Vergleich der beiden Programmiersprachen wurde im vorherigen Kapitel durchgeführt.

\subsubsection{Wegenetzwerk einlesen}

\begin{table}[H]
\centering
\small
\begin{tabularx}{\textwidth}{|l|X|X|}
\hline
\textbf{} & \textbf{YAML} & \textbf{JSON}\\
  \hline
  \textbf{Beschreibung} & Eine Datenseralisierungssprache die gut vom Mensch gelesen werden kann. &  Eine Sprache für den Datenaustausch. Steht für JavaScript Object Notation. \\
  \hline
  \textbf{Vorteile}  & \makecell{-sehr simpel} & \makecell{-Java aehnlicher Syntax} \\
  \hline
  \textbf{Nachteile} & \makecell{-etwas aufwaendiger in Java einzulesen} & \makecell{} \\
  \hline
  \textbf{Links} & \url{https://yaml.org/} & \url{https://www.json.org/json-en.html} \\
  \hline
\end{tabularx}

\begin{tabularx}{\textwidth}{|l|X|X|}
\hline
\textbf{} & \textbf{Bild} & \textbf{}\\
  \hline
  \textbf{Beschreibung} & Das Wegenetz würde als Graph skizziert werden und vom Simulator einglesen werden. &   \\
  \hline
  \textbf{Vorteile}  & \makecell{-Realitätsnah} & \makecell{} \\
  \hline
  \textbf{Nachteile} & \makecell{-sehr viel Aufwand} & \makecell{} \\
  \hline
  \textbf{Links} & \url{} &  \\
  \hline
\end{tabularx}
\caption{Wegenetz einlesen Vergleich}
\label{table:read-path-compare}
\end{table}


\subsubsection{Wegenetz intern speichern}

\begin{table}[H]
\centering
\small
\begin{tabularx}{\textwidth}{|l|X|X|}
\hline
\textbf{} & \textbf{Key-Value Pairs} & \textbf{2D Array}\\
  \hline
  \textbf{Beschreibung} & Jeder Knoten ist ein Key und hat als Value eine Liste der Konten, mit denen er verbunden ist. &  Es wird eine Adjazenzmatrix erstellt in einem zweidimensionalen Array.\\
  \hline
  \textbf{Vorteile}  & \makecell{-lightweight} & \makecell{-Gewichtung moeglich} \\
  \hline
  \textbf{Nachteile} & \makecell{-kann die Kanten nicht gewichten} & \makecell{} \\
  \hline
  \textbf{Links} & \url{https://www.python.org/doc/essays/graphs/} & \url{https://de.wikipedia.org/wiki/Adjazenzmatrix}\\
  \hline
\end{tabularx}

\begin{tabularx}{\textwidth}{|l|X|X|}
\hline
\textbf{} & \textbf{Externe Library} & \textbf{}\\
  \hline
  \textbf{Beschreibung} & Es wird eine externe Library verwendet, die einen Graph Datentyp anbietet.  &   \\
  \hline
  \textbf{Vorteile}  & \makecell{-hat oft andere nuetzliche Funktionen\\-Gewichtung moeglich} & \makecell{} \\
  \hline
  \textbf{Nachteile} & \makecell{-externe Abhaengigkeit\\-eventuell unnuetze andere Funktionen} & \makecell{} \\
  \hline
  \textbf{Links} & \url{https://networkx.org/} &  \\
  \hline
\end{tabularx}
\caption{Wegenetz erfassen Vergleich}
\label{table:store-path-compare}
\end{table}

\subsubsection{Bewegliche Hindernisse erfassen}

\begin{table}[H]
\centering
\small
\begin{tabularx}{\textwidth}{|l|X|X|}
\hline
\textbf{} & \textbf{Gewichtung} & \textbf{Aus Graph entfernen}\\
  \hline
  \textbf{Beschreibung} & Die Strecken, auf denen sich ein Hindernis befindet wird hoeher gewichtet.  & Die Kante wird aus dem Gaph entfernt.  \\
  \hline
  \textbf{Vorteile}  & \makecell{-Natuerlicher Aspekt der Graphentheorie\\-Diese Strecken werden so vermieden} & \makecell{-simpel} \\
  \hline
  \textbf{Nachteile} & \makecell{} & \makecell{-moeglicherweise langer Weg\\-gibt moeglicherweise keine Strecke\\ ohne Hindernisse} \\
  \hline
  \textbf{Links} & \url{https://hyperskill.org/learn/step/5645} &  \\
  \hline
\end{tabularx}
\caption{Bewegliche Hindernisse erfassen Vergleich}
\label{table:analyse-path-compare-1}
\end{table}



\subsubsection{Blockierte Knoten erfassen}

\begin{table}[H]
\centering
\small
\begin{tabularx}{\textwidth}{|l|X|X|}
\hline
\textbf{} & \textbf{Markieren} & \textbf{Aus gespeichertem Netz entfernen}\\
  \hline
  \textbf{Beschreibung} & Die blockierten Knoten werden markiert.  &  Die Knoten inklusive der Verbindungen werden entfernt. \\
  \hline
  \textbf{Vorteile}  & \makecell{} & \makecell{-simpel} \\
  \hline
  \textbf{Nachteile} & \makecell{-ueberkompliziert} & \makecell{} \\
  \hline
  \textbf{Links} & \url{} &  \\
  \hline
\end{tabularx}
\caption{Blockierte Knoten erfassen Vergleich}
\label{table:analyse-path-compare-2}
\end{table}



\subsubsection{Wegfindung}

\begin{table}[H]
\centering
\small
\begin{tabularx}{\textwidth}{|l|X|X|}
\hline
\textbf{} & \textbf{Selber implementieren} & \textbf{Externe Library verwenden}\\
  \hline
  \textbf{Beschreibung} & Der Algorithmus zur Wegfindung wird manuell implementiert.  & Es wird eine externe Library verwendet, um den Algorithmus auszuführen.  \\
  \hline
  \textbf{Vorteile}  & \makecell{-keine externen Abhängigkeiten\\-keine unnötigen Zusatzfunktionen} & \makecell{-simpel\\-wahrscheinlich effizienter\\-weniger Fehleranfaellig} \\
  \hline
  \textbf{Nachteile} & \makecell{-komplex} & \makecell{-externe Abhängigkeit} \\
  \hline
  \textbf{Links} & &\url{https://networkx.org/}  \url{https://pypi.org/project/astar/}  \\
  \hline
\end{tabularx}
\caption{Wegfindung implementieren Vergleich}
\label{table:algorithm-compare}
\end{table}


\subsubsection{Zielauswahl}

\begin{table}[H]
\centering
\small
\begin{tabularx}{\textwidth}{|l|X|X|}
\hline
\textbf{} & \textbf{Human Input} & \textbf{Roboter berechnet}\\
  \hline
  \textbf{Beschreibung} & Durch Input einer Person wird das Ziel gewählt. &  Der Roboter berechnet den kürzeste Weg zu allen Zielen und wählt das nahste Ziel aus.\\
  \hline
  \textbf{Vorteile}  & \makecell{-simpel\\-schnelle Zielauswahl} & \makecell{-schnellste Router gewaehlt} \\
  \hline
  \textbf{Nachteile} & \makecell{-nicht immer das optimal Ziel wird gewhlt} & \makecell{-aufwaendig\\-moeglicherweise langsam} \\
  \hline
  \textbf{Links} &&\\
  \hline
\end{tabularx}

\begin{tabularx}{\textwidth}{|l|X|X|}
\hline
\textbf{} & \textbf{Random} & \textbf{}\\
  \hline
  \textbf{Beschreibung} & Der Roboter wählt zufällig eines der Ziele aus.  &   \\
  \hline
  \textbf{Vorteile}  & \makecell{-schnell} \\
  \hline
  \textbf{Nachteile} & \makecell{-ungenau} \\
  \hline
  \textbf{Links} & \url{} &  \\
  \hline
\end{tabularx}
\caption{Zielauswahl Vergleich}
\label{table:goal-compare}
\end{table}

\subsubsection{Clientseitige Kommunikation (I/O)}

\begin{table}[H]
\centering
\small
\begin{tabularx}{\textwidth}{|l|X|X|}
\hline
\textbf{} & \textbf{GUI} & \textbf{CLI}\\
  \hline
  \textbf{Beschreibung} & Es gibt ein graphisches User Interface. &  Inputs und Outputs werden ueber eine Command Line gemacht.\\
  \hline
  \textbf{Vorteile}  & \makecell{-schoen\\-benutzerfreundlichl} & \makecell{-simpel} \\
  \hline
  \textbf{Nachteile} & \makecell{-mehr Aufwand} & \makecell{-eher unübersichtlich} \\
  \hline
  \textbf{Links} &&\\
  \hline
\end{tabularx}

\begin{tabularx}{\textwidth}{|l|X|X|}
\hline
\textbf{} & \textbf{Keine Inputs oder Outputs} & \textbf{}\\
  \hline
  \textbf{Beschreibung} & Der Mensch kommuniziert nicht mit dem Roboter. Falls der Roboter das Ziel nicht erreicht gibt es eine Fehlermeldung. &   \\
  \hline
  \textbf{Vorteile}  & \makecell{-einfach} \\
  \hline
  \textbf{Nachteile} & \makecell{-sehr intransparent} \\
  \hline
  \textbf{Links} & \url{} &  \\
  \hline
\end{tabularx}
\caption{Kommunikation Vergleich}
\label{table:communication-compare}
\end{table}

%%%%%%%%%%%%%%%%%%%%%%%%%%%%%%%%%%%%%%%%%%
%%%%%%%%%%%%%%%%%%%%%%%%%%%%%%%%%%%%%%%%%%
%%%%%  Notes START (To be deleted afterwards)
%%%%%%%%%%%%%%%%%%%%%%%%%%%%%%%%%%%%%%%%%%
%%%%%%%%%%%%%%%%%%%%%%%%%%%%%%%%%%%%%%%%%%


\newpage
\begin{Huge}
\color{red}
NOTES Section, To be deleted afterwards
\end{Huge}


\vspace{10pt}
\hrule
\textbf{Variante 1: TensorFlow}

Beschreibung:

Vorteile:
\begin{itemize}
    \item Gut Dokumentiert
    \item Grosse Community
    \item Gute Performance dank Hardware Beschleunigung
\end{itemize}


Nachteile:
\begin{itemize}
    \item Internetzugang nötig
    \item Steile Lernkurve
    \item Überdimensioniert für einfacher Aufgaben
\end{itemize}

Links: https://www.tensorflow.org/

\vspace{5pt}
\hrule

\textbf{Variante 1: LiteRT (ehemals: TensorFlow Lite}

Beschreibung:

Vorteile:
\begin{itemize}
    \item Optimiert für On-Device ML Applikationen
    \item Gut Dokumentiert
    \item Grosse Community
    \item Gute Performance dank Hardware Beschleunigung
    \item Kein internetzugang nötig
\end{itemize}


Nachteile:
\begin{itemize}
    \item Steile Lernkurve
    \item Überdimensioniert für einfacher Aufgaben
\end{itemize}

Links: https://www.tensorflow.org/

\vspace{5pt}
\hrule

\textbf{Variante 2: OpenCV}

Beschreibung:

Vorteile:
\begin{itemize}
    \item kein Internetzugang nötig
    \item Leicht zu erlenen und zu implementieren
    \item Grosse Community, viele Tutorials und Beispiele verfügbar.
\end{itemize}

Nachteile:
\begin{itemize}
    \item Rechenintensiv, schlechte Performance (je nach Aufgabe)
    \item Nicht die beste Wahl für Deep-Learning-Fähigkeiten
\end{itemize}
Links: https://opencv.org/

\begin{Huge}
\color{red}
NOTES Section END, To be deleted afterwards
\end{Huge}

%%%%%%%%%%%%%%%%%%%%%%%%%%%%%%%%%%%%%%%%%%
%%%%%%%%%%%%%%%%%%%%%%%%%%%%%%%%%%%%%%%%%%
%%%%%  Notes END
%%%%%%%%%%%%%%%%%%%%%%%%%%%%%%%%%%%%%%%%%%
%%%%%%%%%%%%%%%%%%%%%%%%%%%%%%%%%%%%%%%%%%




