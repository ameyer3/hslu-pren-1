\section{Evaluation Lösungsprinzipien}

Um die Lösungsprinzipien zu evaluieren und verschiedene Varianten zu finden wird pro Teilbereich ein Morphologischer Kasten erstellt. Dieser ergibt sich aus den Technologierecherchen.

\subsection{Maschinentechnik Morphologischer Kasten}

PLACEHOLDER

\subsection{Elektrotechnik Morphologischer Kasten}

PLACEHOLDER

\subsection{Informatik Morphologischer Kasten}

PLACEHOLDER

\subsection{Simulator Morphologischer Kasten}

Kante entfernen nicht, weil gibt ev keinen Weg mehr, zu risky

Knoten nur markieren nicht noetig, bringt keinen Vorteil


\newpage
\section{Auswahl Lösungskombinationen}

Zur Auswahl der Lösungskombinationen wird pro Teilbereich eine Nutzwertanalyse durchgefuehrt. Die Kriterien und deren Gewichtung sind individuell pro Teilbereich, damit sie ideal zum jeweiligen Teil passen.

\subsection{Maschinentechnik Nutzwertanalyse}

PLACEHOLDER

\subsection{Elektrotechnik Nutzwertanalyse}

PLACEHOLDER

\subsection{Informatik Nutzwertanalyse}

PLACEHOLDER

\subsection{Simulator Nutzwertanalyse}

PLACEHOLDER

Schnelligkeit vernachlaessigbar weil: nur 8 Knoten und weil macht sehr wenig aus, viel wichtiger ist Lightweight, da nur 1 Raspi in Realitaet und wollen moeglichst realitaetsgetreu das machen

\newpage
\section{Gesamtkonzept}

Aus den morphologischen Kasten und den Nutzwertanalysen wurde folgendes Gesamtkonzept ermittelt. 

