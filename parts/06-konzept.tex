\section{Evaluation Lösungsprinzipien}

Aus der vorhergehenden Technologierecherche wird pro Studiengang ein Morphologischer Kasten erstellt. Dieser wird befüllt mit den Teilfunktionen und den recherchierten Technologien. Damit werden die Lösungsprinzipien evaluiert und verschiedene Varianten werden gefunden. Ein Überblick ueber alle Morphologischen Kasten befindet sich im Anhang \ref{Morphologischer Kasten}.

\subsection{Maschinentechnik Morphologischer Kasten}

PLACEHOLDER

\subsection{Elektrotechnik Morphologischer Kasten}

PLACEHOLDER

\subsection{Informatik Morphologischer Kasten}

Tensorflow nicht verwendet, da fuer uns in jeder Variante zu gross, also nur TensorLite

\subsection{Simulator Morphologischer Kasten}

Kante entfernen nicht, weil gibt ev keinen Weg mehr, zu risky: Robustheit

Knoten nur markieren nicht noetig, bringt keinen Vorteil


\newpage
\section{Auswahl Lösungskombinationen}

Zur Auswahl der passendsten Lösungskombination wird pro Teilbereich eine Nutzwertanalyse durchgeführt. Die Kriterien und deren Gewichtung sind individuell pro Teilbereich, damit sie ideal passen.

\subsection{Maschinentechnik Nutzwertanalyse}

PLACEHOLDER

\subsection{Elektrotechnik Nutzwertanalyse}

PLACEHOLDER

\subsection{Informatik Nutzwertanalyse}

PLACEHOLDER

\subsection{Simulator Nutzwertanalyse}

PLACEHOLDER

Schnelligkeit vernachlaessigbar weil: nur 8 Knoten und weil macht sehr wenig aus, viel wichtiger ist Lightweight, da nur 1 Raspi in Realitaet und wollen moeglichst realitaetsgetreu das machen

\newpage
\section{Gesamtkonzept}

Aus den morphologischen Kasten und den Nutzwertanalysen wurde folgendes Gesamtkonzept ermittelt. 

