\section{Projektplanung}

Die Projektplanung wurde anhand der vorgegebenen Abgabetermine durchgeführt.
Da es vier Abgabetermine gibt, wurde das Projekt in vier Sprints aufgeteilt, die jeweils mit einem Meilenstein enden.

\subsection{Projektplan}

Der Projektplan in Tabelle \ref{table:projektplan} zeigt die Meilensteine, die zu erreichen sind.
Die Meilensteine sind die einzelnen Testatabgaben.
Die abzugebenden Dokumente pro Testat bilden die einzelnen Sprintziele.

\begin{table}[h!]
\centering
\begin{tabular}{|l  l l|}
\hline
  \textbf{Meilenstein} & \textbf{Datum} & \textbf{Beschreibung} \\
  \hline
  Meilenstein 1  & 04. Oktober 2024 & \makecell{Projektplan, Skizzierung der Aufgabenstellung,\\ Technologierecherche, Anforderungsliste}\\
  \hline
  Meilenstein 2  & 01. November 2024 & \makecell{Evaluation der Lösungsprinzipien, Auswahl der\\ optimalen Lösungeskombinationen}\\
  \hline
  Meilenstein 3  & 06. Dezember 2024 & \makecell{Freigabe des Gesamtkonzepts, Simulator Wegplanung, \\Dokumentation zu 80\% fertiggstellt}\\
  \hline
  Meilenstein 4  & 10. Januar 2025 & \makecell{Schlussbereicht, Präsentation}\\
  \hline
\end{tabular}
\caption{Projektplan}
\label{table:projektplan}
\end{table}

\subsection{Backlog}

Der Backlog dient als zentrales Planungselement.
In diesem Projekt sind die einzelnen Meilensteine und was diese enthalten vorgegeben. Aus diesem Grund wurden bereits zu Beginn alle Sprints geplant und der ganze Product Backlog wurde in Sprintbacklogs aufgeteilt. Nach Erreichen eines Meilensteins wird ein Ausblick auf den nächsten Sprint durchgeführt, um allfällige Anpassungen an dem Sprintbacklog vorzunehmen.

\subsection{Sprintplanung}

In den folgenden Kapiteln wurden für jeden Sprint Sprintziele definiert und ein Sprintbacklog erstellt. Für den Sprintbacklog wurden die einzelnen Meilensteine aus dem Projektplan in Epics beschrieben. Zu den Epics wurden User Stories erstellt. Der Aufwand der einzelnen Stories wurde mit T-Shirt Grössen geschätzt.

\newpage
\subsubsection{Sprint 1: 20. September 2024 - 04. Oktober 2024}

\textbf{Sprintziel:}
\begin{itemize}
    \item Projektplan erstellt
    \item Aufgabenstellung skizziert
    \item Andorderungsliste erstellt
    \item Technologierecherche
\end{itemize}

\textbf{Sprintbacklog:} Der Sprintbacklog von Sprint 1 ist in Tabelle \ref{table:sprint1-backlog} dargestellt.

\begin{table}[H]
\centering
\small
\begin{tabular}{|l|l|l| l|}
\hline
  \textbf{Nr.} & \textbf{Titel} & \textbf{Beschreibung} & \textbf{Size}\\
  \hline
  1  & \textbf{Projektorganisation definieren} &&\\
  \hline
  1.1  & Rollendefinition & \makecell{Die Rollen Projektleiter und\\ Werkstattverwantwortliche werden definiert.} & S\\
  \hline
  1.2 & Datenaustausch definieren & \makecell{Zentrale Datenablage und \\Kommunikationsschnittstellen definieren.}& S\\
  \hline
  1.3 & Ziele definieren & Definieren, wie wir uns den Roboter vorstellen. & S\\
  \hline
  1.4 & Vorgehen definieren & Geeignete Projektmethode wird gewählt. & S\\
  \hline
  2 & \textbf{Aufgabenstellung klären} && \\
  \hline
  2.1 & Anforderungsliste erstellen & \makecell{Anforderungen, die der Roboter erfüllen muss\\ sammeln}& M \\
  \hline
  2.2 & Aufgabenstellung skizzieren & Modellierunge der Aufgabe zum Verständnis. & S \\
  \hline
  3 & \textbf{Projektplanung} && \\
  \hline
  3.1 & Projektplan erstellen & Meilensteine definieren. & S \\
  \hline
  3.2 & Backlog erstellen & Product Backlog für alle Sprints erstellen. & M \\
  \hline
  4 & \textbf{Lösungsvarianten erarbeiten} && \\
  \hline
  4.1 & Teilfunktionen finden & Roboter in Teilfunktionen aufteilen & S \\
  \hline
  4.2 & Technologierecherche & \makecell{Recherche zu den einzelnen Teilfunktionen \\ durchführen.} & L\\
  \hline
 
\end{tabular}
\caption{Sprint 1 Backlog}
\label{table:sprint1-backlog}
\end{table}

\newpage
\subsubsection{Sprint 2: 04. Oktober 2024 - 01. November 2024}

\textbf{Sprintziel:}
\begin{itemize}
    \item Evaluation der Lösungsprinzipien
    \item Auswahl der optimalen Lösungeskombinationen
\end{itemize}

\textbf{Sprintbacklog:} Der Sprintbacklog von Sprint 2 ist in Tabelle \ref{table:sprint2-backlog} dargestellt.


\begin{table}[H]
\centering
\small
\begin{tabular}{|l|l|l| l|}
\hline
  \textbf{Nr.} & \textbf{Titel} & \textbf{Beschreibung} & \textbf{Size}\\
  \hline
  1  & \textbf{Evaluation Lösungsvarianten} &&\\
  \hline
  1.1  & Erstellung Muss-Kriterien & \makecell{Anforderungen an Technologien, die zwingend\\ erfüllt sein müssen.} & S\\
  \hline
  1.2 & Evaluation anhand Muss-Kriterien & \makecell{Lösungsvarianten anhand Muss-Kriterien evaluieren}& M\\
  \hline
  1.3 & Morpholigischer Kasten & \makecell{Morphologischer Kasten erstellen, um \\Lösungskombinationen zu ermitteln.} & L\\
  \hline
  1.4 & Nutzwertanalyse & \makecell{Nutzwertanalyse durchführen, um passendste\\ Lösungskombinationen zu ermitteln.} & L\\
  \hline
  2 & \textbf{Simulator} && \\
  \hline
  2.1 & Entwicklungsumgebung & Entwicklungsumgebung erstellen & S \\
  \hline
  2.2 & Wegfindung implementieren & Wegfindung in einem Graphen implementieren. & M \\
  \hline
  2.3 & Hinderniserkennung & \makecell{Implementieren, dass Hindernisse erkannt und \\unterschieden werden. Die Reaktion ist je nach Hindernis\\ anders.}& M \\
  \hline

\end{tabular}
\caption{Sprint 2 Backlog}
\label{table:sprint2-backlog}
\end{table}

\newpage
\subsubsection{Sprint 3: 01. November 2024 - 06. Dezember 2024}

\textbf{Sprintziel:}
\begin{itemize}
    \item Dokumentation ist zu 80\% fertiggestellt
    \item Simulator ist fertiggstellt
    \item Freigabe des Gesamtkonzepts
\end{itemize}

\textbf{Sprintbacklog:} Der Sprintbacklog von Sprint 2 ist in Tabelle \ref{table:sprint2-backlog} dargestellt.

!!!!!TODO: einzelne Funktionen besprechen

\begin{table}[H]
\centering
\small
\begin{tabular}{|l|l|l| l|}
\hline
  \textbf{Nr.} & \textbf{Titel} & \textbf{Beschreibung} & \textbf{Size}\\
  \hline
  1  & \textbf{Gesamtkonzept} &&\\
  \hline
  1.1  & Konzept Fortbewegung (M) &  Das Konzept der Fortbewegung wird definiert. & M\\
  \hline
  1.2  & Konzept Lenkung (M) & \makecell{Die Lenkung des Roboters wird definiert.}& M\\
  \hline
  1.3 & Konzept Hindernisse bewegen (M/ET) & \makecell{Es wird definiert, wie Hindernisse bewegt werden sollen.}& M\\
  \hline
  1.4 & Konzept Stromversorgung (ET) & \makecell{Die Stromversorgung wird definiert.} & M\\
  \hline
  1.5 & Konzept Antriebe (ET) & \makecell{Der Antrieb wird definiert.} & M\\
  \hline
  1.6 & Konzept Objekterkennung (ET) & \makecell{Es wird definiert, wie Objekte erkannt werden.} & M\\
  \hline
  1.7 & Konzept Steuerung (ET/I) & \makecell{Es wird definiert, wie der Roboter gesteuert wird.} & M\\
  \hline
    1.8 & Konzept Wegfindung (I) & \makecell{Es wird definiert, wie der Weg, den der Roboter\\ gehen soll, ausgewählt wird.}  & M\\
\hline
    1.9 & Konzept Bilderkennung (I) & \makecell{Es wird definiert, wie das Wegenetzwerk\\ und die Hindernisse erkannt werden.} & M\\
\hline
    1.10 & Konzept I/O (M/ET/I) & \makecell{Es wird definiert, wie das Ziel ausgewählt\\ wird und wie kommuniziert wird, \\dass der Roboter am Ziel angekommen ist.} & M\\
\hline
  1.11  & Risikobewertung & \makecell{Die Risikobewertung für das Gesamtkonzept\\ wird durchgeführt.}& M \\
  \hline 
  2  & \textbf{Simulator} &&\\
  \hline
  2.1  & Simulator wird fertiggestellt &  Die Entwicklung des Simulators wird abgeschlossen. & M\\
  \hline
  
\end{tabular}
\caption{Sprint 3 Backlog}
\label{table:sprint3-backlog}
\end{table}

\newpage
\subsubsection{Sprint 4: 06. Dezember 2024 - 10. Januar 2025}

\textbf{Sprintziel:}
\begin{itemize}
    \item Lösung und Gesamkonzept werden präsentiert
    \item Dokumentation wird fertiggestellt und abgegeben
\end{itemize}

\textbf{Sprintbacklog:} Der Sprintbacklog von Sprint 4 ist in Tabelle \ref{table:sprint4-backlog} dargestellt.

\begin{table}[H]
\centering
\small
\begin{tabular}{|l|l|l| l|}
\hline
  \textbf{Nr.} & \textbf{Titel} & \textbf{Beschreibung} & \textbf{Size}\\
  \hline
  1  & \textbf{Dokumentation} &&\\
  \hline
  1.1  & Fertigstellung Dokumentation & \makecell{Die Dokumentation wird fertiggestellt} & M\\
  \hline
  2 & \textbf{Präsentation} && \\
  \hline
  2.1 & Präsentation vorbereiten & Gesamtkonzept zusammenfassen. & M \\
  \hline
  2.2 &Präsentation halten & Gesamtkonzept präsentieren. & S \\
  \hline
\end{tabular}
\caption{Sprint 4 Backlog}
\label{table:sprint4-backlog}
\end{table}
