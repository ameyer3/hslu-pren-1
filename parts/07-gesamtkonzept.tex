\section{Gesamtkonzept}

TODO Einletung zu Konzept

\subsection{Visualisierung}

\begin{figure}[H]
\centering
\includegraphics[width=\textwidth]{assets/gesamtkonzept/Skizze-Fahrzeugkonzept-Beschriftet.jpg}
\caption{Konzeptskizze Gesamtkonzept}
\label{fig:robot_concept-scetch_labeld}
\end{figure}


\subsection{Komponenten}

Folgendes Komponenten für das Konzept wurden mithilfe der Technologierecherche (siehe Anhang \ref{techrecherche}) und anschliessenden morphologische Kästen (siehe Anhang \ref{mk}) und Nutzwertanalysen (siehe Anhang \ref{nutzwertanalyse}) ermittelt. 

\begin{table}[H]
\centering
\includegraphics[width=\textwidth -20mm]{assets/MK-all.pdf}
\caption{Morphologischer Kasten: Gesamtkonzept}
\label{table:mk-all}
\end{table}

Es ist geplant einen Roboter in U-Form zu bauen, der sich mit drei Rädern fortbewegt und eine Motorlenkung besitzt. Hindernisse kann der Roboter mit einem Parallelgreifer anheben. Dies ist ersichtlich in der vorherigen Abbildung \ref{fig:robot_concept-scetch_labeld}.

Die Steuerung wird auf einem TinyK22 laufen. Die Distanz zu den Objekten wird mit Ultraschall erkannt. Die bürstenbehafteten Motoren werden mit einem gekauften Treiber angesteuert. Die Stromversorgung läuft über einen Akku. Der Akkustand und der Status des Roboters werden mit LED's angezeigt. Damit der Roboter die Linien erkennt, wird ein Liniensensor mit Fototransistoren verwendet. Das Ziel wird über einen Schalter vom Benutzer ausgewählt.
Wenn der Roboter das Ziel erreicht, verkündet er dies über einen Lautsprecher. Im Notfall wird der Roboter über einen Taster ausgeschaltet.

Die Bildverarbeitung und Navigation werden in Python geschrieben und laufen auf einem Raspberry Pi. Zur Bilderkennung wird eine Kombination von PyTorch und OpenCV verwendet. Die Bilder werden mit einer Raspberry Camera aufgenommen. Der kürzeste Weg wird mit einem Dijkstra Algorithmus berechnet.


\subsection{Ablauf}

Der Ablauf einer Durchfahrt ist im Ablaufdiagramm \ref{fig:ablaufdiagramm} aufgezeigt.
Die Schritte, die mit einem Plus markiert sind, sind in den folgenden Kapiteln als Subprozesse detailliert definiert.

\begin{figure}[H]
\centering
\includegraphics[width=\textwidth]{assets/gesamtkonzept/ablaufdiagramm.png}
\caption{Ablaufdiagramm}
\label{fig:ablaufdiagramm}
\end{figure}





\subsubsection{Ausgehende Kanten erkennen}

\begin{figure}[H]
\centering
\includegraphics[width=\textwidth]{assets/gesamtkonzept/ablaufdiagramm-kanten-erkennen.png}
\caption{Ablaufdiagramm ausgehende Kanten erkennen}
\label{fig:ablaufdiagramm-kanten-erkennen}
\end{figure}

Der Roboter bewegt sich auf einen Knoten zu und hält 15 cm vor diesem an. Der Knoten wird fotografiert und der Roboter fährt anschliessend auf den Knoten. Das Bild wird rotiertet und mit OpenCV zuerst so verzerrt, sodass der Knoten gerade von oben dargestellt wird. Danach werden die einzelnen Winkel gemessen. Dies ist im Prototyping Kapitel im Anhang \ref{winkelerkennung} ausführlicher beschrieben. Das Resultat dieser Objekterkennung ist eine Liste mit Winkeln.

\begin{figure}[H]
\begin{subfigure}{0.45\textwidth}
\centering
\includegraphics[width=0.95\linewidth]{assets/informatik-prototyp/opencv/angle_detection/image_taken_by_pi_camer_before_node.jpg} 
\caption{Knoten mit 15cm Abstand}
\label{fig:node-15cm-before}
\end{subfigure}
\begin{subfigure}{0.45\textwidth}
\includegraphics[width=0.95\linewidth]{assets/informatik-prototyp/opencv/angle_detection/node_with_edge_angles_annotated.png} 
\caption{Knoten mit gemessenen Winkeln}
\label{fig:node-angles}
\end{subfigure}

\caption{Winkelerkennung}
\label{fig:angle-recognition}
\end{figure}

Die erhaltene Liste mit Winkeln wird nun verwendet, um mögliche fehlende Linien zu erkennen und die Winkel intern zu speichern. Die internen Winkel werden verwendet, um den Roboter richtig auszurichten, wenn dieser auf die nächste Linie fährt.

Der Roboter selber hat einen Grundgraphen mit Winkeln gespeichert. Diese Winkel ergeben sich aus dem zur Verfügung gestellten Graph.
Da der Graph nicht genau so aufgeklebt sein wird, wie auf der Skizze, wurden Bereiche definiert, in denen sich die Winkel befinden sollten.

Ein Ausschnitt, der zeigt, wie der interne Graph in einem YAML File definiert ist, ist hier eingefügt.

\begin{verbatim}
C: [{D: [60, [30, 120]}, {B: [240, [120, 30]}, {H: [60, [30, 30]}]
\end{verbatim}

Auf Grafik a) sind alle Winkel eingezeichnet inklusive der Halbwinkel zwischen allen Kanten. In Grafik b) ist die Kante zwischen C und D gezeigt, dies entspricht \verb|'C: [{D: [60, [30, 120]}'| im YAML File. 
Wenn der Roboter sich von Knoten C zu Knoten D bewegen möchte, dann wäre diese Kante im Idealfall 60\textdegree\ von der Kante links aus. Das der Idealfall nicht eintreten wird, sind die Halbwinkel eingezeichnet, hier 30\textdegree\ und 120\textdegree. Von der linksliegenden Kante aus, darf kann sich die Kante zu Knoten D also im Bereich von 60\textdegree-30\textdegree\ und 60\textdegree+120\textdegree, sprich 30\textdegree\ und 180\textdegree\ befinden. Befindet sich eine gemessene Kante in diesem Bereich, wird es die Kante C zu D sein.

\begin{figure}[H]
\begin{subfigure}{0.65\textwidth}
\includegraphics[width=0.95\linewidth]{assets/informatik-prototyp/graph-angles.png} 
\caption{Graph mit Winkeln und Winkelbereichen}
\label{fig:angled-graph}
\end{subfigure}
\begin{subfigure}{0.32\textwidth}
\includegraphics[width=0.95\linewidth]{assets/informatik-prototyp/c-d-angle-labeled.png} 
\caption{C zu D Winkel}
\label{fig:excerpt-angled-graph}
\end{subfigure}
\caption{Winkel im Graphen}
\label{fig:angles}
\end{figure}

Als erstes wird die Länge der Liste mit Winkeln gemessen. Wenn sich gleich viele Winkel darin befinden, wie es laut dem Grundgraphen ausgehende Linie haben soll, dann bedeutet das, dass keine Linie fehlt. In diesem Fall werden die internen Winkel aktualisiert mit den tatsächlichen Werten. 

Falls es weniger Winkel gibt als erwartet, werden die erhaltenen Winkel zu den einzelnen möglichen Bereichen zugeordnet. In diesem Bereich, dem kein Winkel zugeordnet wurde, fehlt eine Linie. Folglich aktualisiert der Roboter seine internen Informationen: eine Linie wird aus dem Grundgraphen entfernt und die anderen Winkel werden mit den gemessenen Werten ersetzt.

Nachdem der Roboter den nächsten Knoten berechnet hat, wird der Winkel zur richtigen ausgehenden Kante an die Steuerung gesendet. Der Roboter dreht sich, um auf dieser Linie weiterzufahren.

\subsubsection{Zielknoten erkennen}

Damit der Roboter überprüfen kann, ob er wirklich das Ziel erreicht hat, muss erkannt werden können, ob der Knoten der richtige Zielknoten ist. Dazu wird der \acrfull{orb} Algorithmus verwendet, der Teil der OpenCV Bibliothek ist.\gls{orb-gloss} lernt die Merkmale von Objekten und erkennt auf einem Bild, welches Objekt gefunden wurde.

Der Algorithmus wird so eingesetzt werden, dass der Roboter bevor er den Lauf beginnt die Merkmal der drei Buchstaben lernt. Jedes Mal, wenn ein Knoten fotografiert wird, um die ausgehenden Kanten zu prüfen, wird ebenfalls geprüft, ob sich ein Buchstaben darauf befindet. Falls ein Buchstaben detektiert wird, bestimmt ORB um welchen es sich handelt.

Dadurch dass ORB die Merkmal der Buchstaben detektiert, ist es nicht nötig, die Rotation der Buchstaben zu beachten. Unabhängig von welcher Richtung der Roboter den Zielknoten fotografiert, kann der Buchstabe erkannt werden.

Auf der folgenden Grafik \ref{fig:orb-zielknoten-konzept} wird mit den farbigen Linien dargestellt, wie ORB die bekannten Merkmale (jeweils linkes Bild) in den zu analysierenden Bildern (jeweils rechtes Bild) findet.

\begin{figure}[H]
\begin{subfigure}{0.3\textwidth}
\includegraphics[width=0.95\linewidth]{assets/informatik-prototyp/opencv/target_node_detection/orb-a.png} 
\caption{ORB A}
\label{fig:orb-a-konzept}
\end{subfigure}
\begin{subfigure}{0.3\textwidth}
\includegraphics[width=0.95\linewidth]{assets/informatik-prototyp/opencv/target_node_detection/orb-b.png} 
\caption{ORB B}
\label{fig:orb-b-konzept}
\end{subfigure}
\begin{subfigure}{0.3\textwidth}
\includegraphics[width=0.95\linewidth]{assets/informatik-prototyp/opencv/target_node_detection/orb-c.png} 
\caption{ORB C}
\label{fig:orb-c-konzept}
\end{subfigure}

\caption{Zielknotenerkennung mit ORB}
\label{fig:orb-zielknoten-konzept}
\end{figure}


Mit ORB werden nur die Buchstaben A, B oder C erkannt, da der trainierte Algorithmus nur diese drei Zeichen kennt. So kann ausgeschlossen werden, dass ein C beispielsweise als das Klammer Zeichen \verb|(| erkannt werden würde.


\subsubsection{Pylonen und Hindernisse erkennen}

\begin{figure}[H]
\begin{subfigure}{0.45\textwidth}
\includegraphics[width=\textwidth]{assets/gesamtkonzept/ablaufdiagramm-hindernisse-erkennen.png}
\caption{Ablaufdiagramm Hindernis erkennen}
\label{fig:ablaufdiagramm-hindernis-erkennen}
\end{subfigure}
\begin{subfigure}{0.55\textwidth}
\includegraphics[width=\textwidth]{assets/informatik-prototyp/yolo/recognized-images.jpeg}
\caption{YOLOv11 Bilderkennung}
\label{fig:img-recognition-yolo-concept}
\end{subfigure}
\caption{Bilderkennung Hindernisse}
\label{fig:image-detection-obstacles}
\end{figure}

Aus dem vorherigen Schritt der Kanten erkennen, kennt der Roboter alle ausgehenden Linien und deren Position. Er dreht sich nun im Uhrzeigersinn zu jeder Kante und fotografiert diese. Durch das Hochformat der Kameras, sieht er weit und kann auch nur die wichtigen Elemente sehen, sprich, diese, die sich auf der Linie befinden.

Die Bilder werden mit einem YOLO Objekterkennungsalgorithmus ausgewertet. Dabei werden sowohl Knoten, als auch Pylonen und Barrieren erkannt. Das erhaltene Resultat beschreibt, welche Elemente erkannt wurden anhand der definierten Klassen und wo sich diese befinden mit Koordinaten auf dem Bild.
Alle Hindernisse werden intern gespeichert indem die jeweiligen Kanten höher gewichtet werden.


\newpage

\subsubsection{Hindernisse bewegen}
\label{subsubsection:Hindernisse bewegen}

Um ein vorhandenes Hindernis zu bewegen wird ein Mechanismus zum Anheben (nachfolgend Greifer genannt) und das Fahrzeug selbst zum verschieben des Hindernisses verwendet. Dieser Vorgang und der Aufbau des Greifers wird in folgendem Kapitel erläutert.\\

 Die gezeigten Abbildungen des Greifers stammen von einem Prototyp und stellen nicht das finale Design dar. Sie dienen lediglich zur Veranschaulichung der Funktionsweise. Die Dimensionen des Gestänges sowie die Positionen der Lagerstellen sollen für die finale Variante beibehalten werden.\\
 \\
In der Nutzwertanalyse (Anhang \ref{nutzwertanalyse}) hat man sich zum Anheben des Hindernisses für ein Klemm-Design entschieden, welches das Hindernis oben an der längsten Kante an 3 Punkten einspannt (siehe Abb. \ref{fig:obstacle_clamping_concept}). 


\begin{figure}[H]
\centering
\includegraphics[width=0.95\linewidth]{assets/greifer-prototyp/Greifer_Backen_Trimetric.png} 
\caption{Position Klemmbacken}
\label{fig:obstacle_clamping_concept}
\end{figure}

\newpage

Um sowohl das Klemmen als auch das Anheben des Hindernisses mit einem einzelnen Servomotor zu realisieren wurde der Mechanismus als ein Gestänge realisiert (Abb. \ref{fig:gripper_components}).

\begin{figure}[H]
\centering
\includegraphics[width=1.0\linewidth]{assets/greifer-prototyp/Greifer_side_Komponentennamen.png} 
\caption{Komponenten des Greifers}
\label{fig:gripper_components}
\end{figure}

Die Greifbacke ist gelenkig mit dem Greifarm verbunden, welcher von der Pendelstütze geführt wird. Am Ende des Greifarms wird der Arm des Servomotors befestigt sein (der Servomotor ist in der Abbildung durch einen Hebel ersetzt, da der Prototyp noch manuell bedient wird). Die zwei Spannbacken sind am Spannarm befestigt, welcher drehbar am Chassis des Fahrzeugs gelagert ist und durch eine Zugfeder vorgespannt wird (nur die Montagepunkte der Feder sind in der Abbildung \ref{fig:gripper_components} ersichtlich). 

\newpage

Dreht der Servomotor im Uhrzeigersinn, so schwingen Greifarm und Greifbacke nach oben weg. Der Greifer ist geöffnet (Abb. \ref{fig:gripper_opening_side}). So kann das Farzeug auf das Hindernis zu fahren, bis dieses die zwei Spannbacken berührt. An einer der Spannbacken wird ein Endschalter montiert sein (auf den Abbildungen nicht vorhanden), welcher erkennt, wann das Hindernis nahe genug ist.\\
\\
Dreht der der Servomotor aus dieser Position im Gegenuhrzeigersinn, so schwingt die Greifbacke zurück, bis sie in Berührung mit dem Hindernis kommt. Dadurch wird das Hindernis gegen die zwei Spannbacken gedrückt, welche wiederum durch die Vorspannung der Feder gegen das Hindernis drücken. Das Hindernis wird eingeklemmt (Abb. \ref{fig:gripper_gripping_side}).\\
\\
Dreht der Servomotor weiter im Gegenuhrzeigersinn, so dreht nun auch der Spannarm im Gegenuhrzeigersinn mit, da er über das Hindernis vom Greifarm bewegt wird. Die Rotation des Spannarms führt dazu, dass das Hinderniss angehoben wird (Abb. \ref{fig:gripper_lifting_side}). Gleichzeitig wird durch die Rotation des Spannarms die Zugfeder leicht verlängert. Somit wird die Klemmkraft auf das Hindernis erhöht und sichergestellt, dass dieses nicht aus dem Greifer rutscht.

\begin{figure}[H]
\begin{subfigure}{0.55\textwidth}
\includegraphics[width=\textwidth]{assets/greifer-prototyp/Greifer_side_Offen.png}
\caption{öffnen}
\label{fig:gripper_opening_side}
\end{subfigure}
\begin{subfigure}{0.55\textwidth}
\includegraphics[width=\textwidth]{assets/greifer-prototyp/Greifer_side_Klemmen.png}
\caption{klemmen}
\label{fig:gripper_gripping_side}
\end{subfigure}
\begin{subfigure}{0.55\textwidth}
\includegraphics[width=\textwidth]{assets/greifer-prototyp/Greifer_side_Angehoben.png}
\caption{aneheben}
\label{fig:gripper_lifting_side}
\end{subfigure}
\caption{Ablauf Hindernis anheben}
\label{fig:obstacle_gripping_process}
\end{figure}

 \newpage
 
Als Basis zur Auslegung des Greifers dient eine Berechnung der nötigen Klemmkraft, um das Hindernis anzuheben. Dazu wurden ein Haftreibwert von 0.3 zwischen Hindernis und Klemmbacken und eine Sicherheit gegen rutschen von 1.5 angenommen. Mit der Klemmkraft konnte anhand der Geometrie des Greifers eine Feder mit ausreichend hoher Federkonstante ausgewählt werden. Schliesslich wurde zur Auswahl des Servomotors das nötige Drehmoment berechnet, um sowohl die Zugfeder zu verlängern, als auch das Hindernis anzuheben. Die detaillierten Berechnungen sind im Kapitel \textbf{(REF AUSLEGUNG GREIFER)} zu finden. Die Ergebnisse aus den Berechnungen wurden im Kapitel \ref{subsubsection:gripper-prototype-1} anhand eines Prototyps validiert.\\

Das Klemmen und Anheben ist nur ein Teilschritt  zur Beseitigung eines Hindernisses. Nachfolgend wird der gesamte Ablauf erläutert.

\begin{figure}[H]
\centering
\includegraphics[width=0.2\textwidth]{assets/gesamtkonzept/ablaufdiagramm-hindernis-bewegen.png}
\caption{Ablaufdiagramm Hindernis bewegen}
\label{fig:ablaufdiagramm-hindernis-bewegen}
\end{figure}

 Mit einem Ultraschall-Sensor soll das Vorhandensein eines Hindernisses und die ungefähre Distanz dazu bestimmt werden. Zur genauen Bestimmung der Distanz vor dem Greifer wird der Endschalter am Greifmechanismus verwendet.
Greifer und Endschalter werden sich an der Rückseite des Fahrzeugs befinden, der Ultraschallsensor vorne (siehe Abb. \ref{fig:robot_concept-scetch_labeld}). Dadurch muss sich das Fahrzeug nachdem ein Hindernis mittels Ultraschall entdeckt wurde um 180\textdegree\ drehen und langsam rückwärts fahren, bis der Endschalter am Greifer betätigt wird, um das Hindernis anzuheben. Sobald das Hindernis angehoben ist, dreht sich das Fahrzeug wiederum um 180\textdegree\ und fährt 30mm vorwärts, um das Hindernis an dieselbe Stelle zurück zu setzen. Das Fahrzeug steht nach dem Absetzen wieder nach vorne ausgerichtet und kann geradeaus weiter fahren (siehe  Abb. \ref{fig:ablaufdiagramm-hindernis-bewegen}).


\subsection{Fortbewegung}

Das Fahrzeug wird mit zwei seperat angetriebenen Räder angetrieben. Die Motoren besitzen Encoder, somit kann man genau feststellen welche Distanz zurückgelegt wurde. Der Fahrbefehl ensteht aufgrund der Berechnungen der Bilderkennung. Dem Mikrocontroller TinyK22 werden Distanz und Drehwinkel mitgeteilt, wodurch dann die Motoren angesteuert werden können.

\subsection{Linienerkennung}

Mit einem Array von Phototransistoren und Kondensatoren/Widerstände misst man die Entladezeit mittels Mikrocontroller und der Input Capture Funktion. Der Liniensensor dient als Unterstützung damit man die Linie nicht verlässt während dem Fahren. Jedoch ist das Ziel den Robotor gerade vor der Linie zu positionieren, so dass der Liniensensor theoretisch nicht gebraucht wird während des Fahrens. Ein weiterer Nutzen benötigt man um zu prüfen ob man auf einem Knoten steht oder nicht.

\subsection{Distanzsensor}

Um die Distanz zwischen einem Fahrzeug und Hindernis zu detektieren wird ein Ultraschallsensor verwendet. Somit kann man erkennen an welchem Punkt das Fahrzeug eine 180-Grad-Drehung durchführen muss, um sich auf den Hubvorgang vorzubereiten. Ebenfalls wird so ein Zusammenstoss mit einer Pylone verhindert, falls man bei der Wahl des Pfades die falsche Entscheidung getroffen hat.

\subsection{Wegfindung}

Der kürzeste Weg im Graphen vom momentanen Knoten zum Zielknoten wird mit dem Dijkstra Algorithmus\footnote{\url{https://www.w3schools.com/dsa/dsa\_algo\_graphs\_dijkstra.php}} berechnet. Diese wird zum Beginn berechnet und jedes Mal, wenn der Roboter neue Erkenntnisse zum Graph gesammelt hat, welche die Zielführung beeinflussen kann.

Zusätzlich zum zukünftigen Pfad, wird auch der bereits befahrene Pfad gespeichert. Dies dient dazu, dass der Roboter immer in der Lage sein wird im Fehlerzustand, auf den letzten Knoten zurückzufahren und immer noch weiss, wo er sich befindet.

\subsection{Kameraposition}

Die Kamera wird in einer Höhe von 22.5cm und einem Winkel von 56\textdegree\ montiert. Die Position ist fix, das heisst wir benötigen keine schwenkbare Kamera. Dies Kameraposition ist ersichtlich in der folgenden Grafik \ref{fig:camera-position-concept}.
Die Kamera selbst verfügt über ein Horizontales Field of View\footnote{\url{https://en.wikipedia.org/wiki/Field_of_view}} von 66\textdegree. Wir verwenden die Kamera im Hochformat, Somit haben wir im Field of View von 66\textdegree\ sowohl sehr nahe Knoten und Objekte bis zu 10cm im Bild. Aber auch weit entfernte Pylonen, welche 200cm entfernt stehen.

\begin{figure}[H]
    \centering
    \includegraphics[width=1\linewidth]{assets//informatik-prototyp//camera/camera_position.png}
    \caption{Kamera Positionierung}
    \label{fig:camera-position-concept}
\end{figure}



\newpage

\subsection{Schnittstellen zwischen den Kompontenten}

Im Blockschaltdiagramm in Abbildung \ref{Blockdiagramm Steuerung} wird die Hardware der Steuerung aufgezeigt. Die genauen Funktionen des Mikrocontroller sind im Prototyping Kapitel \ref{Blockdiagramm: Schnittstellen zwischen den Komponenten} beschrieben. Im zentralen Punkt steht der Mikrocontroller welcher ein TinyK22 ist. Er steuert und verarbeitet die Signale der verschiedenen Komponenenten wie Motoren, Ultraschallsensor, Liniensensor oder Motortreiber. Zudem ist eine Verbindung über UART zum Raspberry Pi aufgebaut um die Daten der Bilderkennung zu erhalten. Die Stromüberwachung sorgt dafür das keine Überlast entsteht und so die Bauteile beschädigt werden. Die Pfeile zwischen Komponenten und Mikrocontroller zeigen die Aktion die stattfinden. 




\begin{figure}[H]
    \centering
    \includegraphics[width=1\linewidth]{img/Blockdiagramm-ET-drawio.drawio-2.png}
    \caption{Blockdiagramm Steuerung}
    \label{Blockdiagramm Steuerung}
\end{figure}


