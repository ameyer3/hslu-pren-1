\section{Schlussdiskussion}

In den folgenden Kapiteln wird zusammengefasst, was in \acrshort{pren1} bearbeitet wurde.
Dabei werden die Resultate zusammengefasst mit Ausblick auf \acrshort{pren2}.
Ausserdem werden die Kosten von diesem und nächstem Semester aufgezeigt.
Als letztes werden die gesammelten Erfahrungen bezüglich der Arbeiten und der Zusammenarbeit im Team beschrieben.

\subsection{Erfüllung der Anforderungen}

Im Rahmen von \acrshort{pren1} wurde das Konzept des Roboters als erstes in Teilfunktionen zerlegt, daraus wurde eine Technologierecherche duchgefuehrt. Aus den Recherchen wurden mithilfe von morphologischen Kasten fuer die einzelnen Teilbereiche Loesungsvarianten erarbeitet. Diese wurden mit Nutzerwertanalysen bewertet. Die Varianten, die am besten abschlossen, wurden im naechsten Schritt mit Prototyping verfeinert und getestet. Daraus wurde ein Gesamtkonzept erarbeitet und dokumentiert. Ebenfalls wurden 3 physische Prototypen erstellt, die einzelne Teilbereiche des Roboters darstellen. Diese dienen in \acrshort{pren2} dazu, dass damit parallel entwickelt werden kann. Die drei Prototypen sind ein Greifer, ein Kameraturm und ein Fahrwerk.

Das Konzept unseres Roboters hat drei Raeder, wovon zwei mit Moteren angetrieben wird. Diese dienen zur Fortbewegung und zur Lenkung. Fototransistoren bilden den Liniensenor.
Die Hebervorrichtung, die der Roboter braucht, um bewegliche Hindernisse zu beseitigen, wird mit einem Greifer umgesetzt. Mit einem Ultraschallsensor wird die anzuhebende Barriere erkannt. 
Hindernisse werden mit einer Kamera fotografiert und mit Bilderkennung erkannt. So weiss der Roboter, welche Wege befahrbar sind. Die Navigation wurde in einem Simulator simuliert.

Somit wurde das Design fuer den Roboter erfolgreich erstellt und getestet. Mit diesem Design wird der Roboter in der Lage sein, die Anforderungen, die zu Beginn des Projektes erarbeitet wurden (siehe Kapitel \ref{anforderungliste}), zu erfuellen.

\subsubsection{Ausblick PREN 2}

Aus dem erstellten Design wird nun ein Roboter gebaut. Dabei werden die Einzelteile erstellt und moeglichst frueh zusammengefuegt werden, um das Zusammenspiel aller Komponenten testen zu koennen.

\textbf{Ausblick Mechanik}

TODO MECHANIK offene Punkte und Ausblick

\textbf{Ausblick Steuerung}

TODO STEUERUNG offene Punkte und Ausblick

\textbf{Ausblick Navigation}

Bezueglich der Navigation bedeutet das, dass nun die Logik des Simulators mit den Prototypen der Bild- und Objekterkennung zusammengefuegt werden. Die einzelnen Teile der Objekterkennung werden in einem Workflow zusammengefuegt und die Resultate werden in der Simulatorlogik verwendet. Fuer die Bilderkennung wird ein Model traineirt, dass die Hindernisse und Knoten des Wegenetzes erkennen und auswerten kann. Dieses wird ebenfalls in den Workflow integriert.

Dieser erstellte Workflow wird extensiv getestet und bei Bedarf ueberarbeitet werden muessen.

\textbf{Risiken}

Die erarbeiteten Risken werden in PREN 2 weiterhin beruecksichtigt. Dabei wird vor allem auf diese Risiken geachtet, die als die drei groessten Risiken erkannt wurden.

\begin{itemize}
    \item Risiko 2: Knoten werden nicht erkannt.
    \item Risiko 7: Ein Objekt wird fehlerhaft gedeutet und falsche Wege werden intern entfernt.
    \item Risiko 12: Der Roboter wählt einen falschen Pfad, weil die Hinderniserkennung fehlerhaft ist.
\end{itemize}

Zusaetzlich kommt noch das Risiko hinzu des Zeitdruckes. Die Einzelteile muessen parallel erstellt werden und moeglichst frueh kombniert werden, da in diesem Schritt wahrscheinlich noch viele Fehler entdeckt werden, die behoben werden muessen. 

\subsection{Kosten}\label{kosten}

Insgesamt stehen fuer PREN 1 und PREN 2 500.- zur Verfuegung. Im folgenden wird aufgezeigt wie viel davon bereits verwendet wurde und wie viel vorraussichtlich noch naechstes Semester benoetigt wird.

- Kosten
- angefallene Kosten in PREN1
- Ausblick Gesamtkosten

\begin{table}[H]
\centering
\begin{tabularx}\textwidth{|X | X | X | X |}
\hline
  \textbf{Bezeichnung} & \textbf{Anzahl} & \textbf{Stückpreis} & \textbf{Gesamkosten} \\
  \hline
    Getriebemotor mit Encoder & 2 &CHF 29.90 & CHF 59.80\\
  \hline
    Raspberry Pi Camera Module 3 & 1 & CHF 29.90& CHF 29.90\\
  \hline
  Raspberry Pi 5 4GB & 1 & CHF 60.90 & CHF X 60.90\\
  
  \hline
    Servomotor & 1 & CHF 4 & CHF 4\\
        ... & X & CHF X& CHF X\\

    \hline
   Piezo-Buzzer & 1 & CHF 0.92 & CHF 0.92\\



    \hline
Abstandssensor ToF-based & 1 & CHF 5.71 & CHF 5.71\\

    \hline
Ultraschallsensor & 1 & CHF 5.39 & CHF 5.39\\    

    \hline
I/O Expander I2C 8-bit & 1 & CHF 1.57 & CHF 1.57\\

\hline
Motor/Motion/Ignition Controllers \& Drivers AUTOMOTIVE FULL BRIDGE MOSFET & 2 & CHF 3.81 & CHF 7.62\\

\hline
Optical Sensor Development Tools Line Sensor & 2 & CHF 3.00 & CHF 6.00\\


\hline
Pololu QTR-8RC Reflectance Sensor Array & 1 & CHF 8.36 & CHF 8.36\\

  \hline
  \hline
  \textbf{Gesamtkosten} && &CHF 1000\\
  \hline
\end{tabularx}
\caption{Kosten}
\label{table:costs}
\end{table}

\subsection{Lessons Learned}

Wir hatten die Moeglichkeit von Grund auf ein Projekt zu planen. Wir haben gelernt eine Aufgabenstellung zu analysieren und mithilfe eines morphologischen Kastens Loesungsvarianten zu finden und die beste davon zu bestimmen. Durch das viele Prototyping haben wir zum einen mehr ueber die jeweiligen Technologien gelernt und zum anderen haben wir gelernt organisiert unsere Ideen zu testen und aufgrund der Testresultate zu ueberarbeiten.

\acrshort{pren1} war das erste interdisziplinaere Modul fuer uns. Dies stellte fuer uns eine Chance dar.

Wir hatten die Moeglichkeit Einblicke in andere Studiengaenge zu erhalten. Das heisst, dass wir ein oberflaechliches Verstaendnis fuer andere Kompetenzen erlangen konnten und ebenfalls unterschiedliche Herangehensweisen beobachten konnten.
Die Herangehensweise, um eine Loesung fuer ein Problem zu finden, ist bei jeder Person anders. Jedoch konnten klare Unterschiede in unserem Team zwischen den einzelnen Studiengaengen beobachtet werden.
Beispielweise wurde in der Informatik bereits schnell mit dem Entwickeln begonnen, es gab nur eine kurze Designphase, jedoch gab es in der Maschinentechnik grundsaetzlich eine laengere Designphase, bevor gebaut wurde.

Duch diese unterschiedlichen Herangehensweisen konnten wir voneinander lernen.
Wir wissen nun, was zu erwarten ist bei interdisziplinaeren Arbeiten und wir haben neue Arten gelernt, wie wir selber  in Zukunft vorgehen koennten.

Bereits ab sechs Teammitgliedern kann es chaotisch werden, vor allem wenn diese recht klar an unterschiedlichen Punkten arbeitet.
Wir haben zu Beginn eine Projektleiterin festgelegt, um sicherzugehen, dass mindestens eine Person von allen zu einem gewissen GRad weiss, an was gearbeitet wird.
Dies hat definitiv geholfen bei der Organisation. Ebenfalls war es sehr hilfreich, dass wir uns jede Woche getroffen und ausgetauscht haben. Die Kommunikation ist sehr wichtig in Projektarbeiten und dies ist uns gelungen.

Wir sind stolz auf unser finales Design, unsere neu erworbenen Kompetenzen und auf unsere Zusammenarbeit im Team. Waehrend dem letzten Semester haben wir vieles gelernt und wir fuehlen uns bereit fuer \acrshort{pren2}.



