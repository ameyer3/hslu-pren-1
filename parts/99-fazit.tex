\section{Schlussdiskussion}

In den folgenden Kapiteln wird zusammengefasst, was in \acrshort{pren1} bearbeitet wurde.
Dabei werden die Resultate zusammengefasst mit Ausblick auf \acrshort{pren2}.
Ausserdem werden die Kosten von diesem und nächstem Semester aufgezeigt.
Als letztes werden die gesammelten Erfahrungen bezüglich der Arbeiten und der Zusammenarbeit im Team beschrieben.

\subsection{Erfüllung der Anforderungen}

Im Rahmen von \acrshort{pren1} wurde der Roboter in Teilfunktionen zerelgt, daraus wurde eine Technologierecherche duchgefuehrt. Aus den Recherchen wurden mithilfe von morphologischen Kasten fuer die einzelnen Teilbereiche Loesungsvarianten erarbeitet. Diese wurden mit Nutzerwertanalysen bewertet. Die Varianten, die am besten abschlossen, wurden im naechsten Schritt mit Prototyping verfeinert und getestet. Daraus wurde ein Gesamtkonzept erarbeitet und dokumentiert. Ebenfalls wurden 3 physische Prototypen erstellt, die einzelne Teilbereiche des Roboters darstellen. Diese dienen in \acrshort{pren2} dazu, dass damit parallel entwickelt werden kann.

TODO WHAT DID WE PICK NOW

Somit wurde das Design fuer den Roboter erfolgreich erstellt und getestet. Mit diesem Design wird der Roboter in der Lage sein, die Anforderungen, die zu Beginn des Projektes erarbeitet wurden (siehe Kapitel \ref{anforderungliste}), zu erfuellen.

\subsubsection{Ausblick PREN 2}

Aus dem erstellten Design wird nun ein Roboter gebaut. Dabei werden die Einzelteile erstellt und moeglichst frueh zusammengefuegt werden, um das Zusammenspiel aller Komponenten testen zu koennen.

\textbf{Ausblick Mechanik}

TODO MECHANIK offene Punkte und Ausblick

\textbf{Ausblick Steuerung}

TODO STEUERUNG offene Punkte und Ausblick

\textbf{Ausblick Navigation}

Bezueglich der Navigation bedeutet das, dass nun die Logik des Simulators mit den Prototypen der Bild- und Objekterkennung zusammengefuegt werden. Die einzelnen Teile der Objekterkennung werden in einem Workflow zusammengefuegt und die Resultate werden in der Simulatorlogik verwendet. Fuer die Bilderkennung wird ein Model traineirt, dass die Hindernisse und Knoten des Wegenetzes erkennen und auswerten kann. Dieses wird ebenfalls in den Workflow integriert.

Dieser erstellte Workflow wird extensiv getestet und bei Bedarf ueberarbeitet werden muessen.

\textbf{Risiken}

Die erarbeiteten Risken werden in PREN 2 weiterhin beruecksichtigt. Dabei wird vor allem auf diese Risiken geachtet, die als die drei groessten Risiken erkannt wurden.

\begin{itemize}
    \item Risiko 2: Knoten werden nicht erkannt.
    \item Risiko 7: Ein Objekt wird fehlerhaft gedeutet und falsche Wege werden intern entfernt.
    \item Risiko 12: Der Roboter wählt einen falschen Pfad, weil die Hinderniserkennung fehlerhaft ist.
\end{itemize}

Zusaetzlich kommt noch das Risiko hinzu des Zeitdruckes. Die Einzelteile muessen parallel erstellt werden und moeglichst frueh kombniert werden, da in diesem Schritt wahrscheinlich noch viele Fehler entdeckt werden, die behoben werden muessen. 

\subsection{Kosten}\label{kosten}

Insgesamt stehen fuer PREN 1 und PREN 2 500.- zur Verfuegung. Im folgenden wird aufgezeigt wie viel davon bereits verwendet wurde und wie viel vorraussichtlich noch naechstes Semester benoetigt wird.

- Kosten
- angefallene Kosten in PREN1
- Ausblick Gesamtkosten

\begin{table}[H]
\centering
\begin{tabularx}\textwidth{|X | X | X | X |}
\hline
  \textbf{Bezeichnung} & \textbf{Anzahl} & \textbf{Stückpreis} & \textbf{Gesamkosten} \\
  \hline
    Getriebemotor mit Encode & 2 &CHF 29.90 & CHF 59.80\\
  \hline
    Raspberry Pi Camera Module 3 & 1 & CHF 29.90& CHF 29.90\\
  \hline
  Raspberry Pi 5 4GB & 1 & CHF 60.90 & CHF X 60.90\\
  \hline
    ... & X & CHF X& CHF X\\
  \hline
  \hline
  \textbf{Gesamtkosten} && &CHF 1000\\
  \hline
\end{tabularx}
\caption{Kosten}
\label{table:costs}
\end{table}

\subsection{Lessons Learned}

- «Lessons learned», kritische Würdigung der Arbeiten und der Teamzusammenarbeit
-Kompetenzerwerb