\section{Evaluation Lösungsprinzipien}

Aus der vorhergehenden Technologierecherche wird pro Studiengang ein Morphologischer Kasten erstellt. Dieser wird befüllt mit den Teilfunktionen und den recherchierten Technologien. Damit werden die Lösungsprinzipien evaluiert und verschiedene Varianten werden gefunden. Ein Überblick ueber alle Morphologischen Kasten befindet sich im Anhang \ref{Morphologischer Kasten}.

\subsection{Maschinentechnik Morphologischer Kasten}

Auf Basis der Technologierecherche wurden verschiedene Lösungsansätze für die einzelnen Teilfunktionen ausgewählt und im morphologischen Kasten festgehalten. Es wurden nur die Lösungskombinationen markiert in der Nutzwertanalyse verglichen werden.  

\begin{figure}[H]
\centering
\includegraphics[width=\textwidth]{assets/MK-Maschinentechnik.pdf}
\caption{Morphologischer Kasten: Mechanik}
\label{fig:mk-mechanik}
\end{figure}

PLACEHOLDER



\subsection{Elektrotechnik Morphologischer Kasten}

Aus der Technologierecherche wurde folgender Morphologischer Kasten mit drei Varianten für die Elektrotechnik Teilfunktionen erstellt.

\begin{figure}[H]
\centering
\includegraphics[width=\textwidth -5mm]{assets/MK_Elektrotechnik.pdf}
\caption{Morphologischer Kasten: Elektrotechnik}
\label{fig:mk-elektrotechnik}
\end{figure}

In allen drei Varianten ist konsequent ein Mikrocontroller vorgesehen. Einige Komponenten, wie etwa Buzzer und Lautsprecher oder Drehschalter und Tasten, sind größtenteils austauschbar. Spezifische Schaltungselemente wie Strombegrenzer und Spannungsregler wurden hingegen nicht im morphologischen Kasten berücksichtigt, da solche Entscheidungen erst während des Layoutprozesses getroffen werden.


\subsection{Informatik Morphologischer Kasten}

Aus der Technologierecherche wurde folgender Morphologischer Kasten mit drei Varianten für die Informatik Teilfunktionen erstellt.

\begin{figure}[H]
\centering
\includegraphics[width=\textwidth]{assets/MK_Informatik.pdf}
\caption{Morphologischer Kasten: Informatik}
\label{fig:mk-informatik}
\end{figure}


Bei den drei Varianten fällt auf, dass alle Python und den Wegfindealgorithmus Dijkstra verwenden würden. Python wurde immer gewählt, da Bilderkennung und Machine Learning sehr gut mit Python harmonisieren. Viele Libraries, inklusive dieser, die hier ersichtlich sind, sind kompatibel mit Python. Dijkstra wurde immer gewählt, weil dieser Algorithmus simpel und robust ist und die Geschwindigkeit vernachlässigbar ist. Da es im Graphen nur 8 Knoten gib, wird die Berechnung bei jedem Algorithmus schnell sein.

\subsection{Simulator Morphologischer Kasten}


Aus der Technologierecherche wurde folgender Morphologischer Kasten für die Teilfunktonen des Simulators erstellt. Es gibt drei Varianten, die in Frage kommen.

\begin{figure}[H]
\centering
\includegraphics[width=\textwidth]{assets/MK_Simulator.pdf}
\caption{Morphologischer Kasten: Simulator}
\label{fig:mk-simulator}
\end{figure}



In jeder Lösungsvariante wird die Kante nicht entfernt, sondern neu gewichtet bei einem beweglichen Hindernis. Dies liegt daran, dass es zu risikobehaftet wäre im Falle, dass es sonst keinen Weg geben würde. Die spräche gegen das Prinzip der Robustheit.

Ebenfalls wird ein Knoten immer entfernt, falls ein Pylon darauf steht und nicht markiert. Es würde keinen Vorteil bringen, den Knoten weiterhin zu speichern.


\newpage
\section{Auswahl Lösungskombinationen}

Zur Auswahl der passendsten Lösungskombination wird pro Teilbereich eine Nutzwertanalyse durchgeführt. Die Kriterien und deren Gewichtung sind individuell pro Teilbereich, damit sie ideal passen. Die Variante mit der höchsten Punktezahl ist die Variante, die am besten passt.

\subsection{Maschinentechnik Nutzwertanalyse}

Folgende drei Lösungskombinationen aus dem morphologischen Kasten werden in der Nutzwertanalyse untersucht. Lösungskombintaionen die nicht eine mindestpunktzahl in der Nutzwertanalyse erreicht haben wurden nicht im morphologischen Kasten markiert. 

PLACEHOLDER

\subsection{Elektrotechnik Nutzwertanalyse}

Mithilfe des morphologischen Kasten wurden verschiedene Lösungsvarianten aufgezeigt. Die Varianten bestehen aus diversen Komponenten. 


Nun werden die folgenden drei Varianten in der Nutzwertanalyse evaluiert.

\begin{itemize}
    \item Variante A: Das verwenden vom Tiny K22 macht das System echtzeitfähig. Leichte Motoren, Inputs und Outputs einfach gehalten über Leuchten und Schalter. Für die Motorsteuerug werden die Treiber eingekauft.
    \item Variante B: Bei dieser Variante werden die Bauteile Raspberry Pi tauglich verwendet, grob gesagt eine "Plug and Play" Methode. Ebenfalls haben die Motoren ein geringes Gewicht.
    \item Variante C: Durch die Schrittmotoren und die Objekterkennung mit Lidar weist diese Kombination eine hohe Präzision auf. Mit dem Horn als akkustisches Signal wäre dieser Komponent auffällig und einzigartig. Aufgrund der Schrittmotoren wird das ganze schwer, was einen Nachteil ist.
\end{itemize}


\begin{figure}[H]
\centering
\includegraphics[width=\textwidth]{assets/Nutzwertanalyse-ET.pdf}
\caption{Nutzwertanalyse: Elektrotechnik}
\label{fig:nutzwert-ET}
\end{figure}

Aus den Bewertungen aus der Nutzwertanalyse zeigt sich die Variante A als höchst bewertete. Somit wird diese Variante weiter verfolgt und besteht aus folgenden Komponenten.

\begin{itemize}
    \item Rechner: Tiny K22
    \item Objekterkennung: Ultraschall
    \item Motoren: Bürstenbehafteter Motor
    \item Anzeige: LED's 
    \item Akkustische Signalisierung: Lautsprecher
    \item Motorenansteuerung: Treiber einkaufen
    \item Stromversorgung: Akku
    \item Linienerkennung: Fototransistor
    \item Not-Stop: Taster
    \item Inputoption: Schalter
\end{itemize}

\subsection{Informatik Nutzwertanalyse}

Die drei ermittelten Varianten aus dem vorherigen Kapitel werden analysiert und bewertet mithilfe einer Nutzwertanalyse.

Es gibt diese drei Lösungskombinationen, die evaluiert werden.

\begin{itemize}
     \item Variante A: Python verwendet eine Mischung aus PyTorch und OpenCV, um AI und reine Bilderkennung zu verwenden. Dijkstra berechnet den Weg. Das Ganze rechnet auf einem Raspberry Pi mit einer Raspberry Camera angeschlossen.
    \item  Variante B: Python verwendet LiteRT, um mit AI die Umgebung zu erkennen und Dijkstra, um den Weg zu berechnen. Das Ganze rechnet auf einem Coral Dev Board und fotografiert den Graphen mit einer Industriekamera.
    \item Variante C: Python verwenden nur OpenCV zur Bilderkennung und berechnet mit Dijkstra den Weg, es rechnet auf einem Raspberry Pi mit einer Raspberry Camera.
\end{itemize}

TODO BILD

Aus der Nutzwertanalyse kann abgelesen werden, dass die Variante B mit Python, PyTorch und OpenCV am besten passt. Die detaillierte Variante ist eingezeichnet in diesem Morphologischen Kasten.

\begin{itemize}
    \item ...
\end{itemize}

\subsection{Simulator Nutzwertanalyse}

Die drei Varianten, wie der Simulator umgesetzt werden könnte, werden bewertet in folgender Nutzwertanalyse. Die Kriterien und deren Gewichtung sind dieselben, wie bei der Nutzwertanalyse für die Informatik im Roboter. Das Ziel des Simulators ist es, diesen möglichst realistisch und wiederverwendbar für \acrshort{pren2} umzusetzen.


Die folgenden drei Lösungskombinationen, werden im nächsten Schritt evaluiert.

\begin{itemize}
    \item Variante A: Ein Java Simulator liest ein JSON File und speichert es in einem 2D Array. Bei einem beweglichen Hindernis wird die Gewichtung der Kante erhöht, bei einem Pylonen wird der Knoten entfernt. Dijkstra wird mit einer externen Library implementiert. Die Auswahl des Zielknotens erfolgt zufällig und die Aktivitäten des Roboters werden in einem TUI dargestellt.
    \item Variante B: Ein Python Simulator liest ein YAML File und speichert es in einem Dictionary. Bei einem beweglichen Hindernis wird die Gewichtung der Kante erhöht, bei einem Pylonen wird der Knoten entfernt. Dijkstra wird selber implementiert, das Ziel kann vom Menschen und zufällig ausgewählt werden und die Aktivitäten des Roboters sind in einem GUI ersichtlich.
    \item Variante C: Ein Python Simulator liest ein Bild ein und speichert es in einem Graph Datentyp. Bei einem beweglichen Hindernis wird die Gewichtung der Kante erhöht, bei einem Pylonen wird der Knoten entfernt. Eine externe Library implementiert Dijkstra, das Ziel wird manuell ausgwählt und die Aktivitäten des Roboters werden in einem CLI dargestellt.
\end{itemize}

\begin{figure}[H]
\centering
\includegraphics[width=\textwidth]{assets/Nutzwertanalyse-Simulator.pdf}
\caption{Nutzwertanalyse: Simulator}
\label{fig:nutzwert-Simulator}
\end{figure}

Die Nutzwertanalyse zeigt, dass die Variante B: Python mit YAML und einem GUI am besten abschneidet. Diese Variante besteht aus folgenden Teilen.

\begin{itemize}
    \item Programmiersprache: Python
    \item Wegnetz einlesen: YAML
    \item Wegnetz intern speichern: Key-Values
    \item Bewegliche Hindernisse erfassen: Gewichtung
    \item Blockierte Knoten erfassen: Knoten entfernen
    \item Wegfindung: eigene Implementation
    \item Zielauswahl: human input und random
    \item Clientseitige Kommunikation I/O: GUI
\end{itemize}