\section{Einleitung}

In dieser Dokumentation wird im Rahmen von \acrfull{pren1} die Entwicklung eines autonomen Roboters geplant. Dieser Roboter soll in der Lage sein, ein Wegnetzwerk zu erkennen und diesem zu folgen. Auf den einzelnen Wegen können sich Hindernisse befinden. Die eine Art der Hindernisse sind Pylonen. Wenn sich diese auf einem Pfad befinden, kann dieser nicht befahren werden. Die andere Art der Hindernisse können entfernt werden. Damit der Pfad befahrbar ist, müssen diese jedoch wieder an der selben Stelle zurückgestellt werden.
Das Projekt wird in einem interdisziplinären Team durchgeführt. Das Team besteht aus Studenten aus den Studiengängen Maschinentechnik, Elektrotechnik und Informatik. Dadurch sind die Kompetenzen, die benötigt werden für die Planung und die Umsetzung des Projektes vorhanden.

In diesem Bericht werden sowohl die Vorgehensweise, als auch die gesamte Planung des Projektes beschrieben. In der Planung wird beschrieben wie das geplante Gesamtkonzept erarbeitet wurde. Als erstes wurde eine Technologierecherche durchgeführt. Die einzelnen Lösungen wurden evaluiert und kombiniert. Folgend wurden die Lösungskombinationen evaluiert, woraus sich das Konzept ergab.

Diese Planung bildet die Vorbereitung fuer \acrfull{pren2}. In \acrshort{pren2} wird der geplante Roboter entwickelt. Das Modul wird mit einem Wettbewerb abgeschlossen, wobei die einzelne Gruppen ihre Roboter gegeneinander antreten lassen.

Bei der Konzeptentwicklung des Roboters war es das Ziel einen möglichst simplen Roboter zu bauen, der funktioniert. Der Fokus liegt nicht darauf den schnellsten Roboter zu haben, er sollte jedoch effizient sein.