\section{Einleitung}

In dieser Dokumentation wird die Entwicklung eines autonomen Roboters geplant. Dies findet im Rahmen von \acrfull{pren1} statt. Dieser Roboter soll in der Lage sein, ein Wegenetzwerk zu erkennen und diesem zu folgen. Auf dem Weg können sich Hindernisse befinden. Pylonen sind die einen Hindernisse und können nur auf Knoten platziert werden. Befindet sich ein Pylon auf einem Knoten, kann dieser nicht befahren werden. Die andere Art der Hindernisse sind Barrieren. Die Barrieren befinden sich auf den Verbindungsstrecken zwischen den Knoten und dürfen durch den Roboter entfernt werden, damit der Pfad befahrbar wird. Der Roboter muss die Barrieren nach dem Anheben am gleichen Ort wieder abstellen. 
Das Projekt wird in einem interdisziplinären Team durchgeführt. Das Team besteht aus Studenten der verschiedenen Studiengänge \acrfull{maschinentechnik}, \acrfull{elektrotechnik} und \acrfull{informatik}. Alle für die Planung und Umsetzung benötigten Kompetenzen sind vorhanden.

In diesem Bericht wird als erstes die Aufgabenstellung vorgestellt. Danach wird das Gesamtkonzept des Roboters beschrieben. Dabei wird das Design visualisiert, die einzelnen Komponenten und der Ablauf werden beschrieben und der Simulator, der als Vorbereitung für die Navigation erstellt wurde, wird erklärt. Das 3. Kapitel dient der Nachhaltigkeit, die in Betracht gezogen wurde bei dem Design und der Auswahl der Kompontenten für den Roboter. Anschliessend wird das Projektmanagement beschrieben. Die Vorgehensweise und die Planung für jeden Sprint sind Teil davon.
Am Schluss wird erklärt, inwiefern die Anforderungen erfüllt werden konnten, die Kosten für \acrfull{pren1} und \acrfull{pren2} werden aufgezeigt und es wird eine Reflexion zu dem Gelernten durchgeführt.

Im Anhang wird aufgezeigt wie das geplante Gesamtkonzept erarbeitet wurde. Als erstes wird eine Technologierecherche durchgeführt. Daraus bilden sich mit morphologischen Kasten mehrere Varianten zur Umsetzung, wovon die beste Variante mit Nutzwertanalysen gewählt wird. Das gewählte Konzept wird mit Prototyping geprüft und verbessert.

Bei der Konzeptentwicklung des Roboters ist es das Ziel einen möglichst simplen Roboter zu bauen, der funktioniert. Der Fokus liegt nicht darauf den schnellsten Roboter zu haben, er soll jedoch effizient sein. Das bedeutet, dass er strukturiert vorgeht und sich stabil und sicher fortbewegen kann.

Diese Planung bildet die Vorbereitung für \acrshort{pren2}. In \acrshort{pren2} wird der geplante Roboter gebaut. Das Modul wird mit einem Wettbewerb abgeschlossen, indem die einzelne Gruppen ihre Roboter gegeneinander antreten lassen.

