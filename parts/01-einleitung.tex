\section{Einleitung}

In dieser Dokumentation wird die Entwicklung eines autonomen Roboters geplant. Dies findet im Rahmen von \acrfull{pren1} statt. Dieser Roboter soll in der Lage sein, ein Wegenetzwerk zu erkennen und diesem zu folgen. Auf dem Weg können sich Hindernisse befinden. Pylonen sind Hindernisse und können nur auf Knoten platziert werden. Befindet sich ein Pylon auf einem Knoten, kann dieser nicht befahren werden. Die andere Art der Hindernisse sind Barrieren. Die Barrieren befinden sich auf den Verbindungsstrecken zwischen den Knoten und dürfen vom Roboter entfernt werden, damit der Pfad befahrbar ist. Die Barrieren müssen diese jedoch wieder an der selben Stelle zurückgestellt werden.
Das Projekt wird in einem interdisziplinären Team durchgeführt. Das Team besteht aus Studenten der verschiedenen Studiengänge \acrfull{maschinentechnik}, \acrfull{elektrotechnik} und \acrfull{informatik}. Alle für die Planung und Umsetzung benötigten Kompetenzen sind vorhanden.

TODO THIS ausfuehren 
In diesem Bericht werden sowohl die Vorgehensweise, als auch die gesamte Planung des Projektes beschrieben. In der Planung wird beschrieben wie das geplante Gesamtkonzept erarbeitet wurde. Als erstes wird eine Technologierecherche durchgeführt. Die einzelnen Lösungen werden evaluiert und kombiniert. Folgend werden die Lösungskombinationen evaluiert, woraus sich das Konzept ergibt.

Bei der Konzeptentwicklung des Roboters ist es das Ziel einen möglichst simplen Roboter zu bauen, der funktioniert. Der Fokus liegt nicht darauf den schnellsten Roboter zu haben, er soll jedoch effizient sein. Das bedeutet, dass er strukturiert vorgeht und sich stabil und sicher fortbewegen kann.

Diese Planung bildet die Vorbereitung für \acrfull{pren2}. In \acrshort{pren2} wird der geplante Roboter gebaut. Das Modul wird mit einem Wettbewerb abgeschlossen, indem die einzelne Gruppen ihre Roboter gegeneinander antreten lassen.

