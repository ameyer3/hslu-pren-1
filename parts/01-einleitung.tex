\section{Einleitung}

In dieser Dokumentation wird im Rahmen von \acrfull{pren1} die Entwicklung eines autonomen Roboters geplant. Dieser Roboter soll in der Lage sein, ein Wegnetzwerk zu erkennen und diesem zu folgen. Auf den einzelnen Wegen können sich Hindernisse befinden. Die eine Art der Hindernisse sind Pylonen. Wenn sich diese auf einem Knoten befinden, kann dieser nicht befahren werden. Die andere Art der Hindernisse sind Barrieren. Diese dürfen entfernt werden. Damit der Pfad befahrbar ist, müssen diese jedoch wieder an der selben Stelle zurückgestellt werden.
Das Projekt wird in einem interdisziplinären Team durchgeführt. Das Team besteht aus Studenten aus den Studiengängen \acrfull{maschinentechnik}, \acrfull{elektrotechnik} und \acrfull{informatik}. Dadurch sind die Kompetenzen, die benötigt werden für die Planung und die Umsetzung des Projektes vorhanden.

In diesem Bericht werden sowohl die Vorgehensweise, als auch die gesamte Planung des Projektes beschrieben. In der Planung wird beschrieben wie das geplante Gesamtkonzept erarbeitet wurde. Als erstes wird eine Technologierecherche durchgeführt. Die einzelnen Lösungen werden evaluiert und kombiniert. Folgend werden die Lösungskombinationen evaluiert, woraus sich das Konzept ergibt.

Diese Planung bildet die Vorbereitung für \acrfull{pren2}. In \acrshort{pren2} wird der geplante Roboter gebaut. Das Modul wird mit einem Wettbewerb abgeschlossen, indem die einzelne Gruppen ihre Roboter gegeneinander antreten lassen.

Bei der Konzeptentwicklung des Roboters ist es das Ziel einen möglichst simplen Roboter zu bauen, der funktioniert. Der Fokus liegt nicht darauf den schnellsten Roboter zu haben, er soll jedoch effizient sein. Das bedeutet, dass er strukturiert vorgeht und sich stabil und sicher fortbewegen kann.
