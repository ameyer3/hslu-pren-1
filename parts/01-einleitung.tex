\section{Einleitung}

In dieser Dokumentation wird die Entwicklung eines autonomen Roboters geplant. Dies findet im Rahmen von \acrfull{pren1} statt. Dieser Roboter soll in der Lage sein, ein Wegenetzwerk zu erkennen und diesem zu folgen. Auf dem Weg können sich Hindernisse befinden. Pylonen sind die einen Hindernisse und können nur auf Knoten platziert werden. Befindet sich ein Pylon auf einem Knoten, kann dieser nicht befahren werden. Die andere Art der Hindernisse sind Barrieren. Die Barrieren befinden sich auf den Verbindungsstrecken zwischen den Knoten und dürfen vom Roboter entfernt werden, damit der Pfad befahrbar wird. Die Barrieren müssen diese jedoch wieder an der selben Stelle zurückgestellt werden.
Das Projekt wird in einem interdisziplinären Team durchgeführt. Das Team besteht aus Studenten der verschiedenen Studiengänge \acrfull{maschinentechnik}, \acrfull{elektrotechnik} und \acrfull{informatik}. Alle für die Planung und Umsetzung benötigten Kompetenzen sind vorhanden.

In diesem Bericht wird als erstes die Aufgabenstellung vorgestellt. Danach wird das Gesamtkonzept des Roboters beschrieben. Dabei wird das Design visualisiert, die einzelnen Komponenten und der Ablauf werden beschrieben und der Simulator, der als Vorbereitung fuer die Navigation erstellt wurde, wird erklaert. Das 3. Kapitel dient der Nachhaltigkeit, die in Betracht gezogen wurde bei dem Design und der Auswahl der Kompontenten fuer den Roboter. Anschliessend wird das Projektmanagement beschrieben. Die Vorgehensweise und die Planung fuer jeden Sprint sind Teil davon.
Am Schluss wird erklaert, inwiefern die Anforderungen erfuellt werden konnten, die Kosten fuer \acrfull{pren1} und \acrfull{pren2} werden aufgezeigt und es wird eine Reflexion zu dem Gelernente durchegfuehrt.

Im Anhang wird aufgezeigt wie das geplante Gesamtkonzept erarbeitet wurde. Als erstes wird eine Technologierecherche durchgeführt. Daraus bilden sich mehrere Varianten zur Umsetzung, wovon die beste Variante mit Nutzwertanalysen gewaehlt wird. Das gewaehlte Konzept wird mit Prototyping geprueft und verbessert.

Bei der Konzeptentwicklung des Roboters ist es das Ziel einen möglichst simplen Roboter zu bauen, der funktioniert. Der Fokus liegt nicht darauf den schnellsten Roboter zu haben, er soll jedoch effizient sein. Das bedeutet, dass er strukturiert vorgeht und sich stabil und sicher fortbewegen kann.

Diese Planung bildet die Vorbereitung für \acrshort{pren2}. In \acrshort{pren2} wird der geplante Roboter gebaut. Das Modul wird mit einem Wettbewerb abgeschlossen, indem die einzelne Gruppen ihre Roboter gegeneinander antreten lassen.

