\subsection{Technologierecherchen}\label{techrecherche}

Im folgenden Kapitel sind die Technologierecherchen aufgezeigt. Dabei ging es darum sich einen Überblick über mögliche Varianten der Umsetzung zu verschaffen und die einzelnen Technologien neutral miteinander zu vergleichen.

Bevor mit den Technologierecherchen begonnen wurde, wurden zum einen Kriterien definiert, die unser Roboter erfüllen soll, damit die Recherche eingeschränkt wird und zum anderen wurde der Roboter in Teilfunktionen zerlegt.

Als erstes wurden Muss-Kriterien definiert. Es wurde ein Brainstorming durchgeführt, um mögliche Formen des Roboters zu ermitteln. Dabei wurde versucht möglichst offen zu sein und alle Möglichkeiten in Betracht zu ziehen. Danach wurden einige Möglichkeiten ausgeschlossen. Aus diesem Ausschlussverfahren konnte ermittelt werden, auf welche Kriterien wir für unseren Roboter Wert legen. Der Zweck dieser Kriterien ist es, die Technologierecherchen möglichst fokussiert und schlank durchzuführen.
Die Kriterien, die unser Roboter erfüllen soll, sind die folgenden:

\begin{itemize}
    \item Simpel
    \item Nicht zu teuer
    \item Nicht die 30cmx30cmx80cm ausnutzen
    \item Robustheit
\end{itemize}

Als nächstes wurde der Roboter in Teilfunktionen zerlegt. Verschiedene Möglichkeiten der einzelnen Teilfunktionen wurden recherchiert. Dies sind die definierten Teilfunktionen zu den einzelnen 
 Teilbereichen des Roboters zugeordnet:

\begin{table}[H]
\begin{tabularx}\textwidth{X X X X}
    \textbf{Mechanik} & \textbf{Steuerung} & \textbf{Navigation} & \textbf{Simulator} \\
    &&\\
    Fortbewegung  & Eingabeoptionen & Wegfindung & Wegenetz einlesen\\ 
    Anordnung Räder  & Ausgabeoptionen & Bilderkennung & Wegenetz speichern\\ 
    Lenkung  & Stromversorgung &  & Hindernisse erfassen\\ 
    Hebevorrichtung  & Antriebe & & Wegfindung\\ 
    & Objekterkennung & & Zielauswahl\\ 
    & Linienerkennung &  & Client UI\\ 
    &Anzeige&\\
    &Steuerung&\\
    &Not-Stop&\\
\end{tabularx}
\end{table}

Die Technologierecherchen sind in Tabellen aufgezeigt, damit die Informationen möglichst auf einen Blick ersichtlich und einheitlich sind. Alle Links, die verwendet wurden, wurden auf einer Skala von 1-5 bewertet. Eine 1 bedeutet, dass die Quelle nicht hilfreich war und eine 5 bedeutet, dass die Quelle auch im weiteren Verlaufe des Projektes verwendet werden wird.

\subsubsection{Mechanik}

Die Mechanik besteht aus den vier Teilfunktionen Werkstoffe, Fortbewegungsart, Lenkung und Hebevorrichtung. Mögliche Technologien wurden zusammengetragen und verglichen.

\textbf{Prototyping Werkstoffe}

In den Folgenden Tabellen werden Materialien verglichen, welche für den Prototypenbau in PREN leicht zugänglich sind. Der Vergleich dient in erster Linie dazu, einen geeigneten Werkstoff für Einzelteile zu bestimmen und um Abschätzungen bezüglich Nachhaltigkeit zu machen. Die Daten stammen aus dem Werkstoffkunde-Unterricht der Maschinentechnik Studierenden und aus Erfahrungen aus PREN2 des letzten Semesters.

%Tabelle Metalle
\begin{table}[H]
\centering
\small
\begin{tabularx}{\textwidth}{|l|X|X|}
\hline
  \textbf{} & \textbf{Aluminium} & \textbf{Stahl} \\
  \hline
  \textbf{Vorteile}  & \makecell{-hohe spezifische Festigkeit\\ -gute Wärmeleitfähigkeit \\ -Normteile verfügbar \\ -recyclebar} & \makecell{-hohe Härte erreichbar \\-hoher E-Modul\\-hohe Festigkeit \\ -Normteile verfügbar\\ -recyclebar}\\ 
  \hline
  \textbf{Nachteile} & \makecell{-Preis \\ -aufwendige Verarbeitung \\} & \makecell{-Preis\\-aufwendige Verarbeitung \\ -Gewicht}\\
  \hline
\end{tabularx}
\caption{Metalle}
\label{table:metals-comparison}
\end{table}


%Tabelle 3D-Druck Thermoplaste
\begin{table}[H]
\centering
\small
\begin{tabularx}{\textwidth}{|l|X|X|}
\hline
  \textbf{} & \textbf{ABS} & \textbf{PLA} \\
  \hline
  \textbf{Vorteile}  & \makecell{-hohe Stossfestigkeit\\ -duktil \\ -Wärmebeständig \\ -rapid Prototyping} & \makecell{-hohe Festigkeit \\-einfach druckbar\\ -rapid Prototyping}\\
  \hline
  \textbf{Nachteile} & \makecell{-schwer druckbar \\ -VOC's  \\} & \makecell{-spröde\\-schlechte Wärmebeständigkeit}\\
  \hline
\end{tabularx}
\caption{3D-Druck Thermoplaste}
\label{table:fdm-thermoplasts-comparison}
\end{table}


%Tabelle Holzwerkstoffe
\begin{table}[H]
\centering
\small
\begin{tabularx}{\textwidth}{|l|X|X|}
\hline
  \textbf{} & \textbf{Holz} & \textbf{Mitteldichte Faserplatte (MDF)} \\
  \hline
  \textbf{Vorteile}  & \makecell{-Biologisch abbaubar\\ -Preis \\ -einfache Verarbeitung} & \makecell{-kann ``gelasert''  werden \\-Preis\\ -rapid Prototyping}\\
  \hline
  \textbf{Nachteile} & \makecell{-brennbar \\ -Variationen in Festigkeit} & \makecell{-brennbar\\-nimmt Feuchtigkeit auf\\-nicht recyclebar}\\
  \hline
\end{tabularx}
\caption{Holzwerkstoffe}
\label{table:woods-comparison}
\end{table}

\newpage


\textbf{Fortbewegung}

Es gibt verschiedenste Arten einen Roboter autonom fortzubewegen. Aufgrund der vielen Möglichkeiten wurden bei einem Brainstorming im Team Varianten die zu Komplex oder aufwendig in der Umsetzung sind bereits ausgeschlossen, ohne Recherche zu betreiben. Unten aufgeführt sind die beiden Varianten, welche weiter verfolgt wurden.

% Tabelle Fortbewegung
\begin{table}[H]
\centering
\small
\begin{tabularx}{\textwidth}{|l|X|X|}
\hline
\textbf{} & \textbf{Raupen} & \textbf{Räder} \\
  \hline
  \textbf{Beschreibung}  & FZ wird über Ketten/Raupen angetrieben & Fahrzeug wird über Räder bewegt \\
  \hline
  \textbf{Vorteile}  & \makecell{-Punktwenden möglich\\-geringer spez. Bodendruck\\-hohe Traktion} & \makecell{-simpler Aufbau\\-kostengünstig\\-in vielen Varianten verfügbar\\-selber herstellbar}  \\
  \hline
  \textbf{Nachteile} & \makecell{-komplex\\-Gewicht\\-tiefe Vmax} & \makecell{-limitierte Kontaktfläche mit Boden\\-hoher spez. Bodendruck} \\
  \hline
  \textbf{Links} & \url{https://spacecraft.ssl.umd.edu/academics/788XF22/788XF22L23.tracks.pdf} | 4 & \url{https://gtfrobots.com/introduction-to-types-of-robot-wheels/} | 3\\
  \hline
\end{tabularx}
\caption{Fortbewegung}
\label{table:drive}
\end{table}



\textbf{Lenkung}

Hier wurden diverse Möglichkeiten ein Fahrzeug zu lenken verglichen. 

% Tabelle Lenkung
\begin{table}[H]
\centering
\small
\begin{tabularx}{\textwidth}{|l|X|X|}
\hline
\textbf{} & \textbf{Mecanumwheel} & \textbf{Überlagerungslenkgetriebe} \\
  \hline
  \textbf{Beschreibung}  & Rad, welches in der Ebene Bewegungen in alle Richtungen ermöglicht & ``Panzerlenkung'' 1 Motor mit 2 Ausgangswellen, Drehzahl/Richtung unabhängig einstellbar \\
  \hline
  \textbf{Vorteile}  & \makecell{-reine Translation möglich\\-Punktwende möglich\\ -kein Lenkgestänge nötig} & \makecell{-1 Antriebsmotor\\-1 Lenkmotor\\-kein Lenkgestänge}  \\
  \hline
  \textbf{Nachteile} & \makecell{-komplexer Aufbau\\-schlechte tranktion\\-nicht robust} & \makecell{-Getriebe mit Verlusten\\-komplexer Aufbau\\-schlupf nötig zum Lenken} \\
  \hline
  \textbf{Links} & \url{https://www.generationrobots.com/media/Mecanum-wheel-application.pdf?srsltid=AfmBOorTpuZt7XtV3qQiRZx3OUkmpaGSvG7AvhzuWbz61Fu4mzv7UaZ2} | 4 & \url{https://de-academic.com/dic.nsf/dewiki/1563250} | 2\\
  \hline
  
  \hline
  \textbf{} & \textbf{Ackermann-Lenkgetriebe} & \textbf{Differential Wheel Drive} \\
  \hline
  \textbf{Beschreibung}  & Lenkung mit unterschiedlichen Einschlagwinkel an den Rädern& 2 einzeln Angetriebene Räder, 1 Schwenkrad \\
  \hline
  \textbf{Vorteile}  & \makecell{-simpler Aufbau\\-1 Aktuator\\-schlupffreie Kurvenfahrt} & \makecell{-kein Lenkgestänge\\-leicht\\-Punktwendung möglich}  \\
  \hline
  \textbf{Nachteile} & \makecell{-begrenzter Einschlag} & \makecell{-schlupf nötig zum Lenken\\-2 Antriebsmotoren} \\
  \hline
  \textbf{Links} & \url{https://the-contact-patch.com/book/road/c0504-ackermann-geometry} | 2 & \url{https://www.roboticsbook.org/S52_diffdrive_actions.html} | 4 \\
  \hline
\end{tabularx}
\caption{Lenkung}
\label{table:steering}
\end{table}


\textbf{Hebevorrichtung}

Zum bewegen der Hindernisse ist ein Mechanismus notwendig. Nachfolgend wurden Technologien zum Greifen und Anheben von Objekten verglichen.

%Tabelle Hebevorrichtung
\begin{table}[H]
\centering
\small
\begin{tabularx}{\textwidth}{|l|X|X|}
\hline
\textbf{} & \textbf{Bionischer Greifer}&\textbf{Parallel Greifer}\\
  \hline
  \textbf{Beschreibung} & Bionischer Greifer zum Aufheben des Hindernis&   Parallelgreifer zu Anheben des Hindernis  \\
  \hline
  \textbf{Vorteile} & \makecell{-gute Haftung durch flexible Kontur \\-Greifkräfte können geregelt werden}&\makecell{-Einfache Bauweise\\ -Erlaubt ungenaue Vorpositionierung}  \\
  \hline
  \textbf{Nachteile}& \makecell{-aufwändige Ansteuerung \\ -teuer in der Anschaffung} &\makecell{-Greifkräfte sind aufwändiger zu Regeln \\ }     \\
  \hline
  \textbf{Links}  &   \url{https://www.festo.com/ch/de/e/ueber-festo/forschung-und-entwicklung/bionic-learning-network/highlights-2010-2012/bionischer-handling-assistent-id_33759/} | 3      &  \url{https://schunk.com/ch/de/greiftechnik/parallelgreifer/c/PUB_8295} | 3 \\
  \hline
\end{tabularx}
\caption{Hebevorrichtung}
\label{table:lifting-components}
\end{table}



\textbf{Normen}

Nachfolgend sind relevante Normen aufgeführt.

%Tabelle Normen
\begin{table}[H]
\centering
\small
\begin{tabularx}{\textwidth}{|l|X|X|}
\hline
\textbf{} & \textbf{Schutzklassen} & \textbf{}\\
  \hline
  \textbf{Beschreibung}&     DIN-Norm DIN EN 60529       &           \\
  \hline
  \textbf{Definition}&       Anfroderungen für IP44 und IP00      &           \\
  \hline
  \textbf{Links}&    \url{https://www.jungheinrich-profishop.ch/ch-de/profi-guide/ip-schutzarten/} | 4      &               \\
  \hline
\end{tabularx}
\caption{Normen}
\label{table:norms}
\end{table}



\newpage
\subsubsection{Steuerung}

Im nachfolgenden Abschnitt werden die für die Aufgabe notwendigen elektrischen Komponente und Systeme aufgezeigt und verglichen.  Ein- und Ausgabemethoden, Stromversorgung, Antriebe, Objekterkennung und die Steuerung sind die wichtigsten Bestandteile der Elektronik um den Anforderungen gerecht zu werden. Damit die einzelnen Schnittstellen zur Mechanik und Navigation gewährleistet sind, benötigt es die passenden Komponente.

\textbf{Eingabeoptionen}

Für die Eingabe des Zielpunkts und das Starten des Geräts werden verschiedene Eingabemöglichkeiten benötigt. In diesem Abschnitt werden zwei Optionen verglichen.

\begin{table}[H]
\centering
\small
\begin{tabularx}{\textwidth}{|l|X|X|}
\hline
  \textbf{} & \textbf{Schalter} & \textbf{Touchscreen} \\
  \hline
  \textbf{Beschreibung}  & Physische Eingabeoptionen wie Ein-/Aus- oder Stellschalter & Eingabe über ein Touchscreen\\
  \hline
  \textbf{Vorteile}  & \makecell{-keine Software benötigt\\-einfache Implementation} & \makecell{-sieht Modern aus \\ -Gewicht \\-Teuer}\\
  \hline
  \textbf{Nachteile} & \makecell{-Gewicht \\-Platzausnutzung} & \makecell{-Software\\-Reaktionsgeschwindigkeit}\\
  \hline
  \textbf{Links} &  \url{https://de.wikipedia.org/wiki/Schalter_(Elektrotechnik)} | 3 & \url{https://www.pi-shop.ch/display} | 3\\
  \hline
\end{tabularx}
\caption{Eingabeoptionen}
\label{table:inputs-compare}
\end{table}

\textbf{Ausgabeoptionen}

Nach Erreichen des Ziels ist eine Rückmeldung an den Operator erforderlich. In diesem Abschnitt werden zwei Optionen miteinander verglichen.

\begin{table}[H]
\centering
\small
\begin{tabularx}{\textwidth}{|l|X|X|}
\hline
  \textbf{} & \textbf{Horn} & \textbf{Leuchte} \\
  \hline
  \textbf{Beschreibung}  & \makecell{Akustische Ausgabe von Angaben} & \makecell{Optische Ausgabe von Angaben}\\
  \hline
  \textbf{Vorteile}  & \makecell{-nicht überhörbar\\-musikfähig} & \makecell{-Billig \\-einfach einbaubar\\-klare Fehlerangabe}\\
  \hline
  \textbf{Nachteile} & \makecell{-Stromversorgung \\-Unterschiede schwer erkennbar} & \makecell{-Stromverbrauch kann hoch sein}\\
  \hline
  \textbf{Links} &  \url{https://de.wikipedia.org/wiki/Summer_(Elektrik) } | 3 & \url{https://de.wikipedia.org/wiki/Leuchtdiode} | 3\\
  \hline
\end{tabularx}
\caption{Ausgabeoptionen}
\label{table:outputs-compare}
\end{table}



\newpage

\textbf{Stromversorgung}

Aufgrund der Aufgabenstellung ist ein Netzanschluss nicht möglich. Daher werden in diesem Abschnitt zwei mobile Stromversorgungen miteinander verglichen.

\begin{table}[H]
\centering
\small
\begin{tabularx}{\textwidth}{|l|X|X|}
\hline
  \textbf{} & \textbf{Akkumulator} & \textbf{Batterie} \\
  \hline
  \textbf{Beschreibung}  & Spannungsstabilisierte Spannungsversorgung & Leicht wechselbare Spannungsversorgung\\
  \hline
  \textbf{Vorteile}  & \makecell{-geeignet für hoher Stromverbrauch\\-Ausgangsspannungsstabilität} & \makecell{-Billig \\-einfach einbaubar\\-Gewicht}\\
  \hline
  \textbf{Nachteile} & \makecell{-Gewicht} & \makecell{-keine Spannungsstabilität\\-keine grosse Ströme möglich}\\
  \hline
  \textbf{Links} &  \url{https://de.wikipedia.org/wiki/Akkumulator } | 3 & \url{https://de.wikipedia.org/wiki/Batterie_(Elektrotechnik)}  | 3 \\
  \hline
\end{tabularx}
\caption{Stromversorgung}
\label{table:power-supply-compare}
\end{table}


\textbf{Antriebe}

Für die Antriebsfunktion benötigt es Antriebe welche unterschiedliche Eigenschaften brauchen. Zum einen für die Fortbewegung des Fahrzeugs und zum anderen für das Heben des Hindernisses. Diese wurden in der Tabelle aufgezeigt.

\begin{table}[H]
\centering
\small
\begin{tabularx}{\textwidth}{|l|X|X|}
\hline
  \textbf{} & \textbf{Bürstenbehafteter Motor} & \textbf{Bürstenloser Motor (BLDC)}\\
  \hline
  \textbf{Beschreibung}  & Einfach und gutes Preis-Leistungsverhältnis & Leichter Motor benötigt jedoch zusätzliche Schaltung\\
  \hline
  \textbf{Vorteile}  & \makecell{-aufgrund linearer Strom-Drehmoment \\Charakteristik gutes Regelverhalten\\-Drehzahleinstellbereich gross} & \makecell{-Belastbar\\-wartungsfrei\\-niedriges Gewicht}\\
  \hline
  \textbf{Nachteile} & \makecell{-schlechte Wärmeableitung\\-Warten von Kommutator und Bürsten}& \makecell{-Sensorsystem notwendig\\-zusätzliche Systemkosten}\\
  \hline
  \textbf{Links} & \url{https://doi.org/10.1007/978-3-658-37423-5} | 5 & \url{https://doi.org/10.1007/978-3-658-37423-5} | 5 \\
  \hline
\end{tabularx}


\begin{tabularx}{\textwidth}{|l|X|X|}
\hline
  \textbf{} & \textbf{Asynchron Motor} & \textbf{Schrittmotoren}\\
  \hline
  \textbf{Beschreibung}  & \makecell{effizienter Motor} & Präziser Motor ohne zusätzliche Steuerung\\
  \hline
  \textbf{Vorteile}  & \makecell{-robust, wartungsfrei \\-tiefe Herstellungskosten} & \makecell{-wartungsfrei\\-kostengünstig\\-Betrieb ohne weiteren Sensoren}\\
  \hline
  \textbf{Nachteile} & \makecell{-zusätzliche Komponenten nötig für \\Drehzahlverstellung\\-unpräzise ohne Steuerelektronik}& \makecell{-Leistungen müssen bekannt sein\\-niedrige Leistungsdichte\\-Gewicht}\\
  \hline
  \textbf{Links} & \url{https://doi.org/10.1007/978-3-658-37423-5} | 5 & \url{https://doi.org/10.1007/978-3-658-37423-5} | 5\\
  \hline
\end{tabularx}
\caption{Antriebe}
\label{table:motor2-compare}
\end{table}

\textbf{Objekterkennung}

Um die Objekte zu erkennen, welche auf dem Wegnetzwerk stehen, werden zwei verschiedene Technologien verglichen. Ultraschall und Infrarot-Sensor sind zwei Möglichkeiten eine Objekterkennung zu realisieren.

\begin{table}[H]
\centering
\small
\begin{tabularx}{\textwidth}{|l|X|X|}
\hline
  \textbf{} & \textbf{Ultraschall} & \textbf{Optoelektrisch}\\
  \hline
  \textbf{Beschreibung}  & Senden und Empfangen von Ultraschall & Senden und Emfpangen von Licht\\
  \hline
  \textbf{Vorteile}  & \makecell{-geringe Störeinflüsse von Material und \\Umwelt\\-grosse Reichweite} & \makecell{-Präzise Messungen möglich}\\
  \hline
  \textbf{Nachteile} & \makecell{-Blindzone (min. Reichweite)} & \makecell{-Störeinflüsse von Licht\\-Materialbeschaffenheit}\\
  \hline
  \textbf{Links} & \url{https://doi.org/10.1007/978-3-658-12562-2} | 5 & \url{https://doi.org/10.1007/978-3-658-12562-2} | 5\\
  \hline
\end{tabularx}

\begin{tabularx}{\textwidth}{|l|X|X|}
\hline
\textbf{} & \textbf{Lidar} & \textbf{} \\
  \hline
  \textbf{Beschreibung} & Senden und Empfangen von Laserstrahlen & \\
  \hline
  \textbf{Vorteile}  & \makecell{-Genaue Messung möglich \\ -Grosse Distanzen} & \makecell{} \\
  \hline
  \textbf{Nachteile} & \makecell{-Kostspielig} & \makecell{} \\
  \hline
  \textbf{Links} & \url{https://de.wikipedia.org/wiki/Lidar} | 3 & \\
  \hline
\end{tabularx}
\caption{Objekterkennung}
\label{table:et-object-detection-compare}
\end{table}


\textbf{Linienerkennung}

Aufgrund der Streckenführung, dass man über eine Linie folgen muss besteht die Aufgabe die Linien zu erkennen. 

\begin{table}[H]
\centering
\small
\begin{tabularx}{\textwidth}{|l|X|X|}
\hline
  \textbf{} & \textbf{Phototransistor} & \textbf{Farbsensor} \\
  \hline
  \textbf{Beschreibung}  & Ein Fototransistor ist abhängig von der Helligkeit der Fläche. Er leitet je nach dem besser oder schlechter. & Ein Farbsensor kann mittels Photodioden und Farbfiltern Farben erkennen.\\
  \hline
  \textbf{Vorteile}  & \makecell{-hohe Empfindlichkeit\\-kostengünstig} & \makecell{-kompakt und einfache Integration}\\
  \hline
  \textbf{Nachteile} & \makecell{-Belichtungsempfindlich} & \makecell{-Belichtungsempfindlich}\\
  \hline
  \textbf{Links} & \url{https://de.wikipedia.org/wiki/Fototransistor} | 3 & \url{https://de.wikipedia.org/wiki/Farbsensor| 3}\\
\hline


\end{tabularx}
\caption{Linienerkennungsmöglichkeiten}
\label{table:et-line-detection-compare}
\end{table}

\textbf{Anzeige}

Damit wichtige Informationen visuell angezeigt werden können werden in dieser Tabelle drei Möglichkeiten aufgezeigt. Die Varianten sind ein Display, \acrshort{led}'s oder eine Balkenanzeige

\begin{table}[H]
\centering
\small
\begin{tabularx}{\textwidth}{|l|X|X|}
\hline
  \textbf{} & \textbf{Display} & \textbf{LED's} \\
  \hline
  \textbf{Beschreibung}  & \makecell{Auf einem Display werden die\\wichtigsten Daten mit Texten angezeigt.} & \makecell{Die LED können aufleuchten um etwas\\ visuell anzuzeigen}\\
  \hline
  \textbf{Vorteile}  & \makecell{-genauere Anzeige mit Beschreibung \\kompakte Darstellung\\-Übersichtlichkkeit} & \makecell{-einfach und robust}\\
  \hline
  \textbf{Nachteile} & \makecell{-Implementierungsaufwand} & \makecell{-keine genauen Daten}\\
  \hline
   \textbf{Links} & \url{https://www.pi-shop.ch/pimoroni-display-o-tron-3000} | 2 & \url{https://de.wikipedia.org/wiki/Leuchtdiode} | 3\\
  \hline
\end{tabularx}

\begin{tabularx}{\textwidth}{|l|X|X|}
\hline
\textbf{} & \textbf{Balkenanzeige} & \textbf{} \\
  \hline
  \textbf{Beschreibung} & Einfache Darstellung über das Aufleuchten einzelner Balken & \\
  \hline
  \textbf{Vorteile}  & \makecell{-präzisere Anzeige als mit LED's \\ -einfach} & \makecell{} \\
  \hline
  \textbf{Nachteile} & \makecell{-keine genauen Daten} & \makecell{} \\
  \hline
  \textbf{Links} & \url{https://www.mouser.ch/ProductDetail/Kingbright/DC10EWA?} | 3 & \\
  \hline
\end{tabularx}
\caption{Anzeigemöglichkeiten}
\label{table:output-compare}
\end{table}

\textbf{Steuerung}

Zum Befahren vom System und Steuerung von Gerät wird eine Logik benötig.

\begin{table}[H]
\centering
\small
\begin{tabularx}{\textwidth}{|l|X|X|}
\hline
  \textbf{} & \textbf{Mikrocontroller} & \textbf{Raspberry Pi} \\
  \hline
  \textbf{Beschreibung}  & \makecell{Steuerung durch echzeitfähiges System} & \makecell{System mit grosser Rechenleistung}\\
  \hline
  \textbf{Vorteile}  & \makecell{-geringer Stromverbrauch\\-echtzeitfähig} & \makecell{-Multithreading möglich \\-Bildverarbeitung möglich}\\
  \hline
  \textbf{Nachteile} & \makecell{-Bildverarbeitung nicht möglich} & \makecell{-Hoher Stromverbrach\\-Materialbeschaffenheit}\\
  \hline
  \textbf{Links} & \url{https://www.elektronik-kompendium.de/sites/com/1907171.html} | 3 & \url{https://www.raspberrypi.com} | 3\\
  \hline
\end{tabularx}
\caption{Steuerung}
\label{table:controller-compare}
\end{table}


\textbf{Not-Stop}

Die Funktion damit man einen Not-Stop ausführen kann, kann mit einem Not-Aus Pilztaster oder einem normalen Schalter gemacht werden.

\begin{table}[H]
\centering
\small
\begin{tabularx}{\textwidth}{|l|X|X|}
\hline
  \textbf{} & \textbf{Not-Aus Taster} & \textbf{Schalter} \\
  \hline
  \textbf{Beschreibung}  & Ein Not-Aus Taster rastet nach Betätigen in seiner Stellung ein. Das zurückstellen erfolgt nicht so einfach wie bei einem normalen Schalter. & Der Schalter wird gedrückt und löst den Not-Stop aus.\\
  \hline
  \textbf{Vorteile}  & \makecell{-Erkennung von Not-Aus\\-Einrastfunktion} & \makecell{-klein und benötigt wenig Druck}\\
  \hline
  \textbf{Nachteile} & \makecell{-muss mit Kraft gedrückt werden} & \makecell{-keine Einrastfunktion}\\
  \hline
  \textbf{Links} &  \url{https://de.wikipedia.org/wiki/Notausschalter} | 3& \url{https://de.wikipedia.org/wiki/Schalter_(Elektrotechnik)} | 3\\
  \hline
\end{tabularx}
\caption{Not-Aus Schalter}
\label{table:not-stop-compare}
\end{table}



\newpage
\subsubsection{Navigation}

Im folgenden Kapitel werden die einzelnen Teilfunktionen der Navigation verglichen: die Wegfindung und die Bilderkennung. Dabei wurden vor allem die Kriterien 'simpel' und 'Robustheit' berücksichtigt. Ansonsten wurde offen recherchiert.

\textbf{Programmiersprache}

Nachfolgend werden zwei verschiedene Programmiersprachen verglichen, die zur Implementation des Navigation Teils verwendet werden könnten. Nur diese beiden Sprachen wurden verglichen aufgrund des Kriteriums der ``Robustheit''. Beide Entwickler:innen kennen sich zusammen am besten mit diesen beiden Sprachen aus, weswegen die Fehleranfälligkeit kleiner ist.

\begin{table}[H]
\centering
\small
\begin{tabularx}{\textwidth}{|l|X|X|}
\hline
\textbf{} & \textbf{Python} & \textbf{Java}\\
  \hline
  \textbf{Beschreibung}  & Interpretierte Programmiersprache & Kompilierte Programmiersprache\\
  \hline
  \textbf{Vorteile}  & \makecell{-Lightweight\\-Beliebt für ML\\-Umfangreiche Bibliotheken\\-Grosser Community Support} & \makecell{-Schnell \\-Möglichkeit für Threading}\\
  \hline
  \textbf{Nachteile} & \makecell{-Langsam} & \makecell{-Heavyweight}\\
  \hline
  \textbf{Links} & \url{https://www.python.org/} | 5 & \url{https://www.java.com/en/} | 3 \\
  \hline
\end{tabularx}
\caption{Programmiersprachen Vergleich}
\label{table:lang-compare}
\end{table}

\textbf{Wegfindung Algorithmus}

Es gibt mehrere Algorithmen, mit denen der kürzeste Weg in einem Wegenetzwerk gefunden werden können. Nachfolgend werden einige miteinander verglichen. 


\begin{table}[H]
\centering
\small
\begin{tabularx}{\textwidth}{|l|X|X|}
\hline
\textbf{} & \textbf{Dijkstras Algorithmus} & \textbf{A* Algorithmus}\\
  \hline
  \textbf{Beschreibung} & Ein algorithmischer Ansatz, der den kürzesten Pfad von einem Startknoten zu allen anderen Knoten in einem gewichteten Graphen findet, ohne Zielinformationen zu berücksichtigen. & Ein Heuristik-basierter Suchalgorithmus, der den kürzesten Pfad zwischen einem Start- und Zielknoten effizient findet, indem er sowohl die tatsächlichen Kosten als auch geschätzte zukünftige Kosten berücksichtigt. \\
  \hline
  \textbf{Vorteile}  & \makecell{-Findet immer den kürzesten Weg \\ -Einfach zu implementieren} & \makecell{-Effizienter als Dijkstra \\ -Flexibilität durch Heuristik \\ -Anwendbar auf viele Probleme}\\
  \hline
  \textbf{Nachteile} & \makecell{-Keine Berücksichtigung von \\ Zielinformationen \\ -Nicht optimal für dynamische Graphen \\ -Langsame Laufzeit} & \makecell{-Speicherintensiv \\-Abhängig von Heuristik}\\
  \hline
  \textbf{Links} & \url{https://algorithms.discrete.ma.tum.de/} | 5  & \url{https://en.wikipedia.org/wiki/A*_search_algorithm} | 3 \\
  \hline
\end{tabularx}
\begin{tabularx}{\textwidth}{|l|X|X|}
\hline
\textbf{} & \textbf{Brute Force} & \textbf{D* (Iterativ)}\\
  \hline
  \textbf{Beschreibung} & Durchsucht alle möglichen Wege im Netzwerk, um den kürzesten Pfad zu finden. Ein exhaustiver Ansatz, der keine Optimierungen oder Heuristiken verwendet. & Mit D* vom Zielknoten zum Startknoten den Weg berechnet und so mehrere Startknoten hat. Wiederholender Ansatz, der schrittweise den besten Pfad verbessert. Häufig verwendet in dynamischen oder iterativen Umgebungen, bei denen der Pfad durch inkrementelle Schritte gefunden wird.\\
  \hline
  \textbf{Vorteile} & \makecell{-Findet garantiert eine Lösung\\wenn es einen gibt.} & \makecell{-Gut für dynamische Graphen} \\
  \hline
  \textbf{Nachteile} & \makecell{-Sehr ineffizient\\-Hoher Rechenaufwand.} & \makecell{-Konstante Berechnung} \\
  \hline
  \textbf{Links} & \url{https://de.wikipedia.org/wiki/Brute-Force-Methode} | 1 & \url{https://en.wikipedia.org/wiki/Iterative_method} | 2\\
  \hline
\end{tabularx}
\caption{Wegfindung Algorithmus Vergleich}
\label{table:path-algo-compare}
\end{table}


\newpage

\textbf{Bilderkennung}

\textbf{Bilderkennung Software:} Nachfolgend werden einige Softwarelösungen verglichen mit denen eine Bilderkennung durchgeführt werden könnten. Dies wird benötigt, um Teile des Wegenetzwerks inklusive Pylonen, Hindernisse und fehlende Streckenteile zu erkennen.

\begin{table}[H]
\centering
\small
\begin{tabularx}{\textwidth}{|l|X|X|}
\hline
\textbf{} & \textbf{Tensorflow} & \textbf{LiteRT} \\
  \hline
  \textbf{Beschreibung}  & TensorFlow ist ein von Google entwickeltes Open-Source-Framework für maschinelles Lernen, das sich durch seine Skalierbarkeit und umfangreichen Bibliotheken auszeichnet. & TensorFlow Lite (LiteRT) ist die abgespeckte Version von TensorFlow, optimiert für die Ausführung auf mobilen und eingebetteten Geräten mit begrenzten Ressourcen. \\
  \hline
  \textbf{Vorteile}  & \makecell{-Gut Dokumentiert\\-Grosse Community\\-Gute Performance} & \makecell{-Optimiert für On-Device ML \\ -Gut Dokumentiert \\ -Gute Performance} \\
  \hline
  \textbf{Nachteile} & \makecell{-Steile Lernkurve \\-Überdimensioniert für einfachere\\ Aufgaben } & \makecell{-Steile Lernkurve} \\
  \hline
  \textbf{Links} & \url{https://www.tensorflow.org} | 2 & \url{https://ai.google.dev/edge/litert}  | 4 \\
  \hline
\end{tabularx}
\begin{tabularx}{\textwidth}{|l|X|X|}
\hline
\textbf{} & \textbf{PyTorch} & \textbf{OpenCV}\\
  \hline
  \textbf{Beschreibung} & PyTorch ist ein von Facebook entwickeltes Open-Source-Framework für maschinelles Lernen, das für seine dynamischen Berechnungsgraphen und einfache Handhabung bekannt ist. & OpenCV ist eine Open-Source-Bibliothek für Computer Vision, die effiziente Algorithmen für Bild- und Videoverarbeitung bereitstellt. \\
  \hline
  \textbf{Vorteile} & \makecell{-Einfacher zu erlernen als TensorFlow \\ -Unterstützt dynamische Graphen \\ (flexibler bei Modellen) \\ -Stark in Forschung und Experimenten \\ -Grosse Community} & \makecell{-Flache Lernkurve \\ -Grosse Community} \\
  \hline
  \textbf{Nachteile} & \makecell{-Weniger ausgereift für die Produktion \\ im Vergleich zu TensorFlow \\ -Hoher Speicherbedarf} & \makecell{-Rechenintensiv \\ -Schlechte Performance \\ -Nicht die beste Wahl für Deep-Learn-\\ing-Fähigkeiten} \\
  \hline
  \textbf{Links} & \url{https://pytorch.org/} | 4  & \url{https://opencv.org/} | 4\\
  \hline
\end{tabularx}
\caption{Vision Objekterkennung Vergleich}
\label{table:vision-object-detection-compare}
\end{table}

\newpage

\textbf{Bilderkennung Kamera:} Nachfolgend werden einige Varianten verglichen, die als Kamera verwendet werden könnten, um die Bilder für die Bilderkennung zu schiessen.

\begin{table}[H]
\centering
\small
\begin{tabularx}{\textwidth}{|l|X|X|}
\hline
\textbf{} & \textbf{Raspberry Camera Modul} & \textbf{USB Webcam}\\
  \hline
  \textbf{Beschreibung} & Kamera Module die speziell für den Einsatz mit dem Raspberry Pi entwickelt sind. Mit einer eigenen CSI Schnittstelle für den Anschluss an einen Single Board Computer & Handelsübliche Webcam wie Sie für Video Meetings eingesetzt wird. \\
  \hline
  \textbf{Vorteile}  & \makecell{-Kompatibilität mit Raspberry Pi \\ -CSI Konnektor} & \makecell{-Grössere Auswahl an Möglichkeiten \\ -Halterung meist inklusive \\ -Schwenkbar und Drehbar} \\
  \hline
  \textbf{Nachteile} & \makecell{-Kein Gehäuse \\ -Keine Halterung} & \makecell{} \\
  \hline
  \textbf{Links} & \url{https://www.raspberrypi.com/documentation/accessories/camera.html} | 4 & \url{https://www.logitech.com/de-ch/shop/c/webcams} | 1 \\
  \hline
\end{tabularx}
\begin{tabularx}{\textwidth}{|l|X|X|}
\hline
\textbf{} & \textbf{Industriekamera} & \textbf{} \\
  \hline
  \textbf{Beschreibung} & Eine Industriekamera wird oft für Vision Anwendungen im Industriebereich eingesetzt. Sie wird in den meisten Fällen via Ethernet Gigabyte Anbindung angeschlossen. In einigen Konfigurationen auch mit USB 3.0 verfügbar. & \\
  \hline
  \textbf{Vorteile}  & \makecell{-Sehr gute Qualität \\ -Objektiv frei wählbar \\ -Konfigurierbar} & \makecell{} \\
  \hline
  \textbf{Nachteile} & \makecell{-Kostspielig \\ -Schlechte Kompatibilität \\ -Keine Halterung inklusive} & \makecell{} \\
  \hline
  \textbf{Links} & \url{https://machinevisionkamera.de/Industriekamera-USB3-Vision} | 2 & \\
  \hline
\end{tabularx}
\caption{Kamera Vergleich}
\label{table:camera-compare}
\end{table}

\textbf{Bilderkennung Rechner:} Nachfolgend werden einige Varianten verglichen, die als Rechner eingesetzt werden könnten. Auf diesem Rechner wird die Bilderkennungssoftware, sowie die Berechnung für Pfadfindung laufen.

\begin{table}[H]
\centering
\small
\begin{tabularx}{\textwidth}{|l|X|X|}
\hline
\textbf{} & \textbf{Jetson Dev Kit} & \textbf{Raspberry Pi}\\
  \hline
  \textbf{Beschreibung} & Ein leistungsstarkes Entwicklerkit von NVIDIA, das für KI- und Robotikanwendungen optimiert ist und auf der Jetson-Plattform basiert. & Ein kostengünstiger, vielseitiger Einplatinencomputer, der für zahlreiche Projekte von Bildung bis hin zu Heimautomation verwendet wird. \\
  \hline
  \textbf{Vorteile}  & \makecell{
  -Hohe Rechenleistung für KI und\\maschinelles Lernen \\
  -GPU-Unterstützung (NVIDIA CUDA) \\
  -Energieeffizient für seine Leistung \\
  -Gute Integration in das NVIDIA-\\Ökosystem
  } & \makecell{
  -Günstiger Preis \\
  -Grosse Community und umfangreiche\\Dokumentation \\
  -Vielseitig für verschiedene\\Anwendungen \\
  -Zahlreiche Erweiterungen und Zubehör
  } \\
  \hline
  \textbf{Nachteile} & \makecell{
  -Höherer Preis im Vergleich zu anderen\\Entwicklerboards \\
  -Begrenzte Community-Unterstützung\\im Vergleich zu Raspberry Pi \\
  -Komplexere Einrichtung für Anfänger
  } & \makecell{
  -Begrenzte Rechenleistung für intensive\\Aufgaben wie KI \\
  -Kein dedizierter Beschleuniger für\\maschinelles Lernen \\
  -Nicht ideal für energieintensive oder\\hochgradig spezialisierte Anwendungen
  } \\
  \hline
  \textbf{Links} & \url{https://developer.nvidia.com/embedded/jetson-developer-kits} | 4 & \url{https://www.raspberrypi.com/} | 5 \\
  \hline
\end{tabularx}
\begin{tabularx}{\textwidth}{|l|X|X|}
\hline
\textbf{} & \textbf{Coral Dev Board} & \textbf{} \\
  \hline
  \textbf{Beschreibung} & Ein Google-basiertes Entwicklerboard, das speziell für schnelle maschinelle Lernanwendungen mit integrierter Edge-TensorFlow-Beschleunigung entwickelt wurde. & \\
  \hline
  \textbf{Vorteile}  & \makecell{
  -Integrierter Edge-TPU für schnelle\\maschinelle Lerninferenz \\
  -Gute Leistung bei ML-Anwendungen\\(Edge AI) \\
  -Google TensorFlow-Unterstützung \\
  -Energieeffizient für ML-Anwendungen
  } & \makecell{} \\
  \hline
  \textbf{Nachteile} & \makecell{
  -Teurer als Raspberry Pi \\
  -Begrenzte allgemeine Verfügbarkeit und\\Zubehör im Vergleich zu Raspberry Pi \\
  -Kleinere Community
  } & \makecell{} \\
  \hline
  \textbf{Links} & \url{https://coral.ai/} | 4 & \\
  \hline
\end{tabularx}
\caption{SW Rechner Vergleich}
\label{table:sw-pc-compare}
\end{table}

\newpage
\subsubsection{Simulator}

Möglichkeiten für die Teilfunktionen des Simulators sind im folgenden Teil aufgelistet und verglichen. Diese Teilfunktionen sind das Wegenetz einlesen und speichern, Hindernisse erfassen, Wegfindung und ein Client User Interface.
Bei der Recherche des Simulators wurde vor allem auf das festgelegte Kriterium 'simpel' geachtet. Abgesehen davon wurde offen recherchiert.

\textbf{Programmiersprache}

Wie bei der Software für den Roboter kommen für den Simulator Java oder Python infrage. Der Vergleich der beiden Programmiersprachen wurde im vorherigen Kapitel durchgeführt.

\textbf{Wegenetzwerk einlesen}

Damit im Simulator das Wegenetzwerk eingelesen werden kann, wurden die folgenden Möglichkeiten in Betracht gezogen.

\begin{table}[H]
\centering
\small
\begin{tabularx}{\textwidth}{|l|X|X|}
\hline
\textbf{} & \textbf{YAML} & \textbf{JSON}\\
  \hline
  \textbf{Beschreibung} & Eine Datenseralisierungssprache die gut vom Mensch gelesen werden kann. &  Eine Sprache für den Datenaustausch. Steht für JavaScript Object Notation. \\
  \hline
  \textbf{Vorteile}  & \makecell{-sehr simpel} & \makecell{-Java ähnlicher Syntax} \\
  \hline
  \textbf{Nachteile} & \makecell{-etwas aufwändiger in Java einzulesen} & \makecell{-weniger gut lesbar als YAML} \\
  \hline
  \textbf{Links} & \url{https://yaml.org/} | 5 & \url{https://www.json.org/json-en.html} | 4 \\
  \hline
\end{tabularx}

\begin{tabularx}{\textwidth}{|l|X|X|}
\hline
\textbf{} & \textbf{Bild} & \textbf{}\\
  \hline
  \textbf{Beschreibung} & Das Wegenetz würde als Graph skizziert werden und vom Simulator eingelesen werden. &   \\
  \hline
  \textbf{Vorteile}  & \makecell{-Realitätsnah} & \makecell{} \\
  \hline
  \textbf{Nachteile} & \makecell{-sehr viel Aufwand} & \makecell{} \\
  \hline
  \textbf{Links} & \url{https://opencv.org/} | 5 &  \\
  \hline
\end{tabularx}
\caption{Wegenetz einlesen Vergleich}
\label{table:read-path-compare}
\end{table}


\textbf{Wegenetz intern speichern}

Das Wegenetzwerk könnte auf folgende Arten gespeichert werden.

\begin{table}[H]
\centering
\small
\begin{tabularx}{\textwidth}{|l|X|X|}
\hline
\textbf{} & \textbf{Key-Value Pairs} & \textbf{2D Array}\\
  \hline
  \textbf{Beschreibung} & Jeder Knoten ist ein Key und hat als Value eine Liste der Konten, mit denen er verbunden ist. &  Es wird eine Adjazenzmatrix erstellt in einem zweidimensionalen Array.\\
  \hline
  \textbf{Vorteile}  & \makecell{-direkte Liste der Nachbaren} & \makecell{-möglich in allen Sprachen} \\
  \hline
  \textbf{Nachteile} & \makecell{-je nach Sprache aufwändiger} & \makecell{-spezifische Markierung von Nicht-\\Verbindungen -> redundante Info} \\
  \hline
  \textbf{Links} & \url{https://www.python.org/doc/essays/graphs/} | 2 & \url{https://de.wikipedia.org/wiki/Adjazenzmatrix} | 3\\
  &\url{https://campus.datacamp.com/courses/data-structures-and-algorithms-in-python/queues-hash-tables-trees-graphs-and-recursion?ex=9} | 1 &\\
  \hline
\end{tabularx}

\begin{tabularx}{\textwidth}{|l|X|X|}
\hline
\textbf{} & \textbf{Externe Library} & \textbf{}\\
  \hline
  \textbf{Beschreibung} & Es wird eine externe Library verwendet, die einen Graph Datentyp anbietet.  &   \\
  \hline
  \textbf{Vorteile}  & \makecell{-hat oft andere nützliche Funktionen} & \makecell{} \\
  \hline
  \textbf{Nachteile} & \makecell{-externe Abhängigkeit\\-eventuell unnütze andere Funktionen} & \makecell{} \\
  \hline
  \textbf{Links} & \url{https://networkx.org/} | 4 &  \\
  \hline
\end{tabularx}
\caption{Wegenetz erfassen Vergleich}
\label{table:store-path-compare}
\end{table}

\textbf{Bewegliche Hindernisse erfassen}

Damit die beweglichen Hindernisse in unserem gespeicherten Netzwerk erfasst werden können, wurden folgende Methoden bewertet.

\begin{table}[H]
\centering
\small
\begin{tabularx}{\textwidth}{|l|X|X|}
\hline
\textbf{} & \textbf{Gewichtung} & \textbf{Aus Graph entfernen}\\
  \hline
  \textbf{Beschreibung} & Die Strecken, auf denen sich ein Hindernis befindet wird höher gewichtet.  & Die Kante wird aus dem Graph entfernt.  \\
  \hline
  \textbf{Vorteile}  & \makecell{-Natürlicher Aspekt der Graphentheorie\\-Diese Strecken werden so vermieden} & \makecell{-simpel} \\
  \hline
  \textbf{Nachteile} & \makecell{-Gewichtung muss ermittelt werden} & \makecell{-möglicherweise langer Weg\\-gibt möglicherweise keine Strecke\\ ohne Hindernisse} \\
  \hline
  \textbf{Links} & \url{https://hyperskill.org/learn/step/5645} | 5 &  \url{https://www.freecodecamp.org/news/python-remove-key-from-dictionary/} | 4\\
  \hline
\end{tabularx}
\caption{Bewegliche Hindernisse erfassen Vergleich}
\label{table:analyse-path-compare-1}
\end{table}



\textbf{Blockierte Knoten erfassen}

Damit die blockierenden Hindernisse in unserem gespeicherten Netzwerk erfasst werden können, wurden folgende Methoden bewertet.

\begin{table}[H]
\centering
\small
\begin{tabularx}{\textwidth}{|l|X|X|}
\hline
\textbf{} & \textbf{Markieren} & \textbf{Aus gespeichertem Netz entfernen}\\
  \hline
  \textbf{Beschreibung} & Die blockierten Knoten werden markiert.  &  Die Knoten inklusive der Verbindungen werden entfernt. \\
  \hline
  \textbf{Vorteile}  & \makecell{-ürsprünglicher Graph noch vorhanden} & \makecell{-simpel} \\
  \hline
  \textbf{Nachteile} & \makecell{-überkompliziert} & \makecell{} \\
  \hline
\end{tabularx}
\caption{Blockierte Knoten erfassen Vergleich}
\label{table:analyse-path-compare-2}
\end{table}



\textbf{Wegfindung}

Die Implementation der Wegfindung kann auf folgende Arten umgesetzt werden. Die eigentlichen Algorithmen wurden in einem vorherigen Kapitel verglichen.

\begin{table}[H]
\centering
\small
\begin{tabularx}{\textwidth}{|l|X|X|}
\hline
\textbf{} & \textbf{Eigene Implementation} & \textbf{Externe Library verwenden}\\
  \hline
  \textbf{Beschreibung} & Der Algorithmus zur Wegfindung wird manuell implementiert.  & Es wird eine externe Library verwendet, um den Algorithmus auszuführen.  \\
  \hline
  \textbf{Vorteile}  & \makecell{-keine externen Abhängigkeiten\\-keine unnötigen Zusatzfunktionen} & \makecell{-simpel\\-wahrscheinlich effizienter\\-weniger Fehleranfällig} \\
  \hline
  \textbf{Nachteile} & \makecell{-komplex} & \makecell{-externe Abhängigkeit\\-schwergewichtig} \\
  \hline
  \textbf{Links} & \url{https://www.w3schools.com/dsa/dsa_algo_graphs_dijkstra.php} | 5 &\url{https://networkx.org/} | 4\\
  & \url{https://www.geeksforgeeks.org/python-program-for-dijkstras-shortest-path-algorithm-greedy-algo-7/} | 2&\url{https://pypi.org/project/astar/} | 3 \\
  \hline
\end{tabularx}
\caption{Wegfindung implementieren Vergleich}
\label{table:algorithm-compare}
\end{table}


\textbf{Zielauswahl}

Hier wurden Möglichkeiten verglichen, wie das Ziel ausgewählt werden kann im Simulator.

\begin{table}[H]
\centering
\small
\begin{tabularx}{\textwidth}{|l|X|X|}
\hline
\textbf{} & \textbf{Human Input} & \textbf{Random}\\
  \hline
  \textbf{Beschreibung} & Durch Input einer Person wird das Ziel gewählt. & Der Roboter wählt zufällig eines der Ziele aus.\\
  \hline
  \textbf{Vorteile}  & \makecell{-Simpel\\-einfaches Debugging} & \makecell{-Schnell \\ -gut für Performancemessung \\-deckt viele Graph-Varianten ab} \\
  \hline
  \textbf{Nachteile} & \makecell{-langsamer} & \makecell{-schwierigers Debugging} \\
  \hline
  \textbf{Links} &\url{https://www.w3schools.com/python/ref_func_input.asp} | 4& \url{https://docs.python.org/3/library/random.html} | 5\\
  &\url{https://python.land/data-processing/python-yaml} | 3&\\
  \hline
\end{tabularx}
\caption{Zielauswahl Vergleich}
\label{table:goal-compare}
\end{table}

\textbf{Clientseitige Kommunikation (\acrshort{i/o})}

Die Varianten eines Clientinterfaces wurden hier verglichen. Dieses dient dazu, dass die Fahrt des Roboters und dessen Berechnungen dem User mitgeteilt werden.

\begin{table}[H]
\centering
\small
\begin{tabularx}{\textwidth}{|l|X|X|}
\hline
\textbf{} & \textbf{GUI} & \textbf{CLI}\\
  \hline
  \textbf{Beschreibung} & Es gibt ein graphisches User Interface. &  Inputs und Outputs werden über eine Command Line gemacht.\\
  \hline
  \textbf{Vorteile}  & \makecell{-schön\\-benutzerfreundlich} & \makecell{-simpel} \\
  \hline
  \textbf{Nachteile} & \makecell{-mehr Aufwand} & \makecell{-eher unübersichtlich} \\
  \hline
  \textbf{Links} & \url{https://blog.hubspot.com/website/what-is-gui} | 3 & \url{https://www.w3schools.com/whatis/whatis_cli.asp} | 3\\
  &\url{https://docs.python.org/3/library/turtle.html} | 3 & \url{https://click.palletsprojects.com/en/8.1.x/} | 4\\
  &\url{https://www.pygame.org/news} | 5 &\\
  \hline
\end{tabularx}

\begin{tabularx}{\textwidth}{|l|X|X|}
\hline
\textbf{} & \textbf{Keine Inputs oder Outputs} & \textbf{TUI}\\
  \hline
  \textbf{Beschreibung} & Der Mensch kommuniziert nicht mit dem Roboter. Falls der Roboter das Ziel nicht erreicht wird eine Exception raised. &  Text-Based User Interface \\
  \hline
  \textbf{Vorteile}  & \makecell{-Einfach} & \makecell{-einfach\\-übersichtlich} \\
  \hline
  \textbf{Nachteile} & \makecell{-Sehr intransparent} & -wenig Ressourcen/Libraries\\
  \hline
  \textbf{Links} & \url{https://docs.python.org/3/tutorial/errors.html} | 3 & \url{https://en.wikipedia.org/wiki/Text-based_user_interface} | 2 \\
  \hline

\end{tabularx}
\caption{Kommunikation Vergleich}
\label{table:communication-compare}
\end{table}
