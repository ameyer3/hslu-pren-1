\subsection*{Prototyping}\label{prototyping}
\addcontentsline{toc}{subsection}{Prototyping}


Nachdem mithilfe der Nutzwertanalyse im Kapitel \nameref{nutzwertanalyse} ein grundlegendes Gesamtkonzept festgelegt wurde, wird nun an Prototypen der einzelen Teilfunktionen gearbeitet. Das Ziel ist es, die einzelnen Teilfunktionen zu testen und möglichst viele Fehler frühzeitig zu erkennen. Detaillierte Prototypen helfen, die bereits bekannten Risiken besser einzuschätzen und neue Risiken frühzeitig zu erkennen.

Das Prototyping soll mehrere Endprodukte hervorbringen. Es werden physische Prototypen von verschiedenen Teilbereichen erstellt, damit in \acrshort{pren2} parallel an unterschiedlichen Teilfunktionen gearbeitet werden kann. Alle physischen Endprodukte sind am Ende des Prototyping Kapitels in Kapitel \nameref{prototyp-products} aufgelistet.

\subsubsection*{Greifer Prototyp I}
\label{subsubsection:gripper-prototype-1}
\addcontentsline{toc}{subsubsection}{Greifer Prototyp I}



Der Aufbau und die Funktionsweise des Greifers werden in Kapitel \ref{subsubsection:Hindernisse bewegen} erläutert. Um die Auslegung des Greifers (Kapitel \nameref{subsubsection:gripper-calculations}) zu validieren und sein Funktionsprinzip zu überprüfen, wurde ein Prototyp gebaut. Nachfolgend wurden die Ergebnisse aus den Tests am Protoyp I dokumentiert.

\textbf{Aufbau}

 Der Prototyp besteht, abgesehen von Befestigungselementen, aus 3D-Druck Komponenten. Der Aufbau wurde so gewählt, dass Iterationen an Einzelteilen schnell realisiert und verbaut werden können (Abb. \ref{fig:gripper-prototype-1-trimetric-notes}). 

\begin{figure}[H]
\centering
\includegraphics[width=1.0\textwidth]{assets/greifer-prototyp/Greifer_Trimetrisch_Notes.png}
\caption{Greifer Prototyp I}
\label{fig:gripper-prototype-1-trimetric-notes}
\end{figure}

Der Prototyp wird manuell über einen Hebel bedient, welcher anstelle des Servomotors in der Konstruktion verbaut ist. Diese Entscheidung wurde getroffen, da das ausgelegte Motordrehmoment durch mögliche Reibung in den Gelenken nicht ausreichen könnte. Mit dem Hebel kann das ausgelegte Drehmoment einfach durch Aufbringen einer bekannten Last validiert werden.

Die Lagerpunkte des Gestänges wurden für den Prototyp an ein Gestell angebracht (Abb. \ref{fig:gripper-frame-mounting-points}). So kann der Greifer unabhängig vom Fahrzeug getestet werden. Die Abstände zwischen den Lagerpunkten sowie zum Boden sollen für den finalen Greifer dieselben sein. Auch die Dimensionen des Gestänges sollen beibehalten werden (Abb. \ref{fig:gripper-linkage-dimensions}). Somit bleiben die von der Geometrie abhängigen Kräfte dieselben und Ergebnisse aus Prototypentests können auf spätere Iterationen übertragen werden. 

\begin{figure}[H]
\centering
\begin{subfigure}{0.7\textwidth}
\centering
\includegraphics[width=\textwidth]{assets/greifer-prototyp/Greifer_Dimensionen_Lagerpunkte.png}
\caption{Gestell mit Lagerpunkten}
\label{fig:gripper-frame-mounting-points}
\end{subfigure}
\begin{subfigure}{0.7\textwidth}
\centering
\includegraphics[width=\textwidth]{assets/greifer-prototyp/Greifer_Dimensionen_Arme.png}
\caption{Dimensionen des Gestänges}
\label{fig:gripper-linkage-dimensions}
\end{subfigure}
\caption{Dimensionen Prototyp}
\label{gripper-prototype-dimensions}
\end{figure}

Die gelenkigen Verbindungen im Prototyp wurden mit M3 Schrauben realisiert. Dies bedeutet, das in den Gelenken die Gewinde der Schrauben direkt am Kunststoff der 3D-Druckteile reiben. Eine solche Lagerung hat erhöhte Reibung und ungenaue Toleranzen zur Folge, wurde aber bewusst gewählt, da sie kostengünstig und schnell realisiert werden kann. zur Verminderung der Reibung wurden alle Gelenke mit \acrfull{ptfe} Trockenschmiermittel behandelt.

\textbf{Ziele}

Mit dem Prototyp sollen folgende Anforderungen getestet werden (Ergebnisse in Tabelle \ref{tab:test-gripper-prototype-1}):
\begin{enumerate}
    \item Reibung in den Gelenken ist ausreichend gering, Gestänge lässt sich leichtgängig bewegen.
    \item Das Gestänge verklemmt im Betrieb nicht (keine overconstraints).
    \item Das Hindernis wird um mindestens 7.5mm angehoben.
    \item Hindernis kann zuverlässig angehoben werden (50 Testzyklen).
    \item Hindernis kann auch bei bis zu 15\textdegree\ Schrägstellung angehoben werden (50 Testzyklen in verschiedenen Winkeln).
    \item Hindernis wird $\pm$ 2mm an dieselbe Stelle zurückgesetzt (50 Testzyklen).
    \item Hindernis rutscht nicht aus dem Greifer bei Vibrationen (Hinderniss klemmen, Prototyp aufheben und schütteln, Hinderniss darf nicht rutschen).
    \item Nötiges Drehmoment zum Anheben ist maximal gleich gross wie das ausgelegte Drehmoment (20 Ncm).
    \item Die Greifbacke liegt in geöffnetem Zustand min. 5mm über dem Hindernis.
\end{enumerate}
\newpage

\textbf{Messungen und Beobachtungen}

\begin{table}[H]
\centering
\small
\begin{tabularx}{\textwidth}{|c|X|X|X|l|}
        \hline
        \textbf{Index} & \textbf{Kurzbeschreibung} & \textbf{Kriterium zur Erfüllung} & \textbf{Messergebnisse} & \textbf{Bewertung} \\
        \hline
        1 & Reibung in den Gelenken & Gelenke sind leichtgängig & Das Gestänge lässt sich leicht mit einem Finger bewegen & Test erfüllt \\ \hline
        2 & Kein Verklemmen des Gestänges & Keine overconstraints festgestellt & Das Gestänge verklemmt nicht & Test erfüllt \\ \hline
        3 & Hindernis wird angehoben & Hubhöhe $\geq$ 7.5 mm  &  9.3 $\pm$0.2 mm & Test erfüllt\\ \hline
        4 & Zuverlässiges Anheben & 50 Zyklen erfolgreich & 50/50 Zyklen erfolgreich & Test erfüllt \\ \hline
        5 & Anheben bei Schrägstellung & 50 Zyklen bei bis zu 15\textdegree\ erfolgreich & 50/50 Zyklen erfolgreich & Test erfüllt \\ \hline
        6 & Rücksetzen des Hindernisses & Rücksetzgenauigkeit $\pm$ 2 mm in 50 Zyklen& $\pm$1 mm in 50 Zyklen & Test erfüllt \\ \hline
        7 & Kein Rutschen bei Vibrationen & Hindernis bleibt fixiert & Hindernis verrutsch auch bei starkem Schütteln nicht & Test erfüllt \\ \hline
        8 & Max. Drehmoment  & Gemessenes Drehmoment $\leq$ 20 Ncm & 17 Ncm & Test erfüllt \\ \hline
        9 & Greifbacke über Hindernis & Abstand $\geq$ 5 mm in geöffnetem Zustand & 12 mm & Test erfüllt \\ \hline
\end{tabularx}
    \caption{Testergebnisse Greifer Prototyp I}
\label{tab:test-gripper-prototype-1}
\end{table}


\textbf{Fazit}

Alle Anforderungen sind erfüllt. Somit wurde entschieden am Greifer Design festzuhalten. Optimierungsbedarf besteht in folgenden Punkten: 
\begin{itemize}
    \item Die Backen, welche das Hindernis klemmen sind manchmal nicht korrekt ausgerichtet, da sie drehbar gelagert sein müssen. Dies führt dazu dass das Hindernis nicht immer gleich angehoben wird. Die gestellten Anforderungen betreffend dem Anheben wurden zwar erfüllt, ein Risiko, dass nicht korrekt angehoben wird besteht dennoch.
    \item  Die Vorspannfeder berührt das Gestell und wird deswegen nicht korrekt gespannt. Auch hier sind die Anforderungen trotzdem erfüllt, das Problem ist jedoch leicht durch eine Aussparung zu beheben.
\end{itemize}



\textbf{Weiteres Vorgehen}

Da das ausgelegte Motordrehmoment mit dem Prototyp I validiert werden konnte, wird nun ein passend dimensionierter Servomotor bestellt. Der manuell betätigte Greifer Prototyp I soll auf den Betrieb mit Servomotor zum Greifer Prototyp II umgebaut werden. Zusätzlich werden Endschalter zur Erkennung des Hindernisses installiert. Die im Fazit genannten Probleme sollen behandelt werden.
Bei den Tests am Prototyp II sollen die Geschwindigkeit des Anhebevorgangs und die Erkennung des Hindernisses durch die Endschalter im Fokus stehen.


\subsubsection*{Greifer Prototyp II}
\label{subsubsection:gripper-prototype-2}
\addcontentsline{toc}{subsubsection}{Greifer Prototyp II}


Der Prototyp II des Greifers ist eine umgebaute Version vom Protoyp I (Kapitel \nameref{subsubsection:gripper-prototype-1}). Folgende Anpassungen wurden vorgenommen: Der Greifer wurde mit einem Servomotor ausgestattet, welcher in Kapitel \nameref{subsubsection:gripper-calculations} vorausgelegt und durch den Greifer Prototyp I validiert wurde. An den Klemmbacken wurden zwei Endschaltern montiert. Sie sollen bei Berührung mit dem Hindernis den Hebevorgang automatisch auslösen. Des weiteren wurden die bei den Tests am Prototyp I erkannten Probleme behandelt.

\textbf{Aufbau}

\begin{figure}[H]
\centering
\includegraphics[width=1.0\textwidth]{assets/greifer-prototyp/Greifer_Prototyp_2_trimetrisch.png}
\caption{Greifer Prototyp II}
\label{fig:gripper-prototype-2-trimetric}
\end{figure}

Der grundlegende Aufbau des Prototyps bleibt bis auf den Servomotor und die Endschalter zur Erkennung des Hindernisses, derselbe (Abb. \ref{fig:gripper-prototype-2-trimetric}). Alle Im Kapitel \nameref{subsubsection:gripper-prototype-1} aufgeführten Dimensionen wurden beibehalten. Bei den Tests am Prototyp I wurde festgestellt, dass die Vorspannfeder am Rahmen verkantet. Dieses Problem wurde mit einer entsprechenden Aussparung am Rahmen gelöst. Weiter wurden die Klemmbacken überarbeitet, damit die Endschalter montiert werden können. Die drehbar gelagerten Klemmbacken verfügen über Anschläge welche verhindern sollten, dass sie sich zu weit nach unten drehen. Der Winkel dieser Anschläge wurde für den Prototyp II verkleinert, da die alten Klemmbacken sich beim Prototyp I manchmal zu weit drehten und das Hindernis nicht mehr richtig angehoben wurde (Abb. \ref{fig:jaw_iteration}).

\begin{figure}[H]
\centering
\begin{subfigure}{0.49\textwidth}
\centering
\includegraphics[width=0.5\textwidth]{assets/greifer-prototyp/Backe_alt_trimetrisch.png}
\caption{Klemmbacke Prototyp I}
\label{fig:jaw_old_trimetric}
\end{subfigure}
\begin{subfigure}{0.49\textwidth}
\centering
\includegraphics[width=0.95\textwidth]{assets/greifer-prototyp/Backe_neu_trimetrisch.png}
\caption{Klemmbacke Prototyp II}
\label{fig:jaw_new_trimetric}
\end{subfigure}
\begin{subfigure}{0.49\textwidth}
\centering
\includegraphics[width=0.8\textwidth]{assets/greifer-prototyp/Backe_alt_schnitt.png}
\caption{Anschlagwinkel Prototyp I}
\label{fig:jaw_old_section}
\end{subfigure}
\begin{subfigure}{0.49\textwidth}
\centering
\includegraphics[width=0.8\textwidth]{assets/greifer-prototyp/Backe_neu_schnitt.png}
\caption{Anschlagwinkel Prototyp II}
\label{fig:jaw_new_section}
\end{subfigure}
\caption{Änderungen an den Klemmbacken}
\label{fig:jaw_iteration}
\end{figure}

Der Servomotor des Greifers wird mit einem Arduino Nano Mikrokontroller gesteuert. Wenn einer der beiden Endschalter an den Backen durch Berührung mit dem Hindernis betätigt wird, wird der Anhebevorang ausgelöst. Der Servomotor dreht um ca. 90\textdegree,  
um Das Hindernis anzuheben. Wird ein am Mikrokontroller angeschlossener Taster gedrückt, wird das Hindernis wieder abgesetzt. Alle Tests werden auf dem Fliesenboden der Mensa durchgeführt, wo in PREN2 der Wettkampf stattfinden wird.\\

\textbf{Ziele}

Mit dem Prototyp II sollen folgende Anforderungen getestet werden (Ergebnisse in Tabelle \ref{tab:test-gripper-prototype-2}):

\begin{enumerate}
    \item Der Anhebe- und Absetztvorgang dauert jeweils nicht länger als 2.5 Sekunden.
    \item Der Servomotor kann das Hinderniss zuverlässig anheben (50 Testzyklen)
    \item Der Servomotor fährt die Vorgegebenen Winkel zuverlässig an ($\pm$ 2\textdegree unter Last in 50 Testzyklen)
    \item Das Haltemoment des Servomotors ist ausreichend (Servomotor muss Hinderniss min. 2min in Angehobenem Zustand halten)
    \item Das Hindernis löst die Endschalter aus ohne dabei verschoben zu werden (50 Testzyklen)
    \item Das Hindernis löst die Endschalter bei bis zu 15\textdegree  Schrägstellung aus, ohne dabei verschoben zu werden (50 Testzyklen unter verschiedenen Winkeln)
\end{enumerate}

Es ist zu beachten, dass Anforderungen welche bereits am Prototyp I geprüft und erfüllt wurden, hier nicht mehr aufgeführt sind.\\

\textbf{Messungen und Beobachtungen}

\begin{table}[H]
\centering
\small
\begin{tabularx}{\textwidth}{|c|X|X|X|l|}
        \hline
        \textbf{Index} & \textbf{Kurzbeschreibung} & \textbf{Kriterium zur Erfüllung} & \textbf{Messergebnisse} & \textbf{Bewertung} \\
        \hline
        1 & Anhebe- und Absetztdauer & Vorgang dauert jeweils \(\leq 2.5\) Sekunden & 1.0 \pm0.1 Sekunden & Test erfüllt \\
        \hline
        2 & Zuverlässiges Anheben & 50 Testzyklen erfolgreich & 50/50 Zyklen erfolgreich & Test erfüllt \\
        \hline
        3 & Zuverlässige Winkelansteuerung & max. \pm 2\textdegree\  unter Last in 50 Testzyklen & \pm 0.5\textdegree & Test erfüllt \\
        \hline
        4 & Ausreichendes Haltemoment & Hindernis bleibt min. 2min angehoben & \geq 15min & Test erfüllt \\
        \hline
        5 & Auslösen der Endschalter & Hindernis rutscht nicht, Endschalter lösen korrekt aus (50 Zyklen) & 38/50 erfolgreich& Test NICHT erfüllt \\
        \hline
        6 & Auslösen der Endschalter bei Schrägstellung & Hindernis rutscht nicht bei bis zu 15\textdegree\ Schrägstellung, Endschalter lösen korrekt aus (50 Zyklen) & 31/50 erfolgreich & Test NICHT erfüllt \\
        \hline
\end{tabularx}
    \caption{Testergebnisse Greifer Prototyp II}
\label{tab:test-gripper-prototype-2}
\end{table}

\textbf{Fazit}\\

Bis auf die Anforderungen Bezüglich der Hindernisserkennug mittels Endschalter (Anforderungen 5\&6) wurden alle Testbedingungen erfüllt. Die Endschalter, welche bei Berührung mit dem Hindernis den Hebevorgang auslösen sollen, müssen mit einer gewissen Kraft gedruckt werden um betätigt zu werden. Diese Kraft kann je nach Anfahrwinkel und Geschwindigkeit des Greifers bezüglich dem Hindernis nicht überwunden werden. Das Hindernis wird dann über den relativ glatten Fliesenboden geschoben, bevor die Schalter auslösen. Dabei wird das Hindernis zwar angehoben, würde aber beim Absetzten nicht mehr in seiner Ursprünglichen Position landen.

Die kleineren Probleme welche im Fazit zum Prototyp I (Kapitel \nameref{subsubsection:gripper-prototype-1}) aufgeführt wurden, konnten erfolgreich behoben werden.

\textbf{Weiteres Vorgehen}\\

Als nächstes müssen die oben aufgeführten Probleme mit den Endschaltern behoben werden. Folgende Ansätze sollen in Betracht gezogen werden:
\begin{itemize}
    \item Kleinere bzw. Endschalter mit geringerem Auslösegewicht verbauen
    \item Die Hebel an den Endschaltern verlängeren, um die nötige Kraft zur Auslösung zu reduzieren
    \item Einen Zusätlichen Ultraschallsensor am Heck des Fahrzeugs verbauen um die Distanz des Hindernisses festzustellen
    \item Auf Erkennung des Hindernisses am Heck des Fahrzeugs verzichten und stattdessen durch die Encoder der Antriebsmotoren die Korrekte Ausrichtung ab Erkennung des Hindernisses an der Vorderseite des Fahrzeuges Hardcoden.
\end{itemize}

Wenn die Probleme mit den Endschaltern behoben sind, soll der Greifer in das Fahrzeug implementiert werden. Hierbei können alle Komponenten des Prototyp II, abgesehen vom Gestell an welchem dieser montiert ist, übernommen werden. Das spart Zeit, Ressourcen und ist nachhaltiger als die Teile neu zu Fertigen.


\subsubsection*{Auslegung Greifer}
\label{subsubsection:gripper-calculations}
\addcontentsline{toc}{subsubsection}{Auslegung Greifer}


Nachfolgend sind die ausführlichen Berechnungen aufgeführt, nach welchen der Greifer ausgelegt wurde.

\includepdf[pages=-]{assets/greifer-prototyp/Auslegung_Greifer.pdf}

\subsubsection*{Fahrwerk}
\addcontentsline{toc}{subsubsection}{Fahrwerk}


Auf Basis des Gesamtkonzepts wurde anschliessend ein Prototyp für das Fahrwerk konstruiert. Die Baugruppe Prototyp Fahrwerk beinhaltet alle für die selbständige Fortbewegung notwendigen Elemente wie Motoren, Akkus, Liniensensoren und Steuerungseinheiten.  Bei diesem Prototyp stand der einfache und zweckmässige Aufbau im Vordergrund. Bei der Grundplatte wurde darauf geachtet das verschiedene  Versionen von Systemen einfach aufgebaut und ausgetauscht werden können. Ein flexibeler Prototyp ist Ressourcenschonend. Mehr Informationen dazu gibt es im Kapitel \ref{section:Nachhaltigkeit} Nachhaltigkeit. 

In progress...
