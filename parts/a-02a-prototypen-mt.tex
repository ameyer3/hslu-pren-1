\subsection{Prototyping}

Nachdem ein grundlegendes Gesamtkonzept festgelegt wurde, wird nun möglichst viel Prototyping durchgeführt. Das Ziel ist es, zu testen, ob das ermittelte Konzept funktionieren könnte oder ob es überarbeitet werden muss. So kann mit den Risiken, die im vorhergehenden Kapitel ermittelt wurden, besser umgegangen werden.

\subsubsection{Greifer}

Der Aufbau und die Funktionsweise des Greifers werden in Kapitel \textbf{(Kapitel Greifer verlinken)} erläutert. Der Greifmechanismus wurde als Prototyp realisiert.



\subsubsection{Fahrwerk}

Auf Basis der Nutzwertanalyse wurde anschliessend ein Prototyp für das Fahrwerk konstruiert. Bei diesem Prototyp stand der einfache und zweckmässige Aufbau im Vordergrund. Beim der Grundplatte wurde darauf geachtet das verschiedene  Versionen von Systemen einfach aufgebaut und ausgetauscht werden können. Ein flexibel Aufgebuter Prototyp unterstützt uns dabei unsere Nachhaltigkeits ziele zu erreichen. Mehr Informationen dazu im Kapitel \textbf{Nachhaltigkeit (verlinken !!!!).} 

\subsubsection{...}

