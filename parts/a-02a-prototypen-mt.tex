\subsection{Prototyping}

Nachdem mithilfe der Nutzwertanalyse im Kapitel \ref{nutzwertanalyse} ein grundlegendes Gesamtkonzept festgelegt wurde, wird nun an Prototypen der einzelen Teilfunktionen gearbeitet.   Das Ziel ist es, die einzelnen Teilfunktionen zu Testen und möglichst viele Fehler frühzeitig zu erkennen. Detaillierte Prototypen helfen die bereits bekannten Risiken besser einzuschätzen und neue Risiken frühzeitig zu erkennen.

\subsubsection{Greifer}
\label{subsubsection:Greifer}


Der Aufbau und die Funktionsweise des Greifers werden in Kapitel erläutert. Der Greifmechanismus wurde als Prototyp realisiert.



\subsubsection{Fahrwerk}

Auf Basis der Nutzwertanalyse wurde anschliessend ein Prototyp für das Fahrwerk konstruiert. Bei diesem Prototyp stand der einfache und zweckmässige Aufbau im Vordergrund. Beim der Grundplatte wurde darauf geachtet das verschiedene  Versionen von Systemen einfach aufgebaut und ausgetauscht werden können. Ein flexibel Aufgebuter Prototyp unterstützt uns dabei unsere Nachhaltigkeits ziele zu erreichen. Mehr Informationen dazu im Kapitel \textbf{Nachhaltigkeit (verlinken !!!!).} 

\subsubsection{...}

