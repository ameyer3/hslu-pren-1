\subsection{Prototyping}

Nachdem mithilfe der Nutzwertanalyse im Kapitel \ref{nutzwertanalyse} ein grundlegendes Gesamtkonzept festgelegt wurde, wird nun an Prototypen der einzelen Teilfunktionen gearbeitet. Das Ziel ist es, die einzelnen Teilfunktionen zu Testen und möglichst viele Fehler frühzeitig zu erkennen. Detaillierte Prototypen helfen die bereits bekannten Risiken besser einzuschätzen und neue Risiken frühzeitig zu erkennen.

\subsubsection{Greifer}
\label{subsubsection:Greifer}


Der Aufbau und die Funktionsweise des Greifers werden in Kapitel \ref{subsubsection:Hindernisse bewegen} erläutert. Der Greifmechanismus wurde ohne Servo als Prototyp realisiert. Am Prototyp konnten Kräfte gemessen werden um anschließend Servo antrieb auszulegen. 



\subsubsection{Fahrwerk}

Auf Basis des Gesamtkonzepts wurde anschliessend ein Prototyp für das Fahrwerk konstruiert. Die Baugruppe Prototyp Fahrwerk beinhaltet alle für die selbständige Fortbewegung notwendigen Elemente wie Motoren, Akkus, Liniensensoren und Steuerungseinheiten.  Bei diesem Prototyp stand der einfache und zweckmässige Aufbau im Vordergrund. Bei der Grundplatte wurde darauf geachtet das verschiedene  Versionen von Systemen einfach aufgebaut und ausgetauscht werden können. Ein flexibeler Prototyp ist Ressourcenschonend. Mehr Informationen im Kapitel \ref{section:Nachhaltigkeit} Nachhaltigkeit. 
