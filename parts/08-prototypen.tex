\section{Prototyping}

Nachdem ein Gesamtkonzept geplant wurde, wird nun möglichst viel Prototyping durchgeführt. Das Ziel ist es, zu testen, ob das ermittelte Konzept funktionieren könnte oder ob es überarbeitet werden muss. So kann mit den Risiken, die im vorhergehenden Kapitel ermittelt wurden, besser umgegangen werden.

\subsection{...}

\subsection{Kürzester Weg finden}

PLACEHOLDER
-> lightweight, schnelligkeit egal weil nur 8 Knoten, Tests von Lukas mit Screenshot (ev. Code in Anhang)

\subsection{Bilderkennung}

\subsection{Simulator}

Das Erarbeiten des Konzeptes des Simulators, der Implementierung und des Gebrauchs wird in diesem Kapitel festgehalten.

Nach der Nutzwertanalyse war das grundlegende Konzept klar und es konnte mit der Implementierung begonnen werden.

- GitHub, Issues sammeln, Festlegen Git ettiquete (aka MRs)

- Struktur mit Klassen (ERD?): Robot, Reader, Calculator, GUI

- Roboter liest Graph, ueberprueft Nodes und Barrieren

- Calculator holt kürzester Weg, Roboter geht zu nächsten Node

- Poetry

- GUI Roboter bewegt sich

- Die einzelnen Kanten werden identifiziert mit Sortierung + Bilder mit Grad?