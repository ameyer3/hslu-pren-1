\subsubsection{Motoren}

Test mit variabler Spannung und Duty Cycle

Nach ersten Berechnungen welche man für die Motoren gemacht hat konnte man den ersten Motor bestellen und testen. Grössen wie das Drehmoment und die Spannung waren bei dieser Auswahl entscheidend.

Bei dem Test wurden zwei verschiedene Spannungen getestet um den Motor mit einem PWM zu speisen. Zuerst wurde der Test mit der Spannung von 5V durchgeführt. Der Duty Cycle wurde langsam erhöht und man konnte feststellen, dass der Motor bei 58\% des Duty Cycles angefangen hat zu drehen. Ist der Motor angesprungen konnte man bis 44\% runter drehen bis der Motor aufhörte zu drehen. Die zweite Testsequenz erfolgte genau gleich nur mit einer Spannung von 12V. Dabei konnte man den Duty Cycle bis 33\% erhöhen, dann fing der Motor an zu drehen. Drehte man die Einschaltzeit des PWM wieder runter drehte der Motor bis zu 12\% Duty Cycle noch weiter. Da der Motor möglichst genau sein soll müssen die Umdrehungen fein einstellbar sein. Aufgrund der Motor bei 5V sehr lange hat um den Anlaufstrom zu erreichen wird der Motor schon zu schnell drehen um eine feine Steuerung garantieren zu können. Gründe wie Streuungsverluste und die Bauart des Motors sorgen dafür das zum Anlaufen ein höherer Strom notwendig ist. Damit man aber trotzdem mit der minimalen Drehzahl die Motoren starten kann, wird der Motor mit 12V gestartet. Für einen kurzen Moment schaltet man das PWM mit 12V ein, bis der Motor anläuft. Sobald der Motor dreht, wird die Spannung auf 5V gesenkt und man kann der tiefere Duty Cycle nutzen.

\begin{table}[h!]
\centering
\begin{tabular}{|c|c|c|c|}
\hline
\textbf{Spannung} & \textbf{Einschaltzeit (Duty Cycle)} & \textbf{Ausschaltzeit (Duty Cycle)} \\ \hline
12 V & 33 \% & 12 \% \\ \hline
5 V  & 58 \% & 44 \% \\ \hline
\end{tabular}
\caption{Vergleich der Duty-Cycle-Werte bei 12 V und 5 V}
\label{tab:dutycycle}
\end{table}

\subsubsection{...}