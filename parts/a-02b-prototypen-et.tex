\subsubsection{Motoren}

Nach ersten Berechnungen für die Motoren wurde ein passender Motor ausgewählt, bestellt und anschliessend getestet. Entscheidende Faktoren bei der Auswahl waren das Drehmoment und die Betriebsspannung.

Der Test bestand darin, den Motor mit einer Pulsweitenmodulation (PWM) bei zwei unterschiedlichen Spannungen zu betreiben. Zunächst wurde der Motor mit einer Spannung von 5V getestet. Dabei wurde der Duty Cycle schrittweise erhöht, bis der Motor bei 58\% des Duty Cycles zu drehen begann. Nachdem der Motor angelaufen, konnte der Duty Cycle auf bis zu 44\% reduziert werden, bevor der Motor wieder stoppte.

In einer zweiten Testsequenz wurde die gleiche Vorgehensweise mit einer Spannung von 12V durchgeführt. Hier begann der Motor bei einem Duty Cycle von 33\% zu drehen. Beim schrittweisen Reduzieren der Einschaltzeit drehte der Motor bis zu einem Duty Cycle von 12\% weiter.

Da der Motor möglichst präzise gesteuert werden soll, ist eine feine Einstellung der Drehzahl erforderlich. Der Test zeigte jedoch, dass der Motor bei 5V nur verzögert anlief, was zu einer zu schnellen Drehbewegung führen könnte und die präzise Steuerung erschwert. Streuverluste sowie die Bauart des Motors erfordern einen höheren Anlaufstrom.

Um dennoch eine geringe Anfangsdrehzahl zu gewährleisten, wird der Motor initial mit 12V gestartet. Sobald der Motor angelaufen ist, wird die Spannung auf 5V reduzier, wodurch der Motor bei hohen Duty Cycle nicht Überlastet werdet.

\begin{table}[h!]
\centering
\begin{tabular}{|c|c|c|c|}
\hline
\textbf{Spannung} & \textbf{Einschaltzeit (Duty Cycle)} & \textbf{Ausschaltzeit (Duty Cycle)} \\ \hline
12V & 33\% & 12\% \\ \hline
5V  & 58\% & 44\% \\ \hline
\end{tabular}
\caption{Vergleich der Duty-Cycle-Werte bei 12V und 5V}
\label{tab:dutycycle}
\end{table}

\subsubsection{Tiny K22 Pinout}

Für die hardwarenahe Steuerung und Regelung wurde ein Tiny K22 Version 1.4 ausgewählt, basierend auf der Methode des Morphologischen Kastens (Verweis auf Morphologischen Kasten einfügen). Um die Software- und Hardwareauslegung des Projekts optimal zu gestalten, ist eine präzise Funktionszuweisung der einzelnen Pins erforderlich.

Das Tiny K22 PCB bietet insgesamt 28 Pins (Link zu Tiny-Bild). Diese Pins können flexibel als Timer, ADC, IIC, SPI, UART, Input- oder Output-Pins konfiguriert werden.

Im Rahmen des Projekts wurden 25 Pins verwendet (Link zu Pinout-Bild). Die zeitkritischen Funktionen werden direkt auf dem Tiny K22 verarbeitet. Dabei werden folgende Zuweisungen vorgenommen:

Encoder-Auswertung: Der Mikrocontroller nutzt die integrierte Quadratur-Encoder-Auswertung auf Timer 1 und Timer 2.
Linienerfassung: Diese Funktion belegt den gesamten Timer 0.
Motorsteuerung, Ultraschallsensor, Buzzer und Servomotor: Diese teilen sich Timer 3, da keine weiteren Timer zur Verfügung stehen. Dies führt dazu, dass die Motoren im hörbaren Frequenzbereich betrieben werden. Dieser Umstand ist jedoch akzeptabel, da keine Vorgaben zur maximalen Lautstärke existieren.
Zusätzlich wird ein IIC-Bus für zeitunkritische Funktionen verwendet. Zu diesen gehören die Ziellampe und die Zielauswahlschalter.

Durch diese Zuweisung wird sichergestellt, dass die zeitkritischen Aufgaben effizient verarbeitet werden, während gleichzeitig Flexibilität für zusätzliche Funktionen über den IIC-Bus gewährleistet bleibt.




\subsubsection{Software}

Das Programm welches man auf das Tiny K22 laden wird, wird in C geschrieben. Mit der Entwicklungsumgebung MCUXpressoIDE kann der Code geschrieben und kompiliert werden. Debuggen kann man das Programm ebenfalls, da auf dem Tiny K22 ein Debugger ist. Es können vom Modul Mikrocontroller Fundamentals bestehende Libraries verwendet werden können. Einige Anpassungen wie zum Beispiel die UART wurden bereits vorgenommen.



